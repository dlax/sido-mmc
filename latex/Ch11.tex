\chapter{Thermoélasticité linéaire}\label{chap:Ch11}
Jusqu'à présent, nous n'avons pas tenu compte de la variable température. Dans de nombreux problèmes, cependant, il est nécessaire de la prendre en compte (problèmes de contraintes thermiques, par exemple). 
\section{Lois de comportement}\label{sec:Ch11-1}
\subsection{La théorie thermoélastique}\label{ssec:Ch11-1.1}
Un matériau thermoélastique est un matériau dans lequel la seule source de dissipation est la conduction thermique.
Nous considérons une théorie de petites perturbations autour d'un état de référence à contraintes et déformations nulles et à la température de référence $\theta_0$.
Nous poserons 
\begin{equation}
    \theta=\theta_0 + \bar{\theta}
    \label{eq:Ch11-001}
\end{equation}
avec $ \bar{\theta}$ petit.
D'autre part, l'entropie n'étant définie qu'à une constante près, nous la prendrons nulle dans cet état de référence.
Finalement, $ \varepsilon_{ij}$, $\sigma_{ij}$, $\bar{\theta}$ et $\eta$ seront des variables de perturbation, et nous pourrons négliger les termes d'ordre supérieur par rapport à ces variables. 

Dans ces conditions, l'équation de conservation de l'énergie \eqref{eq:Ch01-047} et l'inégalité de Clausias-Duhem \eqref{eq:Ch01-057} s'écrivent 
\begin{equation}
    \rho_0 \frac{\partial e}{\partial t} = \sigma_{ij} \frac{\partial \varepsilon_{ij}}{\partial t} + r - q_{j,j}
    \label{eq:Ch11-002}
\end{equation}
\begin{equation}
    -\rho_0 \left(\frac{\partial e}{\partial t} - \theta\frac{\partial \eta}{\partial t}\right) + \sigma_{ij} \frac{\partial \varepsilon_{ij}}{\partial t} - \frac{1}{\theta_0} q_i \theta_{,i} \geq 0
    \label{eq:Ch11-003}
\end{equation}

En élasticité, $e$ dépendait de $\varepsilon_{ij}$.
En thermoélasticité, $e$ dépendra de $\varepsilon_{ij}$ et de $\eta$ 
\begin{equation}
    e=e(\varepsilon_{ij},\eta)
    \label{eq:Ch11-004}
\end{equation}
L'inégalité de Clausius-Duhem \eqref{eq:Ch11-003} donne alors 
\begin{equation}
    -\rho_0 \left(\frac{\partial e}{\partial t} - \theta \right) \frac{\partial \eta}{\partial t} + \left(\sigma_{ij} -\rho_0\frac{\partial e}{\partial \varepsilon_{ij}}\right)\frac{\partial \varepsilon_{ij}}{\partial t} - \frac{1}{\theta_0} q_i \theta_{,i} \geq 0
    \label{eq:Ch11-005}
\end{equation}
et, pour que la seule source de dissipation soit la conduction thermique, on doit avoir 
\begin{equation}
    \theta=\frac{\partial e}{\partial eta}\qquad \sigma_{ij} =\rho_0\frac{\partial e}{\partial \varepsilon_{ij}}
    \label{eq:Ch11-006}
\end{equation}
\begin{equation}
    q_i \theta_{,i}\leq 0
    \label{eq:Ch11-007}
\end{equation}
Comme dans le cas élastique, nous pouvons faire un développement en série de $e$ 
\begin{equation}
  \left\{
  \begin{aligned}
    \rho_0 e(\varepsilon_{ij},\eta) = \rho_0 e_0 & + a_{ij} \varepsilon_{ij} +\rho_0 a_0 \eta \\
                                                 & + \frac{1}{2} A_{ijkh} \varepsilon_{ij}\varepsilon_{kh} + h_{ij} \varepsilon_{ij} \eta + \frac{1}{2} m \eta^2
  \end{aligned}
  \right.
    \label{eq:Ch11-008}
\end{equation}
\begin{equation}
  \left\{
  \begin{aligned}
    \rho_0 \theta & = \rho_0 a_0 + h_{ij} \varepsilon_{ij} + m \eta \\
    \sigma_{ij}   & = a_{ij} + A_{ijkh} \varepsilon_{kh} + h_{ij} \eta
  \end{aligned}
  \right.
    \label{eq:Ch11-009}
\end{equation}
Or, dans la configuration de référence, on a $\varepsilon_{ij}=\eta=0$, $\sigma_{ij}=0$ et $\theta=\theta_0$.
On doit donc avoir $a_0 = \theta_0$, $a_{ij}=0$ et on peut aussi prendre $e_0=0$.
On obtient alors 
\begin{equation}
  \begin{aligned}
    \rho_0 e & = \rho_0 \theta_0 \eta + \rho_0 \bar{e}(\varepsilon_{ij},\eta) \\
    \rho_0 \bar{e} = \frac{1}{2} A_{ijkh} \varepsilon_{ij}\varepsilon_{kh} + h_{ij} \varepsilon_{ij}\eta + \frac{1}{2} m \eta^2
  \end{aligned}.
    \label{eq:Ch11-010}
\end{equation}
\begin{equation}
    \sigma_{ij} = \rho_0 \frac{\partial \bar{e}}{\partial \varepsilon_{ij}} = A_{ijkh} \varepsilon_{kh} + h_{ij} \eta \quad \rho_0 \bar{\theta} = \frac{\partial \bar{e}}{\partial \eta} = h_{ij} \varepsilon_{ij} + m \eta
    \label{eq:Ch11-011}
\end{equation}

Compte-tenu de \eqref{eq:Ch11-006}, l'équation \eqref{eq:Ch11-002} devient, après linéarisation 
\begin{equation}
    \rho_0 \theta_0 \frac{d \eta}{dt} = r - q_{j,j}
    \label{eq:Ch11-012}
\end{equation}

Pour compléter la théorie, il faut écrire une loi de conduction thermique, donnant le flux de chaleur en fonction du gradient de température.
En théorie linéaire, on prend la loi de Fourier 
\begin{equation}
    q_i = -K_{ij} \theta_{,j}
    \label{eq:Ch11-013}
\end{equation}
avec un tenseur de conduction $K_{ij}$ symétrique -- cette symétrie est expérimentalement bien vérifiée, et théoriquement bien fondée par les relations d'Onsager; elle n'est toutefois pas indispensable, on peut imaginer un tenseur de conduction non symétrique -- et défini positif -- d'après le second principe \eqref{eq:Ch11-007}.

Le formalisme \eqref{eq:Ch11-010}, \eqref{eq:Ch11-011} définit les contraintes et la température en fonction des déformations et de l'entropie, ceci étant lié au choix de l'énergie interne comme potentiel thermodynamique.
Une transformation de Legendre sur $e(\varepsilon_{ij},\eta)$ permettra d'écrire d'autres relations.
Par exemple, on peut introduire une «~enthalpie libre~»
\begin{equation}
 \left\{
  \begin{aligned}
    \rho_0 g(\sigma_{ij},\theta) = \sigma_{ij}\varepsilon_{ij} + \rho_0 \theta \eta - \rho_0 e \\
    \eta = \frac{\partial g}{\partial \theta} \qquad,\qquad \varepsilon{ij} = \rho_0 \frac{\partial g}{\partial \sigma_{ij}}
  \end{aligned}
  \right.
    \label{eq:Ch11-014}
\end{equation}
\begin{equation}
    \rho_0 g = \frac{1}{2} \Sigma_{ijkh} \sigma_{ij}\sigma_{kh} + l_{ij} \sigma_{ij} \bar{\theta} +\frac{1}{2} n \bar{\theta}^2
    \label{eq:Ch11-015}
\end{equation}
\begin{equation}
    \left\{
    \begin{aligned}
        \varepsilon_{ij} & = \Sigma_{ijkh} \sigma_{kh} + l_{ij} \bar{\theta} \\
        \rho_0 \eta      & = l_{ij} \sigma_{ij} + n \bar{\theta}
    \end{aligned}
    \right.
    \label{eq:Ch11-016}
\end{equation}
où les coefficients de la forme quadratique $\rho_0 g$ peuvent s'exprimer à partir de ceux de la forme $\rho_0 \bar{e}$ par les relations 
\begin{equation}
 \left\{
  \begin{aligned}
    n             & = & \left(m - h_{ij} \Lambda_{ijkl} h_{kl}\right)^{-1} \\
    l_{ij}        & = & -n \Lambda_{ijkl} h_{kl} \\
    \Sigma_{ijkl} & = & \Lambda_{ijkl} + n \Lambda_{ijpq}\Lambda_{klmn} h_{pq} h_{mn}
  \end{aligned}
  \right.
    \label{eq:Ch11-016}
\end{equation}
De même, on pourra exprimer $\varepsilon{ij}$ et $\bar{\theta}$ en fonction de l'enthalpie $h(\sigma_{ij},\eta)$, et $\sigma_{ij}$ et $\eta$ en fonction de l'énergie libre $\psi(\varepsilon_{ij},\bar{\theta})$. 

\subsection{Thermoélasticité classique}\label{ssec:Ch11-1.2} 
Dans le cas isotrope, les différents tenseurs prennent des formes simples.
Un tenseur du 4ème ordre prend la forme \eqref{eq:Ch05-022} et dépend donc de 2 coefficients.
Un tenseur du second ordre est sphérique et fait donc intervenir un coefficient.
La théorie fera donc intervenir 5 coefficients scalaires: 4 coefficients pour le potentiel thermodynamique \eqref{eq:Ch11-010} ou \eqref{eq:Ch11-015}, et 1 pour le tenseur de conduction $K_{ij}$.
En effet, \eqref{eq:Ch11-013} devient
\begin{equation}
    q_i = =k \theta_{,i} \quad , \quad k \geq 0
    \label{eq:Ch11-018}
\end{equation}
où $k$ est la conductivité thermique. 

Nous convenons désormais d'omettre les barres sur $e$ et $\theta$.
En particulier, $\theta$ désignera désormais non pas la température absolue, mais sa variation par rapport à la température de référence. 

Si $\alpha$ est la coefficient de dilatation linéaire du matériau, alors en l'absence de contraintes, une variation de température $\theta$ entraîne une dilatation thermique 
\begin{equation}
    \varepsilon_{ij} = \alpha \theta \delta_{ij}
    \label{eq:Ch11-019}
\end{equation}
D'autre part, à la température de référence $\theta_0$, le matériau se comporte comme un matériau élastique isotrope, avec un module d'Young $E$ et un coefficient de Poisson $\nu$, mesurés en conditions isothermes.
Nous pouvons donc écrire 
\begin{equation}
    \varepsilon_{ij} = \frac{1+\nu}{E}\sigma_{ij} - \frac{\nu}{E}\sigma_{kk}\delta_{ij} + \alpha \theta \delta_{ij}
    \label{eq:Ch11-020}
\end{equation}
\begin{equation}
    \sigma_{ij} = \lambda \varepsilon_{kk} \delta_{ij} + 2 \mu \varepsilon_{ij} -3 K \alpha \theta \delta_{ij}
    \label{eq:Ch11-021}
\end{equation}
où $ 3 K = 3 \lambda + 2 \mu $ est le module de rigidité à la compression isotherme, et $\lambda$ et $\mu$ sont les coefficients de Lamé isothermes donnés par \eqref{eq:Ch05-033} à partir de $E$ et $\nu$.

Quant au coefficient de $\theta^2$ dans le potentiel thermodynamique, on peut le relier à la chaleur spécifique à contraintes ou déformations constantes 
\begin{equation}
    c = c_{\varepsilon} = \theta_0 \left(\frac{\partial^2 e}{\partial \theta^2}\right)_{\varepsilon_{ij}=\mbox{Cte}} \quad , \quad c_{\sigma} = \theta_0 \left(\frac{\partial^2 e}{\partial \theta^2}\right)_{\sigma_{ij}=\mbox{Cte}}
    \label{eq:Ch11-022}
\end{equation}
ce qui nous permet d'écrire, à partir de \eqref{eq:Ch11-016}, 
\begin{equation}
    \rho_0\eta = \alpha \sigma_{ii} + \frac{\rho_0 c_{\sigma}}{\theta_0}\theta
    \label{eq:Ch11-023}
\end{equation}
ou bien, en fonction de déformations, 
\begin{equation}
    \rho_0\eta = 3 K \alpha \varepsilon_{ii} + \frac{\rho_0 c}{\theta_0}\theta
    \label{eq:Ch11-024}
\end{equation}
et l'on obtient la relation 
\begin{equation}
    c_{\sigma} = c +\frac{9 K \alpha^2 \theta_0}{\rho_0}
    \label{eq:Ch11-025}
\end{equation}
relation entre les chaleurs spécifiques à déformations constantes et contraintes constantes. 

Ainsi, la théorie classique de la thermoélasticité fait intervenir 5 coefficients: deux coefficients élastiques isothermes, le coefficient de dilatation $\alpha$, la chaleur spécifique à déformations constantes $c$ et la conductibilité thermique $k$.

\section{Problèmes de thermoélasticité}\label{sec:Ch11-2}
\subsection{Problèmes aux limites}\label{ssec:Ch11-2.1}
Les équations de la thermoélasticité sont les équations du mouvement et l'équation de l'énergie \eqref{eq:Ch11-012}
\begin{equation}
    \rho_0 \frac{\partial^2 u_i}{\partial t^2} = \sigma_{ij,j}+f_i
    \label{eq:Ch11-026}
\end{equation}
\begin{equation}
    \rho_0 \theta_0 \frac{\partial \eta}{\partial t} = r - q_{j,j}
    \label{eq:Ch11-027}
\end{equation}
complétées par les lois de comportement, \eqref{eq:Ch11-018}, \eqref{eq:Ch11-021} et \eqref{eq:Ch11-024} par exemple, des conditions initiales 
\begin{equation}
 \left\{
  \begin{aligned}
    u_i(x,0) = u_i^0(x) \quad , \quad \frac{\partial u_i}{\partial t}(x,0) = V_i^0(x) \\
    \theta(x,0) = \theta^0(x)
  \end{aligned}
  \right.
    \label{eq:Ch11-028}
\end{equation}
et des conditions aux limites, qui sont les mêmes qu'en élasticité pour les variables mécaniques et qui, pour les variables thermiques, donnent, soit la température, soit le flux de chaleur.
Nous prendrons par exemple des conditions aux limites mixtes 
\begin{equation}
  \begin{aligned}
    u_i|_{S_u} = u_i^d \quad\quad  \sigma_{ij}n_j|_{S_f} = T_i^d\\
    \theta|_{S_\theta} = \theta^d \quad\quad  q_in_i|_{S_q} = q^d
  \end{aligned}
    \label{eq:Ch11-029}
\end{equation}
avec $S =S_u+S_f=S_{\theta}+S_q$.
Le problème est bien posé, et on peut démontrer un théorème d'unicité analogue à celui du paragraphe \ref{ssec:Ch06-1.2}.
 
Compte-tenu des lois de comportement \eqref{eq:Ch11-021}, \eqref{eq:Ch11-024} et \eqref{eq:Ch11-028}, les équations \eqref{eq:Ch11-026} et \eqref{eq:Ch11-027} donnent 
\begin{equation}
    \rho_0 \frac{\partial^2 u_i}{\partial t^2} = (\lambda +\mu) u_{k,ik} + \mu u_{i,kk} - 3 K \alpha \theta_{,i} +f_i
    \label{eq:Ch11-030}
\end{equation}
\begin{equation}
    \rho_0 c \frac{\partial \theta}{\partial t} = r + k\theta_{,ii} - 3 K \alpha \theta_0 \frac{\partial^2 u_i}{\partial x_i \partial t}
    \label{eq:Ch11-031}
\end{equation}

Dans de nombreux cas, on peut raisonnablement négliger le dernier terme de \eqref{eq:Ch11-031}.
On arrive ainsi à la «~thermoélasticité découplée~».
En effet, le déplacement disparaît de l'équation \eqref{eq:Ch11-031}, on peut donc déjà résoudre le problème thermique 
\begin{equation}
  \begin{aligned}
    \rho_0 c \frac{\partial \theta}{\partial t} = r + k\Delta \theta \\
    \theta|_{S_\theta} = \theta^d \quad\quad  q_in_i|_{S_q} = q^d \\
    \theta (x,0) = \theta^0(x)
  \end{aligned}
    \label{eq:Ch11-032}
\end{equation}
problème classique pour l'équation de la chaleur et qui permet de calculer la répartition de température indépendamment des déformations.
Une fois connu le champ de température, on peut revenir au problème mécanique 
\begin{equation}
  \begin{aligned}
    \rho_0 \frac{\partial^2 vec{u}}{\partial t^2} = (\lambda + \mu) \grad \dive \vec{u} + \mu \Delta \vec{u} +\vec{f} - 3 K \alpha \grad \theta \\
    u_i|_{S_u} = u_i^d \quad,\quad  \sigma_{ij}n_j|_{S_f} = T_i^d \\
    u_i(x,0)= u_i^0(x) \quad,\quad  \frac{\partial u_i}{\partial t}(x,0) = V_i^0(x)
  \end{aligned}
    \label{eq:Ch11-033}
\end{equation}
qui est un problème d'élasticité classique, le couplage thermoélastique se 
manifestant seulement par une modification des données $f_i$ et $T_i^d$. 

Nous avons formulé ici le problème dynamique mais l'on peut aussi envisager le problème statique: les dérivées par rapport au temps et les conditions initiales disparaissent.
On peut alors démontrer des principes variationnels analogues à ceux du cas élastique.
On peut également envisager un problème mécanique quasi-statique avec un problème thermique dynamique. 

\subsection{Un exemple}\label{ssec:Ch11-2.2}
A titre d'exemple, nous allons calculer les contraintes thermiques engendrées par l'échauffement d'une cavité sphérique de rayon $a$ dans un massif infini.
Nous considérons donc le problème défini par 
\begin{equation}
    f_i = 0 \quad,\quad r=0
    \label{eq:Ch11-034}
\end{equation}
\begin{equation}
    r \rightarrow \infty: \quad \theta \rightarrow 0 \quad,\quad \sigma_{ij}\rightarrow 0
    \label{eq:Ch11-035}
\end{equation}
\begin{equation}
    r =a: \quad \theta=\theta_1 \quad,\quad \sigma_{ij}n_j = 0
    \label{eq:Ch11-036}
\end{equation}

Il s'agit d'un problème statique et l'équation thermique \eqref{eq:Ch11-031} donne 
\begin{equation}
    \Delta \theta = 0
    \label{eq:Ch11-037}
\end{equation}
D'après la symétrie du problème, $\theta$ dépend uniquement de $r$, $\theta=\theta(r)$.
On a 
\begin{equation}
    \Delta \theta = \theta''(r) +\frac{2}{r}\theta'(r)
    \label{eq:Ch11-038}
\end{equation}
et l'équation \eqref{eq:Ch11-037} s'intègre en 
\begin{equation}
    \Delta \theta = \frac{A}{r} + B
    \label{eq:Ch11-039}
\end{equation}
où $A$ et $B$ sont deux constantes d'intégration que nous déterminons à partir des conditions aux limites pour $r= a$ et $r \rightarrow \infty$.
On obtient 
\begin{equation}
    \Delta \theta = \theta_1\frac{a}{r}
    \label{eq:Ch11-040}
\end{equation}
et nous avons résolu le problème thermique qui, en statique, est toujours découplé du problème mécanique. 

Pour résoudre le problème mécanique, nous partons de l'équation du mouvement \eqref{eq:Ch11-030} qui, en statique, donne, en supposant $\vec{f}=0$, 
\begin{equation}
    (\lambda+\mu) \grad \dive \vec{u} + \mu \Delta \vec{u} - 3K\alpha \grad \theta = 0
\end{equation}
ou bien, voir \eqref{eq:Ch05-025}, 
\begin{equation}
    (\lambda+2\mu) \grad \dive \vec{u} + \mu \rot \rot \vec{u} - 3K\alpha \grad \theta = 0
    \label{eq:Ch11-041}
\end{equation}
D'après la symétrie du problème, le déplacement est radial 
\begin{equation}
    u_i=g(r) x_i
    \label{eq:Ch11-042}
\end{equation}
et, comme on l'a vu au paragraphe \ref{ssec:Ch05-2.2}, $\rot \vec{u}$. L'équation s'intègre alors pour donner 
\begin{equation} 
\grad\left[(\lambda + 2 \mu) \dive \vec{u} - 3 K \alpha \theta \right] = 0
\end{equation}
\begin{equation}
    \dive \vec{u} = \frac{3K\alpha}{\lambda + 2 \mu} \theta + 3 A
    \label{eq:Ch11-043}
\end{equation}
où $A$ est une constante d'intégration.
Compte-tenu de \eqref{eq:Ch05-047} et de \eqref{eq:Ch11-040}, on obtient alors pour $g(r)$ l'équation différentielle suivante 
\begin{equation}
    3g(r)+\frac{g'(r)}{r} = 3A + \frac{3K\alpha\theta_1 a}{\lambda + 2 \mu}\frac{1}{r}
    \label{eq:Ch11-044}
\end{equation}
qui s'intègre pour donner 
\begin{equation}
    g(r) = A + \frac{B}{r^3}+\frac{3K\alpha\theta_1 a}{2(\lambda + 2\mu)}\frac{1}{r}
    \label{eq:Ch11-045}
\end{equation}

Il reste à déterminer les deux constantes d'intégration $A$ et $B$ en écrivant les conditions aux limites sur les contraintes.
Pour cela, nous posons 
\begin{equation}
    C_0 = \frac{3K\alpha\theta_1 a}{2(\lambda + 2\mu)}
    \label{eq:Ch11-046}
\end{equation}
et nous calculons 
\begin{equation}
    \varepsilon_{ij}=\left(A+\frac{B}{r^3}+\frac{C_0}{r}\right)\delta_{ij}-\left(\frac{3B}{r^3}+\frac{C_0}{r}\right)\frac{x_ix_j}{r^2}
    \label{eq:Ch11-047}
\end{equation}
et les déformations principales sont 
\begin{equation}
    \varepsilon_1 = \varepsilon_2 = A + \frac{B}{r^3} + \frac{C_0}{r} \quad,\quad \varepsilon_3 = A - \frac{2B}{r^3}
    \label{eq:Ch11-048}
\end{equation}
la valeur propre $\varepsilon_3$ étant associée à la direction radiale.
On peut ensuite calculer les contraintes par \eqref{eq:Ch11-021}, et la condition à l'infini donne directement $A= 0$.
On calcule ensuite $\sigma_3$ 
\begin{equation}
  \begin{aligned}
    \sigma_3 & = \frac{2C_0\lambda}{r} - \frac{4\mu B}{r^3} - 2(\lambda+2\mu)\frac{C_0}{r} \\
             & = - \frac{4\mu}{r}\left(\frac{B}{r^2}+C_0\right)
  \end{aligned}
    \label{eq:Ch11-049}
\end{equation}
et en écrivant que $\sigma_3$ est nul pour $r=a$, on obtient la valeur de $B$  
\begin{equation}
    B = - C_0 a^2
    \label{eq:Ch11-050}
\end{equation}
Soit finalement
\begin{equation}
  \left\{
  \begin{aligned}
    g(r)                & = \frac{3K\alpha \theta_1}{2(\lambda + 2\mu}\frac{a}{r}\frac{r^2-a^2}{r^2} \\
    \sigma_3            & = - \frac{6\mu K \alpha \theta_1}{\lambda + 2\mu} \frac{a}{r}\frac{r^2-a^2}{r^2}\\
    \sigma_1 = \sigma_2 & = - \frac{3\mu K \alpha \theta_1}{\lambda + 2\mu} \frac{a}{r}\frac{r^2-a^2}{r^2}
  \end{aligned}
  \right.
    \label{eq:Ch11-051}
\end{equation}
