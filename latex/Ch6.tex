\chapter{Élasticité classique}
\section{Les équations de l'élasticité}
\subsection{Problèmes reguliers}
Pour résoudre un problème d'élasticité, il faut donc trouver un champ de déplacements $u_i\left( x,t \right)$ et un champ de contraintes $\sigma_{ij}\left( x,t \right)$ vérifiant les équations du mouvement ou d'équilibre suivant que l'on s'intéresse au problème dynamique ou quasi-statique
\begin{equation}
    \sigma_{ij,j} + f_i = \rho \frac{\partial^2 u_i}{\partial t^2} \text{ ou } 0
    \label{eq:Ch06-001}
\end{equation}
et la loi de comportement
\begin{equation}
    \sigma_{ij} = A_{ijkl} \varepsilon_{kl}
    \label{eq:Ch06-002}
\end{equation}
où le tenseur des déformations est donné par
\begin{equation}
    \varepsilon_{ij} = \frac{1}{2} \left( u_{i,j} + u_{j,i} \right)
    \label{eq:Ch06-003}
\end{equation}
On obtient donc un système de 9 équations à 9 inconnues, et le problème sera «~bien posé~», c'est à dire admettra une solution unique, pourvu qu'on lui rajoute des CL (conditions aux limites) et éventuellement des CI (conditions initiales) adéquates.
Les CI donnent la position et la vitesse du milieu à l'instant 0
\begin{equation}
    u_i \left( x,0 \right) = u_{i}^0 (x) \quad, \quad \frac{\partial u_i}{\partial t} \left( x,0 \right) = V_i^0 \left( x \right)
    \label{eq:Ch06-004}
\end{equation}

Les différents types de CL que l'on peut rencontrer ont été discutées au paragraphe~\ref{ssec:Ch4-1.2}.
On définit classiquement
\begin{description}
    \item[Problème de type I.] Les déplacements sont donnés à la frontière
        \begin{equation}
            u_i{}_{|_{\partial \Omega}} = u_i^d
            \label{eq:Ch06-005}
        \end{equation}
    \item[Problème de type II.] Les efforts appliqués au solide sur la frontière sont donnés
        \begin{equation}
            \sigma_{ij} n_j{}_{|_{\partial \Omega}} = T_i^d
            \label{eq:Ch06-006}
        \end{equation}
\end{description}
Exemples: le réservoir sphérique au paragraphe~\ref{ssec:Ch04-1.1}, ou le bloc pesant du paragraphe~\ref{ssec:Ch04-1.2} avec la CL~\eqref{eq:Ch04-025}.

Plus généralement, on a affaire à un problème mixte pour lequel sur chaque partie de $\partial \Omega$ on donne
\begin{itemize}
    \item soit les efforts, exemple \eqref{eq:Ch04-008}; 
    \item soit les déplacements, exemple \eqref{eq:Ch04-012};
    \item soit certaines composantes du déplacement et les composantes complémentaires de l'effort, exemple \eqref{eq:Ch04-011}.
\end{itemize}

Un exemple type de problème mixte est celui où l'on se donne les déplacements sur une partie de la surface et les efforts sur la partie complémentaire
\begin{equation}
    u_i{}_{|_{S_u}} =  u_i^d \quad, \quad \sigma_{ij} n_j{}_{|_{S_f}} = T_i^d
    \label{eq:Ch06-007}
\end{equation}
avec $\partial \Omega = S_u + S_f$.
C'est par exemple le cas pour les deux problèmes du paragraphe~\ref{ssec:Ch04-1.2} avec condition d'adhérence, mais pour ces mêmes problèmes avec conditions de non frottement, les CL sur les bases donnent la composante du déplacement sur $x_3$ et les composantes de l'effort sur $x_1$, $x_2$.
De manière générale, nous introduisons la classe des problèmes réguliers, problèmes pour lesquels en tout point de la frontière $\partial \Omega$ sont données 3 composantes complémentaires de l'effort $T_i = \sigma_{ij} n_j$ ou du déplacement $u_i$.
Pour qu'un problème soit régulier, il faut que l'intégrale représentant le travail des efforts de contact puisse se décomposer en deux termes
\begin{equation}
    \iint_{\partial \Omega} \sigma_{ij} n_j u_i \ud S = T_f^d \left( u_i \right) + T_u^d \left( \sigma_{ij} \right)
    \label{eq:Ch06-008}
\end{equation}
le premier terme $T_f^d$ représentant le travail des efforts donnés dans le déplacement (inconnu), et le second le travail des efforts de contact (inconnus) dans les déplacements donnés.
Pour le problème mixte \eqref{eq:Ch06-007}, on a simplement
\begin{equation}
    \iint_{\partial \Omega} \sigma_{ij} n_j u_i \ud S = \underbrace{\iint_{S_u} \sigma_{ij} n_j u_i^d \ud S}_{T_u^d\left( \sigma_{ij} \right)} +  \underbrace{\iint_{\partial \Omega} T_i^d u_i \ud S}_{T_f^d\left( u_i \right)}
    \label{eq:Ch06-009}
\end{equation}
(les problèmes de type I et II sont des cas particuliers du problème mixte~\eqref{eq:Ch06-007}).
Pour les autres problèmes réguliers, cette décomposition est plus longue à écrire.
Par exemple, pour le problème du bloc pesant avec condition de non frottement \eqref{eq:Ch06-019}, on a
\begin{equation}
    \iint_{\partial \Omega} \sigma_{ij} n_j u_i \ud S = \underbrace{\iint_{S_l +S_1} T_i^d u_i \ud S - \iint_{S_0} \left( \sigma_{13}^d u_i + \sigma_{23}^d u_2 \right.}_{T_f^d\left( u_i \right)} + \underbrace{ \left.\sigma_{33} u_3^d \right)}_{T_u^d\left(\sigma_{ij} \right)} \ud S
    \label{eq:Ch06-010}
\end{equation}
Pour ce problème particulier, chacun des termes est nul d'après \eqref{eq:Ch04-018} et \eqref{eq:Ch04-019}, mais peu importe, l'essentiel est d'examiner ce qui est donné par les conditions aux limites, et de vérifier que l'on peut effectuer la décomposition \eqref{eq:Ch06-008} sans ambiguïté.
En particulier, il en résulte que, pour le problème homogène associé, c'est à dire pour le problème correspondant à toutes les données nulles, on a automatiquement
\begin{equation}
    \iint_{\partial \Omega} \sigma_{ij} n_j u_i \ud S = 0
    \label{eq:Ch06-011}
\end{equation}
c'est un moyen commode pour vérifier qu'un problème est régulier.

Cette notion de problème régulier est essentielle, car elle recouvre la formulation naturelle des conditions aux limites en Mécanique des Solides en général.
En élasticité, les problèmes réguliers sont des problèmes linéaires -- on peut donc superposer plusieurs solutions -- et qui permettent de démontrer un certain nombre de théorèmes généraux, notamment des théorèmes d'existence et d'unicité -- un problème régulier est bien posé et les théorème~ d'énergie qui feront l'objet d'un chapitre.

Il existe des problèmes non réguliers, comme par exemple les problèmes de frottement, par exemple le problème du lopin avec la CL \eqref{eq:Ch04-015}, ou les problèmes unilatéraux, comme le problème du bloc avec la CL \eqref{eq:Ch04-024}.
Dans les deux cas, il s'agit de CL non linéaires qui rendent le problème non linéaire, et donc beaucoup plus difficile.
Les liaisons élastiques donnent un exemple de problème linéaire non régulier.
Nous rencontrerons aussi des problèmes non réguliers par manque de données, mais il s'agit alors d'une non régularité superficielle, et qui ne nous gênera guère.

\subsection{Theorème d'unicité en dynamique}
Comme nous l'avons affirmé plus haut, un problème régulier est bien 
posé, c'est à dire admet une solution unique.
A titre d'exemple, nous allons démontrer le théorème d'unicité dans le cas dynamique.
Nous partons donc d'un problème dynamique régulier.
Pour fixer les notations, nous prendrons des CL mixtes de type \eqref{eq:Ch06-007}, mais la démonstration est valable, à des difficultés de notations près, pour tout problème régulier.
Nous cherchons donc $u_i\left( x,t \right)$, $\sigma_{ij}\left( x,t \right)$, $\varepsilon_{ij}\left( x,t \right)$, solution du problème suivant
\begin{equation}
    \left\{
    \begin{aligned}
        \rho \frac{\partial^2 u_i}{\partial t^2} &= \sigma_{ij,j} + f_i \\
        \sigma_{ij} &= A_{ijkl} \varepsilon_{kh} \\
        \varepsilon_{ij} &= \frac{1}{2} \left( u_{i,j} + u_{j,i} \right) \\
        u_i \left( x,0 \right) &= u_i^c \left( x \right) \quad & V_i\left( x,0 \right) = V_i^c\left( x \right) \\
        u_i{}_{|_{S_u}} &= u_i^d  \quad & \sigma_{ij} n_j{}_{S_f} = T_i^d
    \end{aligned}
    \right.
    \label{eq:Ch06-012}
\end{equation}
Au paragraphe~\ref{ssec:Ch01-2.1} nous avons démontré le théorème de l'énergie cinétique \eqref{eq:Ch01-029}, mais en élasticité on a d'après \eqref{eq:Ch05-006} et \eqref{eq:Ch05-008}
\begin{equation}
    \sigma_{ij} D_{ij} = \sigma_{ij} \frac{\ud \varepsilon_{ij}}{\ud t} = \frac{\ud w}{\ud t}
    \label{eq:Ch06-013}
\end{equation}
ce qui permet d'écrire \eqref{eq:Ch01-029} sous la forme
\begin{equation}
    \frac{\ud}{\ud t} \left( K + W \right) = \iiint_{\Omega} f_i \frac{\partial u_i}{\partial t} \ud v + \iint_{\partial \Omega} \sigma_{ij} n_j \frac{\partial u_i}{\partial t} \ud S
    \label{eq:Ch06-014}
\end{equation}
où $W$ est l' énergie de déformation du so1ide
\begin{equation}
    W = \iiint_{\Omega} w \ud v = \frac{1}{2} \iint_{\Omega} A_{ijkh} \varepsilon_{ij} \varepsilon_{kh} \ud v
    \label{eq:Ch06-015}
\end{equation}
La signification de \eqref{eq:Ch06-014} est claire: la dérivée par rapport au temps de l'énergie totale (cinétique + de déformation) du solide est égale à la puissance des efforts extérieurs.

Supposons maintenant que notre problème \eqref{eq:Ch06-012} admette deux solutions correspondant aux mêmes données $\left( f_i, u_i^0, V_i^0, u_i^d, T_i^d \right)$.
La différence de ces deux solutions
\begin{equation}
    \bar{u}_i = u_i^{(2)} - u_i^{(1)}, \quad \bar{\varepsilon}_{ij} = \varepsilon_{ij}^{(2)} - \varepsilon_{ij}^{(1)}, \quad  \bar{\sigma}_{ij} = \sigma_{ij}^{(2)} - \sigma_{ij}^{(1)}
    \label{eq:Ch06-016}
\end{equation}
sera solution du problème homogène associé à \eqref{eq:Ch06-012}, c'est à dire du problème correspondant aux données nulles
\begin{equation}
    \bar{f}_i = 0, \quad \bar{u}_i^0 = \bar{V}_i^0 = 0, \quad \bar{T}_i^d= 0
    \label{eq:Ch06-017}
\end{equation}
En appliquant \eqref{eq:Ch06-014} à ce problème, on trouve que le second membre
\begin{equation}
    \iint_{\Omega} \bar{f}_i \frac{\partial \bar{u}_i}{\partial t} \ud v + \iint_{S_u} \bar{\sigma_{ij}} n_j \frac{\partial \bar{u}_i^d}{\partial t} \ud S + \iint_{S_f} \bar{T}_i^d \frac{\partial \bar{u}_i}{\partial t} \ud S = 0
    \label{eq:Ch06-018}
\end{equation}
est nul d'après \eqref{eq:Ch06-017} , puisque $\bar{f}_i$, $\bar{T}_i^d$ et $\bar{u}_i^d$ et donc $\partial \bar{u}_i^d/\partial t$ sont nuls.
On en tire
\begin{equation}
    \frac{\ud}{\ud t} \left( \bar{K} + \bar{W} \right) = 0, \quad \bar{K} + \bar{W} = \text{Cte} = 0
    \label{eq:Ch06-019}
\end{equation}
puisqu'à l'instant initial on a d'après \eqref{eq:Ch06-017}
\begin{equation}
    \bar{u}_i \left( x,0 \right) = \frac{\partial \bar{u}_i}{\partial t} \left( x,0 \right) = 0
    \label{eq:Ch06-020}
\end{equation}
Or l'énergie cinétique $K$, par définition, et l'énergie de déformation $W$ d'après le postulat de stabilité \eqref{eq:Ch04-004}, sont définis positifs, d'où il résulte que $\bar{K}$ et $\bar{W}$ restent nuls au cours du temps.
On a donc en tout point et à tout instant $\partial \bar{u}_i/\partial t = 0$ d'où l'on tire
\begin{equation}
    \bar{u}_i \left( x,t \right) = 0, \quad u_i^{(2)} \left( x,t \right) = u_i^{(1)} \left( x,t \right)
    \label{eq:Ch06-021}
\end{equation}
Les deux solutions coïncident et le problème \eqref{eq:Ch06-012} a une solution unique.
Nous démontrerons au chapitre \ref{chap:09} le théorème d'unicité peur le problème statique, mais provisoirement nous l'admettrons.

\subsection{Équations de Navier}
Pour résoudre analytiquement un problème d'élasticité, on postule \textit{a priori} une forme particulière pour la solution, et on essaie de vérifier toutes les équations.
Si on y parvient, alors d'après le théorème d'unicité pour un problème régulier, c'est la solution du problème.
Il en résulte donc deux méthodes, suivant que l'on essaie un champ de déplacement ou un champ de contraintes.

Si l'on part du champ de déplacement $u_i$ on peut calculer le tenseur des déformations par \eqref{eq:Ch06-003} et le tenseur des contraintes par la loi de comportement \eqref{eq:Ch06-002}.
Il ne reste donc plus à vérifier que les équations du mouvement \eqref{eq:Ch06-001}, les conditions aux limites de type déplacement et de type effort et éventuellement les conditions initiales.
Si on reporte \eqref{eq:Ch06-002} et \eqref{eq:Ch06-003} dans l'équation du mouvement \eqref{eq:Ch06-001}, on obtient l'équation qui doit être vérifiée par le champ de déplacements $u_i\left( x,t \right)$ en dynamique ou $u_i(x)$ en statique
\begin{equation}
    A_{ijkh} \frac{\partial^2 u_k}{\partial x_j \partial x_h} + f_i = \rho \frac{\partial^2 u_i}{\partial t^2} \text{ ou } 0
    \label{eq:Ch06-022}
\end{equation}
où on a utilisé la symétrie \eqref{eq:Ch05-003} de $A_{ijkh}$ et en supposant le matériau homogène ($A$ constant).
En élasticité, le temps n'intervient pas dans la loi de comportement.
Il n'intervient donc que dans l'équation du mouvement et disparaît donc en quasi-statique à l'exception des problèmes de frottement, où il reste dans la CL \eqref{eq:Ch04-015}.
Ainsi en élasticité, on ne parle jamais de problèmes quasi-statiques, mais uniquement de problèmes statiques.
Pour résoudre un problème quasi-statique, il suffit en effet de résoudre à chaque instant le problème statique correspondant.
Nous n'envisagerons plus désormais que le cas statique.

Dans le cas de l'élastictté linéaire isotrope -- élasticité classique -- l'équation \eqref{eq:Ch06-022} devient d'après \eqref{eq:Ch05-023}, et en nous limitant au cas statique
\begin{equation}
    \left( \lambda + \mu \right) u_{i,ik} + \mu u_{i,kh} + f_i = 0
    \label{eq:Ch06-023}
\end{equation}
soit, en introduisant les opérateurs de l'analyse vectorielle (Annexe~\ref{Ann:A})
\begin{equation}
    \left( \lambda + \mu \right) \grad \left( \dive \vec{u} \right) + \mu \Delta \vec{u} + \vec{f} = 0
    \label{eq:Ch06-024}
\end{equation}
ou de manière équivalente
\begin{equation}
    \left( \lambda + 2\mu \right) \grad \dive \vec{u} - \mu \rot \rot \vec{u} + \vec{f} = 0
    \label{eq:Ch06-025}
\end{equation}
Ces équations sont appelées les équations de Navier.
Elles traduisent les équations d'équilibre pour le champ des déplacements.

Ainsi la première méthode de résolution d'un problème d'élastostatique consiste à 
\begin{itemize}
    \item postuler un champ de déplacements;
    \item vérifier les équations de Navier \eqref{eq:Ch06-024} ou \eqref{eq:Ch06-025};
    \item vérifie, les CL de type déplacement;
    \item vérifier les CL de type effort.
\end{itemize}
Pour postuler le champ de déplacements, on s'inspire habituellement des CL de type déplacement et des symétries.
On verra des exemples de cette méthode au \ref{ssec:Ch06-2.2} et au paragraphe~\ref{ssec:Ch07-2.1}. 

Si on prend la divergence de l'équation \eqref{eq:Ch06-025}, on obtient l'équation de la dilatation
\begin{equation}
    \left( \lambda + 2 \mu \right) \Delta \left( \dive \vec{u} \right) + \dive \vec{f} = 0
    \label{eq:Ch06-026}
\end{equation}
qui nous sera utile plus loin.

\subsection{Équations de Beltrahi}
\endinput
La seconde méthode de résolution consiste à postuler un champ de contraintes. La loi de comportement permet alors de calculer le champ des défotmations, mais pour pouvoir calculer le vecteur déplacement, il faut que Ce champ de déformations soit compatible (§ 111.3). Ainsi le champ de contraintes choisi doit vérifier les équations d'équilibre et un système d'équations traduisant les équations de compatibilité. Nous allons obtenir ce Système d'équations dans le cas statique et en élasticité classique et homOgène. 
Nous partons des équations de compatibilité sous la forme (111.60) et de la loi de comportement sous la forme (V.34). On obtient ainsi 
À -~y
(~G -" "~r" S . ) + <TU, ..
E Ai 
E .~ J~e E )..,~ 
.A +>! 2\l 
( ()iç.,:d.,+ ().<.~, i~ ) + -cru. .. = 0 
)J,o~
E E 
s..
(...J +v) O:t,~t -li O"H te ~a + <r~. .' 
) "'"', ""cl 
-(-1+"') [<Jit.,h + <T.i.,,-,i~ ] = o 
mais d'après les équations d'équilibre \eqref{eq:Ch06-001} 

et; 4'après la loi de comportement (V.34) et l'équation de la dila ta tion \eqref{eq:Ch06-026} 
E E 
= ­
(J.U.,U = éU,U t,À.
--l-~y (:;\ + ~t'")(-1-~)l) 
•
SOle finalement avec (V. 33) 

.A +Ii
\eqref{eq:Ch06-029} 
= ­
.-1-v 
En teportant \eqref{eq:Ch06-028} et \eqref{eq:Ch06-029} dans \eqref{eq:Ch06-027} on obtient 
(30) 

+ 

+ 

= 0 
Ces équations sont appelées équations de Beltrami, et elles traduisent les équations de compatibilité pour les contraintes. Si les forces de volume sont nulles, elles se simplifient en 
(31) 

=0 

En particulier, elles seront automatiquement vérifiées si les contraintes 
-88 ­
,Sont des fonctions linéaires des coordonnées. 
La seconde méthode de résolution d'un problème élasto-statique con­
siste à postuler un champ de contraintes; vérifier les équations d'équilibre; v€rifier les équations de Beltrami; vérifier les CL de type effort; 
puis, le cas échéant 
' intégrer le champ de déplacements; 

• vérifier les CL de type déplacement.
\ On voit donc que cette méthode s'applique tout naturellement aux problèmes de type II pour lesquels on peut sauter les deux dernières étapes. Nous en verrons des exemples au § 2.1 et aux § VII.2.2 et VII.3.1 . 

Comme premier exemple, on peut citer le problème de la compression d'un lopin avec CL de non frottement, qui a été résolu de manière tout à fait générale au § IV.I.3 • Nous allons traiter deux autres exemples tirés du § IV. 1 • 
2.1 DEFORMATION D'UN BLOC PESANT 
:lt~ 


Nous considérons donc le problème du bloc pesant posé sur un ballon de baudruche (§ IV.I.2). On aura donc et les CL
i.e -Pt 
sont 
(T.. 'fI.. .. 0 sur 
St
...t t 
(32) 
et& ..  <l'~\ • <l"1.~ <1"\\
'" c-O x.,. 0 <l'u • <I",!> = 0 , (TU .. -1" 
C'est un problème régulier -la solution sera donc unique -de type II, ce qui conduit à rechercher le champ de contraintes. Les trois équations d'équi­libre s'écrivent 
-89 ­
<r .. .. =-0 
-4.04 lot (J'~/~ (f.~,lo 
e
(33) 
cr-.. (f'1. 1. +-<rJ, ~ ; 0
13., ..f , , Ij + û .. = 
~~, l. a-a.,~ f''J'
A.~,"" 
Les conditions aux limites sur la surface latérale s'écrivent [;n, = (n.,,'n..t., O)J 
0-:." '1'\..1 + a-~ l, tf\.l, = 
(34 ) 1-0
(fl.< '1\., <T~~ '1\.1. =
{ 0 
a-,!~ '1\., +-= 0
0;.'" "'"l. L'examen de ces équations nous conduit à chercher le tenseur des contraintes sous la forme 

o 
(35) 
o o 

qui vérifie automatiquement' -les conditions aux 1imites \eqref{eq:Ch06-034} sur St -les conditions aux aimites portant sur~. et <T.ta en x3=0 et x=.q" •
3Les équations d'équilibre donnent alors 

et les conditions aux limites en x~=~ et x3~O donnent 

On retrouve donc (IV.27). Par ailleurs, ce champ de contraintes est linéaire, 
il vérifie donc automatiquemeQt les équations de Beltrami \eqref{eq:Ch06-031} ( €lcctL). Ainsi, le champ de contraintes \eqref{eq:Ch06-035},\eqref{eq:Ch06-036} virifie toutes les équations du pro­blème. C'est la solution. 
Si l'on veut cpnnaître le dép~acement, alors on calcule le champ de déformations par la loi de cOIDPotteUJent 
• 
_)1 rt~3 0 
-
~ 
•
(37) (. 
0 
-)1 f' t 'Jel 0
-
E 
•
0 0 r~~l 
en prenant x.~ & x-~ , càd en prenant l'origine des coordonnées sur la face
3 supérieure du bloc. Par rapport à cette nouvelle variable, le champ des dé­
-
formations est linéaire, et on peut appliquer la formule (111.65) pour le cal­
-90 ­
cul du déplacement, malS on peut aussi procéder directement à partir de \eqref{eq:Ch06-003}. 
En effet 
~
E~; 
(38) 
=
E" 
=
f H 
.2 f.(:/, 
(39) 
J­
E.. ., :l, Et, 
et on obtient 
êI..u. ~ -\ ' ../
%) E 4; ~ -P'à'X-,~ ... 'f; (tc,,!r.,t.)
=EP~"'; 
l b
'il "':~ 
'GA, , 
~-Xf~'.X\ =? E -U., E
o~. 
'O.u.,. , 
~ -~ P'à-~, => E.u.,2,'0 'X.,t. E 
Ô-U, 
'O.IL" 
~ 
= 
+ 'f-l,,2, +
Ô (JO,
" 'X.... 
'il -U., 'ô .u" = + =-v p ';} le,
'0 t;:e' 'ô 'X-,
1 
'O.u... '0-U.; 
+ =-)lP'à''X"
" 
Ô I.C~ 'Ocr:.... 
une solution part-icu.lière 
, = -V F'} 'X, /.Co + 'f.. (T,~ J 'X.~) 
, 
+
= -;J F~.lC", éC,; 'f.l, ('X~, IX-, ) 
<;0
'h,~ 0 
+ (f\~ + '\'",-1 = 0 
+ % 0
'f~,; + '\\.l, 

d'où finalement la solution, en revenant à x
3 

-VP'à'I.C,(.lC;-..l) 
(41 ) 
-V P~ r.i:" (~; -~) 
:!.. f ~ [(.x, -Ât + V (.x: +r.c;) ) 
.1, à un déplacement de solide près. 

2.2 RESERVOIR 
Nous avec et
Î:.= 0 
f 
r=a 
(42) 
l 
r=b 
Compte-tenu de blème, on peut 
On en tire l'allure de la déforma­tion du bloc. 
/ 
SPHERIQUE SOUS PRESSION 
considérerons le réservoir sphérique sous pression du § IV.l les CL cr·· rn.. :0 -1-Ill· ~
<a ~ 
<J'<'i l'O.i :0 0 
la symétrie du pro­
supposer que le 


-91 ­
vecteur déplacement est radial et ne dépend que de la distance au centre r=OM 
(43) -u.. 
4 
= '3 (-") "L'.. 
(44) /i,~ = "L. 'Y:.. 
... 4 
Un calcul direct donne aiors 
(45 ) 
Le gradient du déplacement étant symétrique, il s'ensuit que son rotationnel ..... 
rot.u., est nul. L'équation de Navier \eqref{eq:Ch06-025} donne alors 
(46) 
Compte-tenu de \eqref{eq:Ch06-045}, il vient 
(47) 


= 
et par intégration 
(48) 	0( +
= 
Il reste à déterminer les constantes d'intégration d.. et f-> pour vérifier les conditions aux limites \eqref{eq:Ch06-042}. Pour cela, nous devons calculer les contraintes 
(49) 
= (eX + et nous allons écrire la loi de comportement sous la forme \eqref{eq:Ch06-039}. En effet, la décomposition de \eqref{eq:Ch06-049} en partie sphérique et déviateur donne directement 4
( € Ho .. = c( ~
= ­
(SU) 
\ .e.. = ?> ~ (Ô' _~ ~À-'.1:~) =>A.â 
Ir,~"'~ It-:' 
D'où finalement 
(51 ) 
Le tenseur des contraintes est de révolution autour de la di ree tion radiale, 
r
et les contraintes principales sont 
"' J.e-r.> ()" I=-?'KC( 4e-f'.>
(52) 	IJ, = <rJ, = 3Ka: + , ­'l.' ~ If} 
-92 ­
avec (J~ associé à la di rec tion radial e. La condi tion \eqref{eq:Ch06-042} s' écri t alors simplement puisque sur les deux sphères frontières, la normale est radiale 
4e-1!> 
r=a (]'~ = ~ Kct ---c -f-
Q}
(53) 
I.j~/,:>
[ r=b = ~ KI)( -i} 0
(]'~ = 
On obtient ainsi un système de 2 équations à 2 inconnues, qui donne les cons­
tantes d'intégration cl. et ~ par 
1'-a: ~l 
(54) 
J:,~ -0.\ et la solution est complètement déterminée. 
Il reste à écrire la condition de limite élastique. Si l'on adopte 
le critère de von Mises, alors il vient directement. par exemple à partir de 

(50) 
et (V.65) 

(55) 
J... J. .. 


= 
...~ "'t 
Si l'on adopte le critère de Tresca, alors à partir de \eqref{eq:Ch06-052} et (V.70) 
(56) 
Les deux critères donnent donc le même résultat, ce qui était évident a priori puisque l'état de contraintes est de révolution, c'est donc, à un tenseur sphérique près, un état de traction simple pour lequel Tresca et von Mises coincident par construction. Ainsi le calcul élastique est justifié si la con­dition 
&t a..? t-~
(57) 
2 (.R,.~-o.~) 1(,'1. 
est vérifiée en tout point. Le point le plus sollicité sera donc le point où r est minimum, càd à l'intérieur (r=a). On obtient donc la condition 
(58) 
qui donne la pression maximale que peut supporter le réservoir en restant dans le domaine élastique. En particulier, quelles que soient les dimensions du réservoir, on ne peut pas dépasser la pression limite 2,~/?, . Nous re­viendrons sur ce problème en plasticité au chapitre X. 
La solution générale \eqref{eq:Ch06-051} que nous avons obtenue permet de traiter 
également d'autres problèmes 
• Réservoir sphérique soumis à une preSSIon intérieure f' et une pression 
extérieure P . On obtient alors 

(59) 
= 
, Cavité sphérique dans un milieu infini 
(60) o
= f 
• Boule dans un fluide à la presslOn l' 
(61) 5Kc( =-1' = o 
malS, pour ce dernier problème, il est inutile d'aller chercher si loin. Soit 
en effet un solide SI.. immergé dans un fluide à la pression l' . En négligeant les forces de volume, on doit résoudre le problème de type II défini par les 
(62) sur 
La solution de ce problème est triviale, quelle que soit la forme du solide 
(63) cr· .
"-i 
On pourrait faire une étude analogue à celle du § IV.I.3 et montrer que cette 
solution peut s'étendre à toute loi de comportement. 
