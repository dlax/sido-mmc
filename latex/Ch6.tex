\chapter{Élasticité classique}
\section{Les équations de l'élasticité}
\subsection{Problèmes reguliers}
Pour résoudre un problème d'élasticité, il faut donc trouver un champ de déplacements $u_i\left( x,t \right)$ et un champ de contraintes $\sigma_{ij}\left( x,t \right)$ vérifiant les équations du mouvement ou d'équilibre suivant que l'on s'intéresse au problème dynamique ou quasi-statique
\begin{equation}
    \sigma_{ij,j} + f_i = \rho \frac{\partial^2 u_i}{\partial t^2} \text{ ou } 0
    \label{eq:Ch06-001}
\end{equation}
et la loi de comportement
\begin{equation}
    \sigma_{ij} = A_{ijkl} \varepsilon_{kl}
    \label{eq:Ch06-002}
\end{equation}
où le tenseur des déformations est donné par
\begin{equation}
    \varepsilon_{ij} = \frac{1}{2} \left( u_{i,j} + u_{j,i} \right)
    \label{eq:Ch06-003}
\end{equation}
On obtient donc un système de 9 équations à 9 inconnues, et le problème sera «~bien posé~», c'est à dire admettra une solution unique, pourvu qu'on lui rajoute des CL (conditions aux limites) et éventuellement des CI (conditions initiales) adéquates.
Les CI donnent la position et la vitesse du milieu à l'instant 0
\begin{equation}
    u_i \left( x,0 \right) = u_{i}^0 (x) \quad, \quad \frac{\partial u_i}{\partial t} \left( x,0 \right) = V_i^0 \left( x \right)
    \label{eq:Ch06-004}
\end{equation}

Les différents types de CL que l'on peut rencontrer ont été discutées au paragraphe~\ref{ssec:Ch4-1.2}.
On définit classiquement
\begin{description}
    \item[Problème de type I.] Les déplacements sont donnés à la frontière
        \begin{equation}
            u_i{}_{|_{\partial \Omega}} = u_i^d
            \label{eq:Ch06-005}
        \end{equation}
    \item[Problème de type II.] Les efforts appliqués au solide sur la frontière sont donnés
        \begin{equation}
            \sigma_{ij} n_j{}_{|_{\partial \Omega}} = T_i^d
            \label{eq:Ch06-006}
        \end{equation}
\end{description}
Exemples: le réservoir sphérique au paragraphe~\ref{ssec:Ch04-1.1}, ou le bloc pesant du paragraphe~\ref{ssec:Ch04-1.2} avec la CL~\eqref{eq:Ch04-025}.

Plus généralement, on a affaire à un problème mixte pour lequel sur chaque partie de $\partial \Omega$ on donne
\begin{itemize}
    \item soit les efforts, exemple \eqref{eq:Ch04-008}; 
    \item soit les déplacements, exemple \eqref{eq:Ch04-012};
    \item soit certaines composantes du déplacement et les composantes complémentaires de l'effort, exemple \eqref{eq:Ch04-011}.
\end{itemize}

Un exemple type de problème mixte est celui où l'on se donne les déplacements sur une partie de la surface et les efforts sur la partie complémentaire
\begin{equation}
    u_i{}_{|_{S_u}} =  u_i^d \quad, \quad \sigma_{ij} n_j{}_{|_{S_f}} = T_i^d
    \label{eq:Ch06-007}
\end{equation}
avec $\partial \Omega = S_u + S_f$.
C'est par exemple le cas pour les deux problèmes du paragraphe~\ref{ssec:Ch04-1.2} avec condition d'adhérence, mais pour ces mêmes problèmes avec conditions de non frottement, les CL sur les bases donnent la composante du déplacement sur $x_3$ et les composantes de l'effort sur $x_1$, $x_2$.
De manière générale, nous introduisons la classe des problèmes réguliers, problèmes pour lesquels en tout point de la frontière $\partial \Omega$ sont données 3 composantes complémentaires de l'effort $T_i = \sigma_{ij} n_j$ ou du déplacement $u_i$.
Pour qu'un problème soit régulier, il faut que l'intégrale représentant le travail des efforts de contact puisse se décomposer en deux termes
\begin{equation}
    \iint_{\partial \Omega} \sigma_{ij} n_j u_i \ud S = T_f^d \left( u_i \right) + T_u^d \left( \sigma_{ij} \right)
    \label{eq:Ch06-008}
\end{equation}
le premier terme $T_f^d$ représentant le travail des efforts donnés dans le déplacement (inconnu), et le second le travail des efforts de contact (inconnus) dans les déplacements donnés.
Pour le problème mixte \eqref{eq:Ch06-007}, on a simplement
\begin{equation}
    \iint_{\partial \Omega} \sigma_{ij} n_j u_i \ud S = \underbrace{\iint_{S_u} \sigma_{ij} n_j u_i^d \ud S}_{T_u^d\left( \sigma_{ij} \right)} +  \underbrace{\iint_{\partial \Omega} T_i^d u_i \ud S}_{T_f^d\left( u_i \right)}
    \label{eq:Ch06-009}
\end{equation}
(les problèmes de type I et II sont des cas particuliers du problème mixte~\eqref{eq:Ch06-007}).
Pour les autres problèmes réguliers, cette décomposition est plus longue à écrire.
Par exemple, pour le problème du bloc pesant avec condition de non frottement \eqref{eq:Ch06-019}, on a
\begin{equation}
    \iint_{\partial \Omega} \sigma_{ij} n_j u_i \ud S = \underbrace{\iint_{S_l +S_1} T_i^d u_i \ud S - \iint_{S_0} \left( \sigma_{13}^d u_i + \sigma_{23}^d u_2 \right.}_{T_f^d\left( u_i \right)} + \underbrace{ \left.\sigma_{33} u_3^d \right)}_{T_u^d\left(\sigma_{ij} \right)} \ud S
    \label{eq:Ch06-010}
\end{equation}
Pour ce problème particulier, chacun des termes est nul d'après \eqref{eq:Ch04-018} et \eqref{eq:Ch04-019}, mais peu importe, l'essentiel est d'examiner ce qui est donné par les conditions aux limites, et de vérifier que l'on peut effectuer la décomposition \eqref{eq:Ch06-008} sans ambiguïté.
En particulier, il en résulte que, pour le problème homogène associé, c'est à dire pour le problème correspondant à toutes les données nulles, on a automatiquement
\begin{equation}
    \iint_{\partial \Omega} \sigma_{ij} n_j u_i \ud S = 0
    \label{eq:Ch06-011}
\end{equation}
c'est un moyen commode pour vérifier qu'un problème est régulier.

Cette notion de problème régulier est essentielle, car elle recouvre la formulation naturelle des conditions aux limites en Mécanique des Solides en général.
En élasticité, les problèmes réguliers sont des problèmes linéaires -- on peut donc superposer plusieurs solutions -- et qui permettent de démontrer un certain nombre de théorèmes généraux, notamment des théorèmes d'existence et d'unicité -- un problème régulier est bien posé et les théorème~ d'énergie qui feront l'objet d'un chapitre.

Il existe des problèmes non réguliers, comme par exemple les problèmes de frottement, par exemple le problème du lopin avec la CL \eqref{eq:Ch04-015}, ou les problèmes unilatéraux, comme le problème du bloc avec la CL \eqref{eq:Ch04-024}.
Dans les deux cas, il s'agit de CL non linéaires qui rendent le problème non linéaire, et donc beaucoup plus difficile.
Les liaisons élastiques donnent un exemple de problème linéaire non régulier.
Nous rencontrerons aussi des problèmes non réguliers par manque de données, mais il s'agit alors d'une non régularité superficielle, et qui ne nous gênera guère.

\subsection{Theorème d'unicité en dynamique}
Comme nous l'avons affirmé plus haut, un problème régulier est bien 
posé, c'est à dire admet une solution unique.
A titre d'exemple, nous allons démontrer le théorème d'unicité dans le cas dynamique.
Nous partons donc d'un problème dynamique régulier.
Pour fixer les notations, nous prendrons des CL mixtes de type \eqref{eq:Ch06-007}, mais la démonstration est valable, à des difficultés de notations près, pour tout problème régulier.
Nous cherchons donc $u_i\left( x,t \right)$, $\sigma_{ij}\left( x,t \right)$, $\varepsilon_{ij}\left( x,t \right)$, solution du problème suivant
\begin{equation}
    \left\{
    \begin{aligned}
        \rho \frac{\partial^2 u_i}{\partial t^2} &= \sigma_{ij,j} + f_i \\
        \sigma_{ij} &= A_{ijkl} \varepsilon_{kh} \\
        \varepsilon_{ij} &= \frac{1}{2} \left( u_{i,j} + u_{j,i} \right) \\
        u_i \left( x,0 \right) &= u_i^c \left( x \right) \quad & V_i\left( x,0 \right) = V_i^c\left( x \right) \\
        u_i{}_{|_{S_u}} &= u_i^d  \quad & \sigma_{ij} n_j{}_{S_f} = T_i^d
    \end{aligned}
    \right.
    \label{eq:Ch06-012}
\end{equation}
Au paragraphe~\ref{ssec:Ch01-2.1} nous avons démontré le théorème de l'énergie cinétique \eqref{eq:Ch01-029}, mais en élasticité on a d'après \eqref{eq:Ch05-006} et \eqref{eq:Ch05-008}
\begin{equation}
    \sigma_{ij} D_{ij} = \sigma_{ij} \frac{\ud \varepsilon_{ij}}{\ud t} = \frac{\ud w}{\ud t}
    \label{eq:Ch06-013}
\end{equation}
ce qui permet d'écrire \eqref{eq:Ch01-029} sous la forme
\begin{equation}
    \frac{\ud}{\ud t} \left( K + W \right) = \iiint_{\Omega} f_i \frac{\partial u_i}{\partial t} \ud v + \iint_{\partial \Omega} \sigma_{ij} n_j \frac{\partial u_i}{\partial t} \ud S
    \label{eq:Ch06-014}
\end{equation}
où $W$ est l' énergie de déformation du so1ide
\begin{equation}
    W = \iiint_{\Omega} w \ud v = \frac{1}{2} \iint_{\Omega} A_{ijkh} \varepsilon_{ij} \varepsilon_{kh} \ud v
    \label{eq:Ch06-015}
\end{equation}
La signification de \eqref{eq:Ch06-014} est claire: la dérivée par rapport au temps de l'énergie totale (cinétique + de déformation) du solide est égale à la puissance des efforts extérieurs.

Supposons maintenant que notre problème \eqref{eq:Ch06-012} admette deux solutions correspondant aux mêmes données $\left( f_i, u_i^0, V_i^0, u_i^d, T_i^d \right)$.
La différence de ces deux solutions
\begin{equation}
    \bar{u}_i = u_i^{(2)} - u_i^{(1)}, \quad \bar{\varepsilon}_{ij} = \varepsilon_{ij}^{(2)} - \varepsilon_{ij}^{(1)}, \quad  \bar{\sigma}_{ij} = \sigma_{ij}^{(2)} - \sigma_{ij}^{(1)}
    \label{eq:Ch06-016}
\end{equation}
sera solution du problème homogène associé à \eqref{eq:Ch06-012}, c'est à dire du problème correspondant aux données nulles
\begin{equation}
    \bar{f}_i = 0, \quad \bar{u}_i^0 = \bar{V}_i^0 = 0, \quad \bar{T}_i^d= 0
    \label{eq:Ch06-017}
\end{equation}
En appliquant \eqref{eq:Ch06-014} à ce problème, on trouve que le second membre
\begin{equation}
    \iint_{\Omega} \bar{f}_i \frac{\partial \bar{u}_i}{\partial t} \ud v + \iint_{S_u} \bar{\sigma_{ij}} n_j \frac{\partial \bar{u}_i^d}{\partial t} \ud S + \iint_{S_f} \bar{T}_i^d \frac{\partial \bar{u}_i}{\partial t} \ud S = 0
    \label{eq:Ch06-018}
\end{equation}
est nul d'après \eqref{eq:Ch06-017} , puisque $\bar{f}_i$, $\bar{T}_i^d$ et $\bar{u}_i^d$ et donc $\partial \bar{u}_i^d/\partial t$ sont nuls.
On en tire
\begin{equation}
    \frac{\ud}{\ud t} \left( \bar{K} + \bar{W} \right) = 0, \quad \bar{K} + \bar{W} = \text{Cte} = 0
    \label{eq:Ch06-019}
\end{equation}
puisqu'à l'instant initial on a d'après \eqref{eq:Ch06-017}
\begin{equation}
    \bar{u}_i \left( x,0 \right) = \frac{\partial \bar{u}_i}{\partial t} \left( x,0 \right) = 0
    \label{eq:Ch06-020}
\end{equation}
Or l'énergie cinétique $K$, par définition, et l'énergie de déformation $W$ d'après le postulat de stabilité \eqref{eq:Ch04-004}, sont définis positifs, d'où il résulte que $\bar{K}$ et $\bar{W}$ restent nuls au cours du temps.
On a donc en tout point et à tout instant $\partial \bar{u}_i/\partial t = 0$ d'où l'on tire
\begin{equation}
    \bar{u}_i \left( x,t \right) = 0, \quad u_i^{(2)} \left( x,t \right) = u_i^{(1)} \left( x,t \right)
    \label{eq:Ch06-021}
\end{equation}
Les deux solutions coïncident et le problème \eqref{eq:Ch06-012} a une solution unique.
Nous démontrerons au chapitre \ref{chap:09} le théorème d'unicité peur le problème statique, mais provisoirement nous l'admettrons.

\subsection{Équations de Navier}
\endinput
Pour résoudre analytiquement un problème d'élasticité, on postule 
a priori une forme particulière pour la solution, et on essaie de vérifier 
toutes les équations. Si on y parvient, alors d'après le théorème d'unicité pour un problème régulier, c'est la solution du problème. Il en résulte donc deux méthodes, suïvant que l'on essaie un champ de déplacement ou un champ de contraintes. 
Si l'on part du champ de déplacement ~.i. ' on peut calculer le tenseur des déformations par (3) et le tenseur des contraintes par la loi de comportement (2). Il ne reste donc plus à vérifier que les équations du mou­vement (1), les conditions aux limites de type déplacement et de type effort et éventuellement les conditions initiales. Si on reporte (2) et (3) dans l'équation du mouvement (1), on obtient l'équation qui doit être vérifiée par le champ de déplacements .....;,(':.c,t) en dynamique ou .JJ.-.i.('~) en statique 
(22) 

ou o
+ = 
F 

-8G ..... 

où on a utilisé la symétrie (V.3) de Al~~t.' et en supposant le matériau homogène (A constant). En élasticité, le temps n'intervient pas dans la loi de comportement. Il n'intervient donc que dans l'équation du mouvement et disparaît donc en quasi-statique à l'exception des problèmes de frot­tement, où il reste dans la CL (IV.15) Ainsi en élasticité, on ne parle jamais de problèmes quasi-statiques, mais uniquement de problèmes statiques. Pour résoudre un problème quasi-statique, il suffit en effet de résoudre à chaque instant le problème statique correspondant. Nous n'envisagerons plus désormais que le cas statique. 
Dans le cas de l'élastictté linéa~re isotrope -élasticité clas­SIque -l'équation \eqref{eq:Ch06-022} dev~ent d'après (V.23), et en nous limitant au cas statique 

soit, en introduisant les opératetlts de l'analyse vectorielle (Annexe A) 
(24) (~+ ~) ~(dJ....rÂ) + ll + { -0
~ ~ 
ou de manière équivalente 
(25) (';\+~e-) rt'o.Q. ctw~ \'" .wt Jtg\; Ai, + { -0
~ 
Ces équationS Gont: appelées ~es lêquations de Navier. Elles traduisent les 
équations d1équilibre pour le champ des déplacements. 
~~si la premt~re méthode de résolution d'un problème d'élastosta­
tique consiste â 
postulev un champ dé d~placements; 
vérifier les équationS de Navier \eqref{eq:Ch06-024} ou \eqref{eq:Ch06-025}; 
• 
vérifie, les CL de type déplacement; 

• 
vérifier les CL de type effort. 
Pour postuler le champ de déplacements, on s'inspire habituellement des CL 
de typa déplacement et des symétries. On verra des exemples de cette méthode 
au § 2.2 et au § VII.2. 1 . 



Si ort prend la divergence de l'équation \eqref{eq:Ch06-025}, on obtient l'équation de ïa dilatation 

~ui nous sera utile plus loin. 
-87 ­
1.4 EQUATIONS DE BELTRAHI 
La seconde méthode de résolution consiste à postuler un champ de contraintes. La loi de comportement permet alors de calculer le champ des défotmations, mais pour pouvoir calculer le vecteur déplacement, il faut que Ce champ de déformations soit compatible (§ 111.3). Ainsi le champ de contraintes choisi doit vérifier les équations d'équilibre et un système d'équations traduisant les équations de compatibilité. Nous allons obtenir ce Système d'équations dans le cas statique et en élasticité classique et homOgène. 
Nous partons des équations de compatibilité sous la forme (111.60) et de la loi de comportement sous la forme (V.34). On obtient ainsi 
À -~y
(~G -" "~r" S . ) + <TU, ..
E Ai 
E .~ J~e E )..,~ 
.A +>! 2\l 
( ()iç.,:d.,+ ().<.~, i~ ) + -cru. .. = 0 
)J,o~
E E 
s..
(...J +v) O:t,~t -li O"H te ~a + <r~. .' 
) "'"', ""cl 
-(-1+"') [<Jit.,h + <T.i.,,-,i~ ] = o 
mais d'après les équations d'équilibre (1) 

et; 4'après la loi de comportement (V.34) et l'équation de la dila ta tion \eqref{eq:Ch06-026} 
E E 
= ­
(J.U.,U = éU,U t,À.
--l-~y (:;\ + ~t'")(-1-~)l) 
•
SOle finalement avec (V. 33) 

.A +Ii
\eqref{eq:Ch06-029} 
= ­
.-1-v 
En teportant \eqref{eq:Ch06-028} et \eqref{eq:Ch06-029} dans \eqref{eq:Ch06-027} on obtient 
(30) 

+ 

+ 

= 0 
Ces équations sont appelées équations de Beltrami, et elles traduisent les équations de compatibilité pour les contraintes. Si les forces de volume sont nulles, elles se simplifient en 
(31) 

=0 

En particulier, elles seront automatiquement vérifiées si les contraintes 
-88 ­
,Sont des fonctions linéaires des coordonnées. 
La seconde méthode de résolution d'un problème élasto-statique con­
siste à postuler un champ de contraintes; vérifier les équations d'équilibre; v€rifier les équations de Beltrami; vérifier les CL de type effort; 
puis, le cas échéant 
' intégrer le champ de déplacements; 

• vérifier les CL de type déplacement.
\ On voit donc que cette méthode s'applique tout naturellement aux problèmes de type II pour lesquels on peut sauter les deux dernières étapes. Nous en verrons des exemples au § 2.1 et aux § VII.2.2 et VII.3.1 . 

Comme premier exemple, on peut citer le problème de la compression d'un lopin avec CL de non frottement, qui a été résolu de manière tout à fait générale au § IV.I.3 • Nous allons traiter deux autres exemples tirés du § IV. 1 • 
2.1 DEFORMATION D'UN BLOC PESANT 
:lt~ 


Nous considérons donc le problème du bloc pesant posé sur un ballon de baudruche (§ IV.I.2). On aura donc et les CL
i.e -Pt 
sont 
(T.. 'fI.. .. 0 sur 
St
...t t 
(32) 
et& ..  <l'~\ • <l"1.~ <1"\\
'" c-O x.,. 0 <l'u • <I",!> = 0 , (TU .. -1" 
C'est un problème régulier -la solution sera donc unique -de type II, ce qui conduit à rechercher le champ de contraintes. Les trois équations d'équi­libre s'écrivent 
-89 ­
<r .. .. =-0 
-4.04 lot (J'~/~ (f.~,lo 
e
(33) 
cr-.. (f'1. 1. +-<rJ, ~ ; 0
13., ..f , , Ij + û .. = 
~~, l. a-a.,~ f''J'
A.~,"" 
Les conditions aux limites sur la surface latérale s'écrivent [;n, = (n.,,'n..t., O)J 
0-:." '1'\..1 + a-~ l, tf\.l, = 
(34 ) 1-0
(fl.< '1\., <T~~ '1\.1. =
{ 0 
a-,!~ '1\., +-= 0
0;.'" "'"l. L'examen de ces équations nous conduit à chercher le tenseur des contraintes sous la forme 

o 
(35) 
o o 

qui vérifie automatiquement' -les conditions aux 1imites \eqref{eq:Ch06-034} sur St -les conditions aux aimites portant sur~. et <T.ta en x3=0 et x=.q" •
3Les équations d'équilibre donnent alors 

et les conditions aux limites en x~=~ et x3~O donnent 

On retrouve donc (IV.27). Par ailleurs, ce champ de contraintes est linéaire, 
il vérifie donc automatiquemeQt les équations de Beltrami \eqref{eq:Ch06-031} ( €lcctL). Ainsi, le champ de contraintes \eqref{eq:Ch06-035},\eqref{eq:Ch06-036} virifie toutes les équations du pro­blème. C'est la solution. 
Si l'on veut cpnnaître le dép~acement, alors on calcule le champ de déformations par la loi de cOIDPotteUJent 
• 
_)1 rt~3 0 
-
~ 
•
(37) (. 
0 
-)1 f' t 'Jel 0
-
E 
•
0 0 r~~l 
en prenant x.~ & x-~ , càd en prenant l'origine des coordonnées sur la face
3 supérieure du bloc. Par rapport à cette nouvelle variable, le champ des dé­
-
formations est linéaire, et on peut appliquer la formule (111.65) pour le cal­
-90 ­
cul du déplacement, malS on peut aussi procéder directement à partir de (3). 
En effet 
~
E~; 
(38) 
=
E" 
=
f H 
.2 f.(:/, 
(39) 
J­
E.. ., :l, Et, 
et on obtient 
êI..u. ~ -\ ' ../
%) E 4; ~ -P'à'X-,~ ... 'f; (tc,,!r.,t.)
=EP~"'; 
l b
'il "':~ 
'GA, , 
~-Xf~'.X\ =? E -U., E
o~. 
'O.u.,. , 
~ -~ P'à-~, => E.u.,2,'0 'X.,t. E 
Ô-U, 
'O.IL" 
~ 
= 
+ 'f-l,,2, +
Ô (JO,
" 'X.... 
'il -U., 'ô .u" = + =-v p ';} le,
'0 t;:e' 'ô 'X-,
1 
'O.u... '0-U.; 
+ =-)lP'à''X"
" 
Ô I.C~ 'Ocr:.... 
une solution part-icu.lière 
, = -V F'} 'X, /.Co + 'f.. (T,~ J 'X.~) 
, 
+
= -;J F~.lC", éC,; 'f.l, ('X~, IX-, ) 
<;0
'h,~ 0 
+ (f\~ + '\'",-1 = 0 
+ % 0
'f~,; + '\\.l, 

d'où finalement la solution, en revenant à x
3 

-VP'à'I.C,(.lC;-..l) 
(41 ) 
-V P~ r.i:" (~; -~) 
:!.. f ~ [(.x, -Ât + V (.x: +r.c;) ) 
.1, à un déplacement de solide près. 

2.2 RESERVOIR 
Nous avec et
Î:.= 0 
f 
r=a 
(42) 
l 
r=b 
Compte-tenu de blème, on peut 
On en tire l'allure de la déforma­tion du bloc. 
/ 
SPHERIQUE SOUS PRESSION 
considérerons le réservoir sphérique sous pression du § IV.l les CL cr·· rn.. :0 -1-Ill· ~
<a ~ 
<J'<'i l'O.i :0 0 
la symétrie du pro­
supposer que le 


-91 ­
vecteur déplacement est radial et ne dépend que de la distance au centre r=OM 
(43) -u.. 
4 
= '3 (-") "L'.. 
(44) /i,~ = "L. 'Y:.. 
... 4 
Un calcul direct donne aiors 
(45 ) 
Le gradient du déplacement étant symétrique, il s'ensuit que son rotationnel ..... 
rot.u., est nul. L'équation de Navier \eqref{eq:Ch06-025} donne alors 
(46) 
Compte-tenu de \eqref{eq:Ch06-045}, il vient 
(47) 


= 
et par intégration 
(48) 	0( +
= 
Il reste à déterminer les constantes d'intégration d.. et f-> pour vérifier les conditions aux limites \eqref{eq:Ch06-042}. Pour cela, nous devons calculer les contraintes 
(49) 
= (eX + et nous allons écrire la loi de comportement sous la forme \eqref{eq:Ch06-039}. En effet, la décomposition de \eqref{eq:Ch06-049} en partie sphérique et déviateur donne directement 4
( € Ho .. = c( ~
= ­
(SU) 
\ .e.. = ?> ~ (Ô' _~ ~À-'.1:~) =>A.â 
Ir,~"'~ It-:' 
D'où finalement 
(51 ) 
Le tenseur des contraintes est de révolution autour de la di ree tion radiale, 
r
et les contraintes principales sont 
"' J.e-r.> ()" I=-?'KC( 4e-f'.>
(52) 	IJ, = <rJ, = 3Ka: + , ­'l.' ~ If} 
-92 ­
avec (J~ associé à la di rec tion radial e. La condi tion \eqref{eq:Ch06-042} s' écri t alors simplement puisque sur les deux sphères frontières, la normale est radiale 
4e-1!> 
r=a (]'~ = ~ Kct ---c -f-
Q}
(53) 
I.j~/,:>
[ r=b = ~ KI)( -i} 0
(]'~ = 
On obtient ainsi un système de 2 équations à 2 inconnues, qui donne les cons­
tantes d'intégration cl. et ~ par 
1'-a: ~l 
(54) 
J:,~ -0.\ et la solution est complètement déterminée. 
Il reste à écrire la condition de limite élastique. Si l'on adopte 
le critère de von Mises, alors il vient directement. par exemple à partir de 

(50) 
et (V.65) 

(55) 
J... J. .. 


= 
...~ "'t 
Si l'on adopte le critère de Tresca, alors à partir de \eqref{eq:Ch06-052} et (V.70) 
(56) 
Les deux critères donnent donc le même résultat, ce qui était évident a priori puisque l'état de contraintes est de révolution, c'est donc, à un tenseur sphérique près, un état de traction simple pour lequel Tresca et von Mises coincident par construction. Ainsi le calcul élastique est justifié si la con­dition 
&t a..? t-~
(57) 
2 (.R,.~-o.~) 1(,'1. 
est vérifiée en tout point. Le point le plus sollicité sera donc le point où r est minimum, càd à l'intérieur (r=a). On obtient donc la condition 
(58) 
qui donne la pression maximale que peut supporter le réservoir en restant dans le domaine élastique. En particulier, quelles que soient les dimensions du réservoir, on ne peut pas dépasser la pression limite 2,~/?, . Nous re­viendrons sur ce problème en plasticité au chapitre X. 
La solution générale \eqref{eq:Ch06-051} que nous avons obtenue permet de traiter 
également d'autres problèmes 
• Réservoir sphérique soumis à une preSSIon intérieure f' et une pression 
extérieure P . On obtient alors 

(59) 
= 
, Cavité sphérique dans un milieu infini 
(60) o
= f 
• Boule dans un fluide à la presslOn l' 
(61) 5Kc( =-1' = o 
malS, pour ce dernier problème, il est inutile d'aller chercher si loin. Soit 
en effet un solide SI.. immergé dans un fluide à la pression l' . En négligeant les forces de volume, on doit résoudre le problème de type II défini par les 
(62) sur 
La solution de ce problème est triviale, quelle que soit la forme du solide 
(63) cr· .
"-i 
On pourrait faire une étude analogue à celle du § IV.I.3 et montrer que cette 
solution peut s'étendre à toute loi de comportement. 
-95 -Chapi tre VII 
LE PROBLEME DE SAINT VENANT 

1.1 LE PRINCIPE DE SAINT VENANT 
Le p~oblème de Saint Venant est le problème de base de la Résis­tance des Natériaux. Une poutre cylindrique est sollicitee à ses deux extré­mités, les efforts exercés étant caractérisés par leur torseur résultant. 
X2­
-

On considère une poutre cylindrique 
\ 
de sec tian L et de longueur 1 


Les efforts de volume sont supposés nuls, t~= 0 la surface latérale
J 
5.[ = [O,t] x"OL est libre de contrainte 

et les efforts exercés sur les extrémités Lo (xl=O) et L, (xJ=Z) sont caractérisés par leurs torseurs résultants [1;01 et [~~1 . Nous représente­
~ 
~ 
rons [0, J par sa résul tante 'R et par son moment résultant ~ au point 

~ 
A (1,0,0), centre de L--( , et de 
par sa résultante ~ et son 
-
o 
rnornen t cm
"lo au point A. (0,0,0), centre de Lo . La poutre étant en équi­1ibre, les efforts exerces sur et doivent s'équilibrer, ce qui don­
2.. L. 
ne les relations vectorielles 
~
.... 
0
{ 
1\0 i" "R = 
(2) 
~
.... .... .... 
"nl.o i" 'l1l i" AoA "1<: = 0 Ainsi, on peut calculer en fonc tion de et ~ ,et les efforts 
.... 
exercés sur la poutre seront 
caractérisés par les deux vecteurs "R et 'fil­
Pour relier ces efforts aux contraintes, il faut considérer les· 
efforts exercés sur ~ à travers "'" .
"-1 . 
(3) ~ = (.-l, 0,0) T 

et intégrer sur toute la section. Il vient 
(4) 	, 
soit, en composantes 
~
(5) 1<, 0:;, cl:x:~ dn;3
$I: 
1 
'Rl,= §I:, a-.~ .1,x" .1,x~ 
~
'R~ 0:., .1,1" cl:X:3
§r, 
(7) 
cm = § ( x" 0:" -'Xl <r..,) clx.. cl'X:~ 
1 L 
1 
'n1 = $ J.C~ 0;... olx.. cl'X.~ 
l, l:
•
(8) 	{ 'lYLÔ=-§1: 'Xl, <1".... cl;!:.. cLx.~ 
1 
On aurait des formules analogues sur ï:. 0 • 
Il est clair que le problème de Saint Venant ainsi posé n'est pas régulier, par manque de données. En effet, les conditions (5),(6),(7),(8), 
.... 
sont insuffisantes pour déterminer la répartition des efforts T sur ~1 Pour obtenir un problème régulier, il faudrait préciser la manière dont les torseurs ['(;.] et ['b.] sont appl iqués. Plus précisément, on peut imaginer 
plusieurs -	en fait une infinité -répartitions d'efforts surfaciques vé­
rifiant (5), (6), (7), (8), sur r , et de même sur 1:. • A chacune de ces
,< 	0 
répartitions sera associé un problème régulier (de ~ype II), donc avec solu­tion unique. Ainsi le problème de Saint Venant, tel que nous l'avons formulé, 
--.0=__< _.
admet une infinité de solutions. 
princiPe de Saint Venant. L'état de contrainte et de déformation loin des 
extrémités dépend uniquement du torseur des efforts appliqués, et non de la 
[
manière précise dont ces efforts sont appliqués. 
En d'autres termes, deux répartitions d'efforts surfaciques condui­sant au même torseur, conduiront à deux solutions très voisines, sauf au VOl­
sinage immédiat des extrémités. Ainsi le problème de Saint Venant admet une 
infinité de solutions, mais ces solutions sont très voisines les unes des au­
tres, et il n'y a pas lieu de les distinguer -à moins de vouloir des irifor­
mations précises 'sur c~ qui se passe au voisinage des extrémités ­
Initialement, 	ce principe était d'origine intuitive; c'est lui qu~ 
se trouve à la base du célèbre mémoire de Saint Venant qUl, déjà en 1856, con­
tenait l'essentiel de ce chapitre. Depuis, il a reçu dr ~mbreuses VéLifica­tions expérimentales directes ou indirectes, car c'est le postulat de base de la Résistance des Matériaux. Récemment, on a pu le démontrer dans certains cas particuliers par une étude mathématique des équations de l'élasticité. 
Ce principe Joue un rôle tout à fait essentiel pour deux raIsons. Tout d'abord, dans la pratique, on verra que l'on connaît assez rarement la répartition des efforts, alors que l'on a facilement leur torseur résultant. Ensuite, c'est grâce à lui que nous pourrons résoudre le problème de Saint Venant, en jouant sur la latitude qui nous est laissée sur la répartition pré­cise des efforts. Notre démarche va être la suivante. Nous allons tout d'abord décomposer le problème en 6 problèmes élémentaires correspondant à chacune des
-.
composantes de et 'r\1.. . Ensu~te, pour chacun de ces problèmes élémentaires, nous trouverons une solution, et par superposition, nous aurons une solution 
du problème complet. 
Nous décomposons donc le problème de Saint Venant en 6 problèmes. Problème 1 : Traction 
(9) 

"R~ ="R; = 0 , 

soit, d'après (2) 

= o F" 1:)(2. 
7 F 
Cl ~ ::> :r~ 
c'est la situation de l'essai de traction. 
"'3 
Problèmes 2 et 3 Flexion composée 

'nt = 0
(10) 'R, = 0 = 'R, 
pour le problème 2 (le problème 3 s'obtient en échangeant les indices 2 et 3), 
d'où 

~ -i.~
o ~. "~ " 
F 
'>
ft.ft F t ~1 • Problème 4 : Torsion 
:t:;, 
( 1 1 ) 

, 'Ill.~ ='Ill, = 0 = '!\1. ê, m 

~ 
.:t,
1 f
:x'lY\... 
3 
Problèmes 5 et 6 Flexion pure 
(12) 

,
'R-+ = -'R." 0 

pour le problème 6 (le problème 5 s'obtient en échangeant les indices 2 et 3). 

Pour chacun de ces problèmes, les conditions portent sur les eon­traintes. Nous adopterons donc l'approche du § VI.I.4 en cherchant un ahamp de contraintes vérifiant 
-les équations d' êquil ibre avec 
t= 0 
les équations de Beltrami -les CL (1 ) sur la surface latérale -les conditions (5),(6),(7),(8), pour le problème étudié. 
1.2 REPARTITION DES CONTRAINTES NORMALES 
On constate tout d'abord sur (5), (6), (7), (8), que 'R, ' 'n1.~ et 'm~ ne font intervenir que la contrainte norrr~le (pour une facette de la section droite) <r" Nous cherchons donc le champ des contraintes sous la forme 
CT•• (~.. ~".'ltl) o 
(13) o
or" 0 
[ 
o o 

Les CL (1) sur 11). surface latérale sont alors automatique~ijt vé­
..
rifiées, pUlsque 'Y\. .. «(},-"".,""',) Les équations d'équilibre se rédttis~I)t à 

Les équations de Beltrami (VI.30) se réduisent alors à 

qui donne (3"",,, fonction linéaire de x2, x3. 

Il ne reste plus. à écrire que les conditions sur les extrémités, en reportant le tenseur des contraintes défini par (13) et (16) dans (5),(6),(7) et (8), on obtient 
~
'R, Q, § d,s + .R,. § ~J, <is ... .c ~ f.(~ cls 
r-I: r. 
-'Ill = CL $f.(~ cis t ~ ~ I.e; ris t .(. § .oc~ "". Js
( 17) 
~ r r. 1­
'ml, = 0-S) f.(~ J.s + ~ %.ocJ, I.e. ris ... .c § lX; cLs 
1: L-E 
'RJ, '" 1<, = 0 'hl. = 0
• 
Les intégrales qui interviennent dans (17) dépendent uniquement de la forme de la section. Ainsi, le système (17) donnera 0.., ),.., ,Co, en fonction de 1<., ~ et ~ . Nous pourrons donc résoudre par un champ de contraintes de la 
~ ~ 
forme (13),(16), les problèmes 1, 5 et 6 -traction et flexion pure 
Pour déterminer complètement les contraintes, il reste à calculer 
Q, , .,g. et .G , càd résoudre le système (17). Un choix judicieux du système d'axes xxdans le plan de la section droite ! va faciliter cette réso­
2 3 lution. Tout d'abord, on choisit l'origine au centre de gravité de L , ce 
qui assure 
( 18) 

= 0 
Ensuite, on remarque que (17) fait intervenir les composantes du tenseur d'inertie de la section! 


c'est un tenseur plan symétrique, donc diagonalisable. On peut trouver dans le plan xZ,xdeux directions principales d'inertie perpendiculaires telles
3 que le moment produit J"t~ soit nul 
(ZO) 
Nous choisirons ces directions principales comme axes x' xPratiquement,
2 30 
si la section a un axe de symétrie, alors cet axe est principal d'inertie, 
car la symétrie entraîne (ZO). Sinon la diagonalisation est facile, et on peut en particulier utiliser la méthode géométrique exposée au § II.3.Z pour 
le tenseur des contraintes. 
Avec ce choix d'axes, (17) devient simplement 
~
(21) a.,S • -'hl, ! JJ, '!YI. J, ~ .c. J~
l 
'R, = 
Jl, =§ ,ocl. t cLS , J= § ~~ d,5
3 
r. r. 
où S est la surface de L ) et Jl,' J~ , les moments d'inertie pr~nc~paux. 
On obtient donc pour 0:;, 

et on a trouvé un champ de contraintes convenables pour l'es problèmes l, 5 et 6. 
En particulier, pour la traction -Problème 1 -on retrouve la solution 
présentée au §IV.I.4 
Fis 0 0 
F / ES 'X..
t~ .
0 0 0
(23) CF : 
~~~ -v FI ES 'X.~ 0 0 0 
-).J FIE';, '):.,
..(,\.3 = 
ct es~t la solution du problème régulier associé aux CL 
(24) X-, : 0 et ~,=.Q, 

En général, les CL réelles -, par exemple dans un essai de traction -sont dif­
férentes, mais le principe de Saint Venant nous assure que cela n'a guère d'im­portance, à condition de se placer loin des têtes d'amarrage, et c'est bien ce que l'on fait dans un essal de traction. 
1 • 3 FLEXION PURE 
Considérons maintenant le problème 6 -le problème 5 se traiterait de ma­
nière identique -. Le tenseur des contraintes a la forme suivante 

o o 
o o o o o 
(25) <li = 
où l'on a supprimé l'indice 2 sur J 
(26) J = 
= 

Il reste à calculer les déplacements. Comme au § VI.2. 1, nous procéderons direc­tement en écrivant le tenseUL des déformations 
'ln 
0 0
X"
fJ "m
(27) [, 
0 "<:~ a
= 
EJ "m
0 0 --'L E''J " 
(l-U., 'm, 
'Y1t. 
eH -~ --':cl. =':> 4, = ':c, -::Cl-+ 9< ('X~. 'l:.o)â -::c, [J EJ 
'à .u.., ,,'n1. v"rn. :,
(28) 
E.tt-= --'X:, ~ -U..t = IX.., + er~ (-::c, • -:c3) Cl -:Cl, E.J ~. 
E _ 'CA3 v'n1. v 'In. 
r.:c., 9 ---:x: 'l:. + '1', ('X,. -::c.,)
'H-= ..lt" = 
!; J a, l
Cl ':1::3 EJ 'rr\.œ.
r1<4' '~<. 'C A" = + r<,t + <f't.~ = 0
Ej ,
Ô':1:., lb::, 
ÔA3 v'm,
(29) x 0
= + + (\\ J, =
( ' ',,' h,. 3 'ft,ô ,
'à ':C3 Ô ':cJ, EJ 
ÔA~ o..ll,
:l, E3~ :: + --= f<•• + 'f~, ~ = 0 Ô':1:, ~ ':C~ 
et on obtient la solution particulière 

Le déplacement est donc donne par 
.u.i  ::  - tn'L EJ  ':1:, -:X:.,  •-w ~  ':c,2i  
(31 )  .u.l,  '"  tn'l. ~EJ  [ t:I::,"  +  V ('l:~ _-;x:~).l,]  •+ W 3 ':1::,  •+ .<.l'l,  
.u.1>  ::  v"lYt. EJ  ':I::.t ':1::3  

où, en vue des applications futures, nous n'avons conservé qu'une partie du déplacement de solide, La déformation de la ligne moyenne est donnée par 

o
tandis que la déformée d'une section droite ";)C1= -!Cest caractérisée par
1 

cm,. • ) 
(33) 
=
( EJ 'J::, + w~ 

Les relations (32) et (33) montrent qu'après la déformation la ligne moyenne devient une parabole et que les sections droites restent planes et perpendi­culaires à la ligne moyenne 
:X
2 

On construit souvent la théorie élémentaire de la flexion à partir de 
Hypothèse de Navier-Bernoulli. Les sections droites restent planes et norma­[ les à la fibre moyenne. 
Cette hypothèse se trouve donc vérifiée lCl. On constate également que le moment 'ltL appliqué produit une courbure de la ligne moyenne 
(34) 
x = 

Ainsi, on pourrait envisager de mesurer le module d'Young E d'un matériau élastique par un essai de flexion: on impose un moment de flexion ~ et on observe la courbure X , ce qui détermine la "rigidité" de la poutre E J produit d'une rigidité matérielle E , liée au matériau, et d'une rigidité géométrique J , donnée par (26) et liée à la géométrie de la section droite 
~. 
En chaque point, on a un état de contraintes de traction simple, et le critère de limite d'élasticité donnera 
(35) 


soit, en introduisant 'YI ' valeur maximale de l'X.~\ 
(36) 

< 

Ainsi, d'un point de vue géométrique, la rigidité d'une poutre est 
carac térisée  par le moment  cl 1 inertie  .J  ,  tandis  que  sa  résistance  est  carac­ 
térisée par le rapport Section rectangulaire'"  J /"'1..  
i!b~ -:J.-; W  
 =  ,  '"  ,  J "[S  -ft G  
Sec tian  en  l  
~t.I~'l:l  J  '"  1,.~".e ~  ,  J "l  .t-Â.e.  J s  =  ~l­-4  :r -"15  =  ~ 1­ 

f,. <­Ceci montre la supériorité, à poids égal,de la sec tion en l sur la sec tian rectangulaire et, plus généralement, des sections en profil mince sur les sections massives. 
-103 ­
2. TORSION 
========== 
2.1 SECTION CIRCULAIRE OU ANNULAIRE 
Avant d'aborder le cas général, nous allons envisager le cas simple d'une section circulaire ou annulaire. On ohserve alors qu'en torsion, chaque section droite tourne, par rapport à la section x1=O, d'un angle proportion­nel à la distance 


Nous postulons donc un champ de déplacements 
(37) .M., = 0
i 

On obtient alors pour le gradient du déplacement et pour le tenseur des dé­formations 
0 0 0 
(38) oU-. 
-cl. X 0 -cl. '):., 
~
""i = 
<IX cl. x, 0 
~ 
é . 
= 
...~ 
cl.
0 -.ï 'X, ~ X.l,
.2­-~ 'X 0 0 .l, ~ cf. 
0 0
-'X.l,
.t 
La la i de comportement donne alors le tenseur des contraintes 

Ce champ de contraintes vérifie directement -les équations d'équilibre, 
1-les conditions aux limites sur la surface latérale 

Il  n'est  pas  nécessaire  de  vérifier les  équations  de  BeltraQi,  puisque  l'on  
est  parti d'un champ  de  déplacements.  Il  ne  reste  donc  plus qu'à écrire  le  
' 

torseur des efforts appliqués sur L . Puisque <T~~ est nul, 1<:, "IYl,. et 'l'fL"
1 
sont nuls. Par symétrie, 1?!j, et 'R.., sont nuls, et il reste simplement 
(40) 


On a donc résolu le problème 4 avec 
(41 ) G I 0( l 
o 
o 


-104 ­
Le moment de torsion crrt crée un "angle unitaire de torsion" a.. , et le mo­
dule de rigidité G l est à nouveau le produit d'une rigidité matérielle 
o 
<7= t!-et d'une rigidité géométrique Io' moment d'inertie polaire de la 

section. 
Le vecteur contrainte associé à la section droite est 
a uniquement une contrainte de cisaillement perpendiculaire au 
rayon Et de module GoI. 't. . En no­
... ....
.e>\. ' ..e. e ' les vec teurs de base associés aux coordonnées polai­res ~, e dans le plan x2,x' on a
3 
..... ­
(43) T. G-O('t,(,e 
... ... ­
et dans le repère associé aux coordonnées cylindriques autour
e,1l. ' .e.fi ' .e.1 de x], le tenseur des contraintes a pour composantes 

o 
(44) 
o 
G-dl(, 

L'état de contraintes en chaque point est un état de cisaillement simple, et le critère de limite d'élasticité s'écrit 
(45) 


où T~ est la limite élastique en cisaillement simple donnée par (V.71). En introduisant"R rayon de la section, il vient, en combinant avec (41), 
(46) 


La rigidité à la tors ion d'un arbre circulaire ou annulaire est donc caractérisée par le moment d'inertie polaire de sa section I.et sa 
o 
résistance par le rapport 
IJR 
Section circulaire 
B 
Tf. 1J~ 'J)2. 1t 1J' :D
~o ~ 1.
I = = =
= 
" ~t S g "R -1b ~5 4­
Section en tube mince 
n 'J)~e-10 'J)~ 1. n J)'.e. 1. "D
, ­
1. = = = = 
e~ 
4 S 4 1< ~ ~s ~ 
Ces relations montrent la supériorité, à poids éBal, des sections annulaires sur les sections massives. Dans le cas des tubes minces, on constate aussi 
que ~N~/t et que les composantes (44) du tenseur des contraintes dans le 
.... ..... ....
repère (e.I\.,.e.e,.e-1) peuvent s'écrire 

o 
o 

soit, compte tenu de (41) et de la valeur de Io 
0 0 0 
~'l11.
(47) 
0 0
or '" 
TI Ir-.e. 
.2"rf1. 
0 0 
R rr-.e. 
ce qui ,superposé à l'état de contraintes dû à une traction simple, redonne 
bien la forme (IV.36) obtenue au § IV.I.4 . 
2.2 THEORIE GENERALE 
po 
Nous considérons maintenant le problème 4 dans le cas d'une section 
quelConque. Le § 1 a montré que la contrainte normale ~ était déterminée
. 
~~ 
par 'R~ , 'lll~, <ro..~. Puisqu'ici ils sont nuls, on prendra donc ~~ = O. Les contraintes de cisaillement 0::.:. et cr~; par contre ne peuvent pas être nulles d'après (7). Nous cherchons donc un champ de contraintes sous la forme 
ocr;:. <ro~,]
(48) <li 
Sl.. 0
= [
<r~) 0 0 
avec 

fonctions de (x ,x2,x). Les équations d'équilibre donnent
l 3
'd 0.;1, ô 0:.,
(49) 0
+ '" 
Cl -:1:" Ô 'X.~ 

()c:;~ 0
'" ""> (J~!j, = ()''ù ( :C" l'Xl) 
à 'X.,
(50) r 
l â 0:;; 
0 
â 'X, 

= 0::., = ~; ('X.:., 'X!) 
L'équation (49) montre alors -voir par exemple le Lemme 2 du § 111.3.1 -que la forme différentielle 
(51) 

est intégrable, càd il existe une "fonction des contraintes" P(OC.tJ6:!) telle 
que 
ô<Ïl '6p
(52) ,
<ï.;~ '" ~ô = 
ô ((.~ o-x.:, 

Les équations de Beltrami donnent alors 
d ô
(53) j~ 0
~ 
j~ '" 
Ô/.C.I, Ô -X~ 

ce qUl montre que est constant; nous noterons -.2Gd. cette constante,
lt~ 
~ étant une constante d'intégration dont nous verrons plus loin la signifi­cation 
(54) 

La CL (1) sur la surface latérale s'écrit 
(55) 

= 0 
En introduisant le vecteur unitaire 
tangent à Ô L 
X = ( d.':!:.)clJ. 

il vient 
(56) 
l'(t, = -.k.t -­
de sorte que, compte tenu de (52), la CL (55) devient 
êlg>
(57) (b:~ dei? o..:t." ci~ 
t-=' 0 Ô 'l:~ 11. '0 -:c.... M v\b ~ 
La fonction p(IX..." '.ta) reste constante lorsque l'on suit ô~ , donc sur cha­que composante connexe de "èL Nous supposerons désormais que la section L 
est simplement connexe. On tire alors de (57) que 'f est constant sur 'à,[ et 
on peut toujours choisir cette constante nulle 
(58) = o 
La fonction de contrainte ~ est donc déterminée par (54) et (55), équations 
qui définissent "problème de Dirichlet", qui admet donc une solution uni­
l1n 
que. Par le changement de fonction 
(59) 

c(' rr0r.l ème se tr<LDsforme en 
LJ4'+.t =0 
(GO) f 
=0 
-107 ­
et la fonction 9P' unique solution du problème (60), dépend uniquement de la forme de la section ~. 
Il ne reste plus qu'à déterminer la constante OC , ce que nous al­lons faire en calculant les efforts exercés sur L~. On sait déjà que 'R, = 'l1l~ = '11l., = 0 . Pour calculer les autres composantes, nOUS utiliserons la formule de Stokes 

A partir de (6) et (52), il vient 
§ o<p cL .
'R~ = ci 'X~ d,x,1\ = -x, .. dJJ:;;
~L u1:l" 
L o:J:1\
( 62) = -~ il> ch~ = 0 
dL 
d'après la formule de Stokes (61) et (58). De la même manière, on a 'R" = 0 . A partir de (7) on a 

= -Ir [~CL~(~~ p) ~ 
= ~ § p clxcl:x"
t
L 


Finalement, le champ de contraintes construit permet de résoudre le problème 4 avec 
(64) 

l = 

La constante l est appelée "module de rigidité" de la section L , et, com­
me <f ' elle ne dépend que de la forrne de L . 
En chaque point de la section, l'état de contraintes est un état
-
de cisaillement simple caractérisé par le vecteur contrainte T associé à la section droite 
... 
(65) 
T = (0 1 <r1~ , ~;) 
-+ 
La CL (1) expnme que, sur la frontière dL , T est tangent à dL 
(66) sur ÔL 

Les relations (12) montrent que les courbes iJ1: Cli sont les en­
.... 
veloppes du champ de vecteurs T 

L'état de contraintes étant un état de cisaillement simple, le critère de limite d'élasticité donne 
.... 
(67) \ T \ ~ 

< (;e 
avec (;e donné par (V. 71). A partir de f ou <r cette candi tian donne 

= 

" 

Soit finalement 

où est une longueur caractéristique de la section L . On peut par ail­leurs montrer que la borne supérieure de 1~'fI est nécessairement atteinte sur la fronti~re de ~ . Ainsi le problème général de la torsion est résolu, sitôt que l'on connaît la solution 'f('r~I';C~)du problème (60). 
2.3 CALCUL DU DEPLACEHENT 
Pour terminer le calcul, il reste à.calculer le déplacement. A par­
tir de (48),(52) et (59), le tenseur des déformations est donné par 
o 
-'f,!l, 
(69) ( = d o o o o 

et on peut intégrer le champ de déplacements 
Eq = ...u.1.I~ = 0 ~ ..l.L, = »-, (œl,' -:c~) 

(70) 0
t~l. = .u~l. = => .u.2,~ ..l.L:, ( 'l:." êCa)
1
{ 
= = 0 => .Al,> = A.t:!> (x., , 'Ll,)
t" ..u, >, " 
2 E.t" :-4-1,, " ... AL,,:l, =-0 ~ ..ll.!" , --..u. =-A(x., )
-"~ (ca r Al ~}1> =-..l.L.t.,t:.(x-1,1 -.: ~) et .u.." , ~ = Ai. _ ,~( 'J." -x..<> ) et on a 

et on peut interpréter A(');.,) comme étant la rotat ion de la section cl rD i te 
d'abscisse 
'X., 
, 
~
J; €~~ = .J.t..,-1.I~ t--'L~ ,~ ...u.."11. ;-A'( "-,) 'J:, .;-\) CT,) = c< 'f,?,
(72) 


f :t = Â 	= 4 -A' l'x:,) r:r.:. t C'('L,) -cl. <t',.l,

t~" + ...u...~J1.
~,' 
"" 
Mais dans ces relations, seules les dérivées A'(T,)(= riA/J./r..~), 17,'('.1:,) et dépendent de X ; elles doivent donc être constantes 
1 
Aex,) = -a..X, + cl 
-Ll:, a.. 'X, 'X., .;-cL'l:, ;-~:x., +.e

'" 
(73) 
f 43 = Q., 'X, XJ, 	-'-cL 'J:, 1-.c Xi + { 
Déplacement de solide rigide 

On voit sur (73) que les 5 constantes t, .c , .a.. e. , f . correspondent au déplacement de solide rigide arbitraire, qui doit nécessairement s'introduire dans l'intégration. A un déplace-ment de solide près, on a donc 


ce qui correspond, comme dans le cas de la section circulaire, à une rotation de chaque section d'un angle proportionnel à la distance à l'origine; la cons­tante a.. , rotation par unité de longueur, est donc "l'angle unitaire de tor­sionl! introduit au § 1.1 pour la section circulaire. On peut maintenant calcu­

à partir de (72) 
(75) 

système qUI sera intégrable SSi 


ce qU1, d'après (60), donne finalement Q..= ex , et la constante ex. , introduite en (54) est l'angle unitaire de torsion, et le déplacement est finale~ent, à un déplacement de solide près 
! 
0(
.1.\., = '\f (X.!" ':C,) 
(76) 
.I.\.:l, = cl. ':C. ':C~ 
At, = C( 'J:" 'l:.J., 

Ainsi, la rotation de chaque section s'accompagne d'un "gauchissement" que l'on peut observer expérimentalement. La fonction de gauchissement l( est donnée par 
(77) 


et \f est la fonction harmonique conjuguée de la fonction harmonique 
~(3) 3» 

Si 	on calcule sur '0 z:. la dérivée normale de \fl ' il
'f + 	.ï + :CJ
'X-.1 
vient, en utilisant (56) et (57) 
ci'f o1ji é,,+,
'Tt.l, t-'Tt,
'" 
d."" â :C.l, '() 'X. ~ d.:c~ 'èl'f' ci:C, d,'L
( 'à 'fi 	+ !;:J, _l,
+ 	+ ~'\
'" 	'à ):.3 dl. d ::Col-~") <th J.b cl (~('x2,+ :1:.1)J
'" .2, .t ;
ru 
quantité connue le long de 'dL. Ainsi la fonction,+, vérifie 
0
f 	6'1' '" 
(78) 	cL,,+, cl. [~ (~. 2,)] sur dL
--'Xl, + (1;,
'" 
ci'1\. dA .~
l 
c'est un problème de Neumann, qui admet une solution unique. Ainsi, pour résoudre le problème de torsion, on peut, soit calculer f par le problème 

(60) et en déduire ensuite '"'1' par (77), soit calculer l' par le problème (78) et en déduire ensuite 'f' par (77). 
Finalement, si l'on compare les relations (64) et (68) du cas géné­ral, aux relations (41) et (46) relatives à la section circulaire, on consta­te que, dans le cas général également, la rigidité de la section est caracté­risée par le module de rigidité l et sa résistance par le rapport Ilrrt Mais dans le cas général, a) il faut résoudre le problème (60) pour pouvolr calculer ces constantes (nous verrons cependant au chapitre IX que l'on peut obtenir des estimations de l sans calculer Cf ), b) la torsion SI accompagne d'un gauchissement des sections. Si ce gauchissement est empêché, par exemple par des CL d'encastrement, on rencontre le difficile problème de la torsion gênée (par opposition à la torsion libre). 
Bien entendu, conformément à la démarche générale décrite au § 1.1, nous avons résolu un problème particulier correspondant au probl~me de la tor­sion, et le principe de Saint Venant nous permet d'affirmer que loin des ex­trémités c'est la solution. Il peut être utile de fonnuler expliciter..ent le problème régulier que nous avons résolu. Pour cela, il faut compléter les CL 
(1) par des CL sur les extrémités. On pourrait écrire des CL donnant sur les extrémités le dêplacement (.él, Al~, ÂJ.,,) connu par (76), ou bien donnant les 
1 
~ 
efforts appliqués T connus par (65) et (52), mais la formulation la plus com­
mode, que nous utiliserons au chapitre IX, fait intervenir des données mixtes 

_ 0 

Â,j, S .tl.! = 0 
(79) 
= 0 ..tJ..J.,= -O(.Q'X-~
• 

Rajoutées à (1), ces CL définissent bien un problème régulier (§ VI.I.I), et cette formulation présente l'avantage de ne pas faire intervenir les fonctions 
<f ou 1t' a priori inconnues. 
2.4 SECTIONS PARTICULIERES 
Section circulaire 
D'après la symétrie, les fonctions If et 'Y ne dépendent que de r. Il vient directement 
(80) ./ (1, 1,)
'f=i; Q.-'I. ~ = 0 

et on retrouve tous les résultats du § J.I . En particulier, la fonction de 
gauchissement est nulle. 
.r 
J
Section elliptique 
La section ~ est limitée par 
l'ellipse d'équation 

Q 
(81 ) 

On trouve alors pour <p et "t' 
t, ~ 
0.1, }}' 
( -! -Xl. X~ )
(82) c
i 
Q.,l,
Q.,.L+ ~? t.L 
a.,.L _ J..'" 
(83) 
l' = ');.,1, 'l:3 
Q.J., +;"1, Pour le module de rigidité à la torsion, on trouve 
(84) 
l = 

Les contraintes sont données par 
[ 
a-'l'Il. o'f .t'lll. 
.. J., = -= ':(.3
l 'è a!. 11 o.t~ 
(85) 
'IY1. 'Of .2,'fQ 
cr~~ = = ');.~
n Q.;..p,.
l "ô :x:.J., 
et la contrainte de cisaillement maximale 
'I~ 'Il"
2'Tl1.. 
~"nl
~ . l"
(86) ­
+ =
1 T ItmOJ:c = .bI>f­
!" (L" 7t a. !)}
lla...Q,. 
est atteinte à l'extrémité du petit axe OC!.:=' Z. , ce qui donne 

:0
(87) f = 
Q,l,+ p,.l" Section ree tangulaire 
~ 
CL ,
On recherche la solution sous forme d'un développement en série de Fourier double 

qui vérifie automatiquement la CL (58). On dérive (88) terme à terme, ce qUl 
permet  d'obtenir le développement  de 6~ que  l'on identifie  avec le dévelop­ 
pernent  de  la fonction  constante· -2 , et  on  obtient les  constantes A . tm.'1\,  
On obtient des  calculs plus simples  en  cherchant  la solution  sous  

la forme 
co 
(89) 'f = 
L 

développement en série de Fourier simple (mais qui présente l'inconvénient 
de  détruire  la symétrie  en  x2  et  x3). On  calcule  ~~  par dérivation  terme  
à  terme,  on  identifie  avec  le développement  de  la  constante  -2,  et  on  ob­ 
tient  pour  1f~ une  équation différentielle du  second ordre  


q "' l, \:1,
("" rrn. --1 ) 1\ .1 ('I ) 
( l ! 1',m, ô 
, Cl. / 

qui donne "f"", par intégration avec les conditions aux limites "1'~t)=O. On obtient finalement 

qui permet de calculer la solution et en particulier le module de rigidité à la torsioI;1 l et la longueur f qUi intervient dans (68) . On trouve 
. s <,
I. /1 b ~, JIr ~,(h!Q,J = i; CL JdYr(a.)
= 
1
(91) < 
.z a. i(Yrja.)
l F = 
-113 ­
la contrainte tan8entielle maximale étant obtenue au milieu du grand côté 
«~= ! k si on suppose (t> -&-. Les fonctions ~, et.t sont données par le 
tableau suivant 
Rr/O­ 1  1,5  2  3  5  00  
-Pt,  0,675  0,848  0,930  0,985  0,999  1  
-foi  0,141  0, 196  U,229  0,263  0,291  1/3  

Plus généralement, on sait résoudre explicitement le problème pour quelques sections particulières (triangle équilatéral, section circulaire en­taillée d'un demi-cercle, etc ... ). Comme le problème se ramène à des calculs de fonctions harmoniques, on peut également utiliser les techniques de varia­
ble complexe (voir(l7),[19]). Enfin, le problème (60) se prête bien au calcul 
numérique. 
3. FLEXION CO~œOSEE 
==========~======== 
3.1 CHAMP DE CONTRAINTES 
Il reste à résoudre le problème 2. :l:'t 
F~ It==~======~~_=t--l----,»F"''l::~ 

Considérons la section d'abscisse xl' et considérons la poutre [OJ~"'])(. r Elle est en équilibre sous l'action du torseur [~o]des efforts appliqués sur la section x =0 et du torseur ['b(:c,)] des efforts de contact exercés sur la
1 
poutre  [O/xJ x  L  par  la partiè supprimée ['l:".eJ. L- Gomme  précédemment,  
nous  représentons ~  ('CCx,)]  par  sa  résultante  'R('l:.)  et  son  moment  .-..'l'Q.(1.,)  par  


L'équilibre de la poutre [D,X,]. L donne alors 
'R. Fe" de sorte que la répartition des contraintes dans la section xl doit être telle que 


(92) : 0
$r ':!~ o:;~ cLx,t cL'r-; 
0;, 
,:
-! '):~ G.1i J..,X.l. 0"X,3 F(t-/.1:.,) 
Lx 

1 
~
( 93) $ tr~!l, ck.,t J.;x,~ F 
t",,< 
(94) § 0:., dr.cJ.,~. ,: 0 
~'X." . 
(95) t ('X-J-o:i,-:x:.,<J:i:o) d:.x".:z,clx" :> 0 
['L< 
En partant de (92), les résultats du § 1.2 nous suggèrent de prendre 
(96) 

, 

D'autre part, (93) montre 
dans un premler temps 
(97) 


avec ~< donné par (96). que 
Û~J, 
o o 
Les 
(î ne peut pas être nul, Nous prenons donc 
~:/, 

équations d'équilibre nous donnent alors 

= 0 , : 0 
soit, compte-tenu de (96), 

Les équations de Beltrami sont toutes 
aux indices 1,2 qui donne 
(j"~~, ~~ + ~t,U + f 
-[ r(X;) +"1 + 
3,J 
vérifiées, sauf l'équation relative 
(5' 0
= 
..(.(J~ 
F 
-= 0 
J 

A 
Ât)! 
1 
-f+v 
-115 ­
r(r.,) '" -~'l:~ i" 	Cl, X, + t 
~+v 
(99) 	F _.:l.... 'X" + Cl. ')C. + t}
2,J .-1+>' ; 

Par contre, puisque ~3 est nul, la CL sur la surface latérale, qui s'écrit encore sous la forme (55), ne peut pas être vérifiée. Nous super­aro
posons donc à l'état de contraintes obtenu jusqu'à présent, un état de 
con traintes or avec non nul 

<l1:l­
_[û",; o 
(100) 	o t
-0.;<­o o 
o
(101) 
(j..\<-= 
Par construction, le champ aro vérifie les équations d'équilibre 
et les équations de Beltrami; le champ or devra donc les vérifier égale­

ment. On peut alors reprendre l'analyse du § 2.2 et obtenir 
i ~ 
..., = 
F ~ 	F ~ 
(j1~ 	o:ï~ = -­
J -a (Co . 	J o'X-:.
(102) 
..., ,1<f = di = -le 
Nous faisons le changement de fone tian 
(103 ) 
où est la fonction introduite au § 2.2, solution du problème (60), de sor­
te que 
(104) 
11 X = 0 
Comme au § 2.2, la CL sur la surface latérale peut s'écrire 
o 
+ C = 0 
Le dernier terme 	s'annule d'après (60) et on a 
(105) 	d.X A [ K.",'-_ ~,x'-] dA:. '" At;> 3
d.b .2-	cU 
Sur d~ , x2 et xsont fonctions de .6 et par intégration de (105) sur oz.
3 on peut obtenir la valeur de sur ,,[ à une constante près
1. 
(106) 
Pour s'assurer que (106) définit sans ambiguité la fonction X sur '02. , il 

o 
faut vérifier que 
(107) 	~ 0 
Ceci 	résulte directement de la formule de Stokes et de (18). 
Ainsi, la fonction 1. est déterminée par 

(108) 
définie par (106)
[ 
c'est un ·problème de Dirichlet qui admet une solution unique dépendant unique­
ment de 	la section ~ 
3.2 CALCUL DES EFFORTS APPLIQUES 
Pour terminer la détermination du champ de contraintes, en parti~u­
lier pour déterminer la constante C , il convient de vérifier les cO;'1dit~Qns 

aux limites sur l'extrémité -:.:,,,1, càd de vérl.fier les conditions (2),(93), 
(94) et 	(95) pour ~i= 1,. Par constructicn de ~~ les relations (92) sont 
vérifiées 	pour tout xl' A par-tir des calculs du paragraphe précédent, on a 
cr~.l, = 	i [ _i (~~ _~)J:~ ) + dt + e ~ J2., 3., .-l+V Ô!j:3
(109) 	è X3 
F f 'Ot ôcp l
(j~:. = -	--+-c 
J.t 'O/(.I, o~" J et a;,~ et 0:.." dépendent uniquement de xet x3 · 
2 
Pour (93), nous partons de (109) et 

F
'R.t = -	­
lJ~ 
(110) 
+ ~ § (aI -t-C ~) d,.r. <k. 
J L ÔO: "0 ':C J, 3 
~ 3 :> Compte-tenu de la définition (21) de J~ et Jon obtient pour le i'C€Dler
3 
terme 
(II 1) 
Pour le second terme, on utilise la formule de Stokes 
Le terme en <.p diparaît par (60), et on intègre le terme en X. par parties. Compte-tenu de (105), on obtient 
On utilise à nouveau la formule de Stokes 
§'3I: J.C:),(J.C~ -~ )Ln ck~
~~~ 
-i+v 
F ~ ~ ( K., ( JL! -~~~)] ckJ, ckô
-.n:. 1:. Ô./C,:}, -l+v = F ~ (?lK; -~4~) tkl, dJt,~ 
~JJ, I: .-1..\1 et compte-tenu de (21), le deuxième terme de (110) donne 
(1 12) 
La combinaison de (110) et (112) donne alors 'R = F, et permet de vérifier 

J, 
(93). La vérification de (94) est analogue. 
( 113) 
d'après (20) . Il reste à écrire (95). Le calcul est mené de manière similaire 
./J:. ~:),) MJ, d..tL 3 
D'après le calcul qui a donné (63), il vient 
de sorte que nous pouvons écrire 
(1 14)  .  F(Ii-CI)  
J:t  
où  la  constante  H , donnée  par  

y
(1 15) -l ,.,,; -1-x 0'X i àX 
~
li -l ~JI.C;~, -] ck,t ck..
/:[.3
1 Àt" 3 l-041. O~, dépend uniquement de la section L. . La condition (95) donne alors la valeur de la constante 
H
(116) 
c = 
l 
et le champ de contraintes est parfaitement défini. Il est de la forme 
-(.e -4.) I.CJ,  
(1 17)  or  =  F  a(",-.l.'I.C.)  
J,t  [  f';,(~.l.'~;)  

où 0{ et ~ sont deux fone tions homogènes au carré cl 1 une longueur, et qui 
dépendent uniquement de la section L . 
La solution du problème de la flexion composée peut donc, comme pour la torsion, s'obtenir par résolution de problèmes de Dirichlet, mais les cal­
culs sont beaucoup plus laborieux. En particulier, on peut écrire le critère de limite d'élasticité et calculer le déplacement, malS on ne peut pas en 
tirer une interprétation simple cormne pour les autres problèmes. En particu­
lier, l'hypothèse de Navier-Bernoulli (voir § 1.3) n'est plus vérifiée, les 
sections droites ne restent plus planes: en torsion comme en flexion composée, l'apparition de contraintes de cisaillement entraîne un gauchissement de la section. 
3.3 SECTION CIRCULAIRE 
Si la section est symétrique par rapport à l'axe xz' alors, en pre­nant l'origine des abscisses curvilignes sur l'axe xz' on voit que la fonction X définie sur .. L par (106) prend des valeurs opposées en deux points symé­
o 
triques par rapport à l'axe des x2. Il en résulte que la fonction ~ définie par le problème (108) est impaire en x
3 

( 118) 


La quantité intégrée dans (115) est donc impaire en x, H est nul, et la cons­
3 
-119 ­
tante C est nulle. On obtient pour le vecteur contrainte tangentielle sur la section droite (0, ~~, ô1.) la symétrie par rapport à l'axe x
2 f (J"~l, (/f:..z" -J.L~)
(1 19) 


t (J"1~ (1)(./-1 -J:i:-!) '" -<r~ô (h::~1 !.Cl) 
A titre d'exemple, nous allons calculer la fonction ~ et la répar­tition de contraintes pour une section L circulaire, de rayon ~ . Pour cal­
culer X on écrit (105)
• 
-l
cl'X = -[t " .t] ck. 
~.z, --\+>' K3 3
l 
...t 
= [(a..L_)<:~) -~~"] ck
A+V.3 J
l 
J .. 2,\> 4" 

= i [é -1ck,3
J+V .3 .,1+.2»
~ [ a..~ IX _
1. = ~n
~ .3 ?>(AtV) 
a..> J +tv
( 120) [~e ~e]
~ = 
0 l ~(À+V) 

Pour calculer 1 a fone tian X(1.:!~/~·3)' nous devons trouver la fone tion A 
harmonique qui prend la valeur (120) pour '1,,, a. . Pour cela on remarque que
• 
les fonctions 


sont harmoniques povr tout entier ~ On écrit alors 
(122) 

et on peut réécrire (120) sous la forme 
il À+.2,v
A. o " ~[.èMve (3.&iM.6 -,oiM., ~e) ] j.2, (.Mv)
(123) 
~-tl,,, 
= a..' [ --1+:!.V + ~eJ
~:,e ~ ,,12(.-hV) li (-1+>') 
ce qUl permet d'écrire 
:'+2.V
-1 [ A+2.v Il} ~:,6 .. ---a...t ft., ».m, e ] 1 ="ï -I.i,(.tl+V) 4(.hv) 
On utilise à nouveau (122) pour écrire 

-120 ­

ce qui donne pour les contraintes 
F 3+1\1 l-L -1-3.11 
~.?. = [ Cl, -/.CJ, -I.(~ ]8J,t My ?+lv
(125) 
F À+.'I,y 
(T~'!. = 1.(:1. &3 
4J2-...\+\1 
répartition assez complexe des contraintes de cisaillement. 
OC" 
En particulier, on a 
"" 
~~.nul sur l'axe des x2 et sur t 
\ 
l'axe des x, La répartition de
3 
~L sur les deux diamètres AA' 


6' 
et BB 1 est représentée sur les diagrammes suivants 
?>
0
A.' t>. x"­
:x" 
F A+.'iV r
en B = -= ....I,l?> ­
~~ 
5 A·w S F ?> + lv f
en 0 <f~!/, -=.A,:'~ S .2,( A+v) 0;, 
pour V = 03
1 

e,' 0 
-121 ­Chapitre VIn 
PROBLEMES PLANS EN ELASTICITE 

1. ELASTICITE PLANE 
=================== 
1 . 1 DEFORNATIONS PLANES Dans de nombreux problèmes, on peut supposer les déformations pla­
nes 

On en tire le tenseur des déformations E~t E",,, = .A.l....,...
['..
(2) 
E = E.j!/' EU Et:/. = A",t. 
0 0 )
:l 
E~t =1(..u.<,~ -t..u.l, ,~ et,par la loi de comportement, le tenseur des contraintes 
(3) 
avec les relations suivantes 

EE11 = ~.j -V ( cr~2. + <T~~ ) 
(4) 
E El,l, = <T1,2, -V ( o-;,~ + cr~:,) 
.-(A + v)
E é-t.t ~l, 
(5) E E~~ 0 = a-~, -v (o:;~ + (fu)
" 
L'équation (5) donne alors 
( 6) 
cr,~ = V (<r<~ T crl.,l, ) 
et en reportant dans (4) il vient 
= ( 4-v.2.) -,,(A.v) <Ttl.
E E~. ~~ 
(7) 
CA -V") V( .-l+v)
E EJ,!!' = cr~l, -~1 E E~ 2-" (A +\1) <T~l, 

Pour résoudre un problème en déformations planes, il faut trouver les déplacements ~ ,.J..t et les contraintes 

en fonction 
1 ~ 
des coordonnées (xl'x) dans le plan.
Z
Si nous travaillons sur les contraintes, càd en utilisant l'appro­
che du § VI.I.4 , il faudra vérifier les équations d'équilibre et les équa­
tions de Beltrami. 
Les équations d'équilibre s'écrivent 
(8) 
{ 
en supposant nulles les forces de volume (sinon, l'analyse qUl suit peut s'é­
tendre, avec des résultats plus compliqués). Ces équations expriment que les formes 

, 


sont des différentielles totales. Il existe donc deux fonctions qtx:." ':1:.. ) et ''l'('X,, 'X~) telles que 

En comparant les deux expressions de ~ ,on voit que la forme 
-Il, 

est une différentielle totale, il existe donc une fonction X(':I:"~~) telle que 

La fonction X(I.:C""I,(..1,) est appelée "fonction d'Airy" ou "fonction de contrain­
tes" du problème. Elle permet de calculer les contraintes par 

les équations de l'équilibre étant alors automatiquement vérifiées. En remarquant que, d'après ( 6) et (9) , 
(10) <r~p" = CA + v) (f" + ~l,) = (A +v) 61. les équations de Beltrami (VI. 31) donnent 
-",a = A,..\ (.-I+v) !J. "-,~~ + (A+V) f::, 1,~~ " 0 l,l (A+>') + (Â+v) =0
!J 'X.,-'l1 f::, 1,:/,~
( 1 1 ) 
-l," -(A~1,-I1, + (..\~'X.,,,:/, " 0 3,~ v (A +v) f:,/:::.1... " 0 
Toutes ces équations seront vérifiées SSi la fonction ex est biharmonique 

Ainsi, pour résoudre un problème en déformations planes, il faut trouver une fonction de contraintes X biharmonique vérifiant les condi­tions aux limites. On en tire alors les contraintes 1T-\1' (f.2,~ et cr-1J.. par 
(9), cr" par (6). les déformations par (7) et les déplacements par inté­
gration du système 
.A+v 
..u, 
~
€~~ = ",~ [ (A -).Il 'X-,'J,'J, -V l,~~ ]
~ 
.À+V 	­
( 13) 
~
EU = 	--u.:" t [ (--i -v) X,~A VA,u)
E 
.2. (~ +'11) 
=
t cA:!-= .,(.LA 'J, + .,(.L.'l.-I
, , 	l, At<
E 
système qui est intégrable puisque les équations de Beltrami sont vérifiées. 
1.2 CONTRAINTES PLANES 
L'hypothèse des déformations planes convient pour une pièce suffi­
samment iongue pour que l'on puisse négliger la déformation longitudinale. 
Pour une 	plaque mince, chargée dâns son plan, la condition aux limites pour 

~~= :!: 'f./J, donne 

et on recherche donc un état de contraintes planes 
~~ (1:(,,4:2,) o:i'J, (I:C, • $:,) 
( 15) <F = ()~.'l. (~., 4",) (J.'l.!I, (1:1:" )L~)[ 
o 	o 

Le tenseur des déformations est donné par 
( 16) 
= 

E€~~ 	= <r;A -v cr:/,2, = -V cr-
E C!l,l, 	Cf2,l, ~~ 
( 17) 
E C" = 	-v (<T~ + cru) (A+ v)
E E.~l-	= ~2­
Les équations d'équilibre se traitent comme en déformations planes et con­~duisent à (9). On a alors 
( 18) 

[ 
et les équations de Beltrami donnent i.,i • ..1 •..1 (...\+11) + ,. 0
Ô 'X.,2-1-l!. 'X.,~~ 
( 19) t,2. ( ..1+\1) ... ='0
A.,~ = l!. 1..., ~~ l!.1...,u .<, ~ = ... .2--(..\+\1) .. ., 0
l!. 1,-12-6 1..,-IJ. 
équations qui ne pourront être vérifiées que si Dl est fonction linéaire des coordonnées, ce qui est bien trop restrictif pour permettre de résoudre des problèmes réels. Nous oublions donc provisoirement les équat~ons de Bel­trami, et nous allons chercher à calculer les déplacements à partir de (17) 

et finalement on obtient 
= [X,n 
(ZO) 

( X,~~ 
= .t("... ,,) 
E 


Le système (20) est formellement identique au système (13) en remplaçant V 
par vl(" ... ,,) Il permettra donc de calculer .u., (Il/:"JJ: .. ) et .A.l.s,(Il/:"Il/:.,) SSi la 
fonction lest biharmonique. Il reste à intégrer ·les équations (ZI) pour 
calculer .u.~ 
(22) 
= -E » 6,1.. "'> .Ll ~ =--E y t;,. 1(4,,«-.1) 'X:1 + Cl. (4",JJ:.2, ) 

E. 
~b 
{ 2Ev 



H = -1.1. ~,~ .. oU. ~, ~ = !J. 'X.I~ 'l:~ ... Cl.d = 0 
(Z3) E 
+ Cl. 0
.t E.2,!) = -1.1..,J. .. -U.!!,, ~ = v E 61-,.2, 1:3 ,li '" 
équations qui ne pourront jamais être vérifiées pU1sque a. et X ne dépendent 
que de xI et xz. Ainsi, si X est biharmonique, on ne peut pas calculer les déplacements; c'est tout à fait normal, puisque les équations de Beltrami (19) 
donnent 
(24) 

= 0
= 

= 

-125 ­
conditions que nOUS avonS volontairement laissées de côté. Cependant, pour 
une plaque mince, xest petit, et en première approximation, (23) donne
3 ct, = ct =0 , <1.= ct, et la solution ainsi construite est une approximation
,... 11. satisfaisante de la réalité: c'est l'approximation "contraintes planes". 
Ainsi, en déformations planes comme en contraintes planes, la solu­
tion est donnée par une fonction de contraintes r....("L""'X,~) biharmonique, donnant les contraintes par (19) et les déplacements .A.t. (-X -X ) et .u. (I.e. ~) par inté­
., "" ~ li 4' .a. 
gration du système 
.-1 ... " 
LI. -l{
= f1,l.), ~ X}
~,A 
E 
A+v
(25) 
.u. 
t,l. = t1,AA -K Ô X1 
E 
H·-\ ..,,)
-u. = 
A,.L + .4:/,,A -l,AL
E 
avec ><.=v en déformations planes 1){ = "'/(A+") en contraintes planes De plus, en déformations planes, la contrainte axiale ~~~ est donnée par (6), 
tandis que, en contraintes planes, la variation d'épaisseur de la plaque min­
ce est donnée par (22) 
V
(26) 
~ ­
E 
1.3 UTILISATION DE LA VARIABLE COMPLEXE 
Introduisons la variable complexe 

Théorème de la représentation. Toute fonction réelle harmonique peut s'écrire 
, 
sous la forme 
(28) 


Toute fonction réelle biharmonique peut s'écrire sous la forme 

avec F , G et K fonc tions holomorphes. 
Dem. La première représentation est classique: on sait que les parties réelle 

-126 '­
et imaginaire d'une fonction holomorphe sont deux fonctions harmoniques con­juguées, càd reliées par les conditions de Cauchy 
(30) 

La seconde représentation peut s'obtenir à par[ir de la précédente par deux 
méthodes. 
1ère méthode. A partir de (27), on voit que toufe fonction de ('':l'x) peut
2
être considérée comme fonction de (~,~ ). On obtient alors facilement 

'"0" '"0"'f "0" <f'
_Cf
= + ,...
Ll'f '0 I.e" 'Ob:" ô~è~

• 
" 
de sorte que si est biharmonique
Cf 
-;,~ Cf 


= = o 
ô\o"} 
on obtient 
'\' = ~ F~ ("11) + &1 (~) .. if ~(~) +. G-J,(~) 
et en écrivant que 'f est réelle, on obtient (29). 
2ème méthode. Si 'X. est biharmonique, la fonction t:o (:j'A est ha=onique, et on peut écrire 

Nous introduisons la fonction 

avec .Al.= ft 'P = 4 Q . On obtient alors 
-\-;..f ) 2.­

et la fonction 'X. -'PI.e,-Q'X2, est harmonique, d'où 1.. = 'P'X, .. Q~2, + ~(Kl'~)) = <n ( ~ G(~) .. K(?!)] cgfd L'application de ce théorème montre que la fonction de contrainte d'un problème d'élasticité plane est déterminée par deux fonctions holomor­phes G et K On posera 

-127 ­
et les relations (9) donneront 
c
<r.... r: 1." 1.1:. + Q,u «',1, + ), Q,l, + "R,,,2. 
(32) = l' Ir. + Q,~. I.I:./, + + "R
<rU l-i-t -1 l ~. '., ~!l. = Q,A.\ «., -l''AA + :)'.A
IX." 
En re~roupant et en utilisant les relations de Cauchy (30), on obtient 
<r." + (J"~" = ~ ~[G-'('ll))
(33) 
f <ru -O:;A + ~..l. ~~ -.t [~G'(~) + K·(~) J 
L'intégration de (25) donne également 
A+V 
.J..(., = t(~ -4 )() l' -l' IX -Q'l:-1? + C'X.'" + 0( }
'A • ,A '" ,-1
(34 ) ~ 
Q «. -CIX.
{ .ut = ~+v 1n-4~)Q + -~A IX", + S'A • • + f.> l
,A •
E: l ou sous forme complexe 

les trois derniers termes représentant le mouvement de solide. Ces représen­tations sont à la base de la théorie de l'élasticité plane qui permet de pous­ser très loin les calculs (voir [19)). Nous présenterons simplement quelques exemples. 
2. EXEMPLES D'APPLICATIONS
========================== 
2.1 PROBLEME DE SAINT VENANT 
Une classe de solutions s'obtient en prenant pour X un polynôme homogène de degré n, .OU, ce qUi revient au même, 

On obtient ainsi pour tout entier TI une solution dépendant de 4 constantes. 
n=2 ! Un polynôme du second degré est automatiquement biharmonique, et con­duit à un état de contraintes constant 
'X ~ 1t IX. l:1.:,t .. Z f.> 1:>-,1:>-" .,. ~ x,," 1 
(37) 
[ 
a:;~ = t (J"tl, : 01 cr:,l, : -~ 
~ : Un polynôme du 3ème degré est aussi automatiquement biharmonique, et conduit à un état de contraintes linéaire 
-t
'X = { a. 1:1.:, ~ + ~ t. ~:oc" + ~ C-IX, oc~ + .d. X; 1 
G 
(38~ 
{ (\""\0\ =-c 1:1.:, + <toc./, <fJ.1, = a..~, + k'X../, (f"l-= -(k -:x., + ,c >:Cl,) 
-128 ­
n=4 Pour un polynôme du 4ème degré, on a 
~ 3 	t 1, ~ 
'X = 	~ tAIX, + J, B'l:.~ /.t2" -?> (A + 1») ~1 1.tJ, +.t C/.t,ft}, + 1> ();l." 1 cr-= -(A .. 1») 'L+ .l, c 'l:., 'l:.J, + 1, J> x;
f 	t 
(39) 'H 	1 
2, A 'LI, 	_ ( A <Il) IX.,t,
<T!LJ,= ~ + 1, B~I.t,t, ,t,l /.t1, C 1,
cr:;Q,~ -P-> + J., (A ... D)~• ./l:J, -4"
1 
et ainsi de suite. A titre d'application, montrons qu'une superposition de solutions de ce type permet de résoudre le problème de Saint-Venant en contraintes ou 

e. 
Le matériau occupe le rectangle [0,2] x[-{/~J t-/l]; la surface latérale «-l,=:!:f./:t est libre de contrainte, et les extrémités x)=O et x = l sont soumises à
l 
deux torseurs plans en équilibre (voir § VII.I. 1). Les conditions aux limi­teS sont donc, sur la surface latérale 

et sur l'extrémité x I= l 

Bien entendu, comme nous l'avons discuté au § VII.I.I, ce problème admet plusieurs solutions, et nous allons chercher s'il en existe une correspondant à une fonction de contrainte X(1:t.4: ) polynôme non homogène du 4ème degré,
11 1I
càd 	superposition de (37),(38) et (39). La condition aux limites (40) donne 

relations qU1 doivent être vérifiées pour tout Il vient
xI' 
~ 
= e, 	: 0 , a..=lr.O e, = A .. 'D 0 Jr s oC = 0 
(43) 
lA cl. -(A+ 1» ,!t = 0 {:> + C _ .l" 0
'" 
4 	4 
(44) C .,:c, ');~ -t-d. -:c~ 1 'X ~ C Yt."
x = --+ -t --/.IC,/.IC~
, " G " l, 4 
cr = .l,C 1.1:. ILJ, .. cl. ACt + t
~~ 
(45) 
O"~'l, = C ( ~" le:)
4 
= 0 
Les trois conditions (41) permettent alors de déterminer les constantes C , cl et r en fonction de 1', Q et M , càd des efforts appliqués 
(46) 

c = 

La répartition des contraintes normales est linéaire, COmme dans le cas général (§ VI.I.2) et la répartition des contraintes tangentielles~" est parabolique. C'est ce que l'on obtiendrait à partir de l'analyse du § VI.3 pour une section rectangulaire très large (déformation, ulanes) ou très étroite (contraintes planes). 
2.2 TRACTION D'UNE PLAQUE PERFOREE 
Pour certaines géométries simples, l'utilisation de la variable complexe permet de construire explicitement la solution d'une vaste classe 
de problèmes. C'est en particulier le cas pour les domaines intérieurs ou extérieurs limités par un cercle. A titre d'exemple, nous allons donner la 
solution qui correspond à la traction d'une plaque perforée. 
'X 4 ~~ 

--+----f-Ar:-o..---+~X1 


Si le rayon Cl. du trou est petit par rapport aux dimensions de la plaque, on peut supposer en première approximation la plaque infinie. Dans le repère (x ,x), l'état de contraintes à l'infini est donc
l 2
(47) or 
= 

ce qui, d'après (33), correspond à 
cr­
(48) ; -r 
. 4 

c'est la solution qui se réalise en l'absence de trou, mais en présence d'un trou cette solution ne vérifie pas les CL sur le trou, qui s'écrivent 
(49) 


en notant c:r",~, (J"Jt.9 ,(J'9ft les composantes du tenseur des contraintes sur le repère (.e. ..... .e.e) associé aux coordonnées cylindriques. Le trou indui t donc dans (48) une perturbation, et on montre qu'alors 

(JO' [ CL'"
(50) 

-r --­
t r 

Les contraintes sont alors données par 
(J""[ é t, 4 
_(À .;. 4IL + 3 CL) ~.2,e 1
(J"II-JI. = 
9-e -~",) 
'1,'" fi} 
4
(j'"
(51) 
(j'Le = ( -1+-ié ~ CL ) Mm. .te 9-",'" '1,4 
é 1
?> CL'"
(J"Q9 = (j" f (À + '1,"') + ( À .. '1.'" ) .c«> .2,e
t 
En particulier, sur le trou on a bien (49) et 

L'état de contraintes sur le bord du trou est un état de traction simple avec une contrainte variant de .. ~ cr" (traction, sur l'axe des xl) à a~ (compression, sur l'axe des xZ). La contrainte maximale est trois fois plus grande que la contrainte à l'infini. C'est un exemple de concen­tration de contrainte: la présence d'un trou, ou plus généralement d'un dé­faut, aussi petit soit-il, cause une augmentation importante des contraintes locales au voisinage du trou. 
Chapitre IX 
METHODES VARIATIONNELLES 

1. THEOREMES VARIATIONNELS 
========================== 
Dans tout ce chapitre, nous nous intéresserons ~ un problème sta­
tique régulier (§VI.1 .1) pour un matériau élastique linéaire quelconque (isotrope ou anisotrope) caractérisé par un tenseur d'élasticité A~i\t POUr simplifier l'écriture, nous supposerons le matériau homogène, A~itl= di et nous prendrons les CL sous la forme mixte (VI.7). Pour un autre problème 
régulier, l'écriture serait plus lourde, mais les résultats et les raison­nements seraient identiques. 
1.1 NOTIONS FONDAMENTALES 
Nous cherchons donc un champ de déplacements et un champ de con­
traintes vérifiant les équations suivantes 
( 1 ) 
(2) T.el. 
= ..
<r~i Il\.i 15f 
(3) ..((.. 1 = M..el. 
... 5.... • 

(4) 

(5) 



Parmi ces équations, certaines sont de nature statique et portE~nt 
uniquement  sur  les contraintes  -les équations  (1)  et  (2)  - d'autres  s0nt  
de  nature cinématique  et portent uniquement  sur  les déplacements  -(3)  -.  
Enfin,  un  troisième groupe  d'équations  -(4)  - relie les contraintes  Et  
les déplacements.  

N 
Définition]. Un champ de déplacements ...u...c.. est un champ cinêmatiquerncnt 
,1(.­
~ 
missible ( M.. est un CCA) s'il vérifie les conditions cinématiques (.
.. 
el.
( 6) N 
.L(..A.1 .A.l. :: ..u...4,
5
'" 
Partant d'un CCA ..u.. , on peut lui associer un champ de déformations 
.... 
N 
( .. par (5), puis un champ de contraintes cr.. par la loi de comportement (4), 
.... i .... ~ 
mais ce champ de contraintes n'a aucune raison de vérifier les conditions sta­
tiques (1) et (2). 
" 
Définition 2. Un champ de contraintes ~4t est un champ statiquement admis si­
" 
ble (cr.... est un CSA) s'il vérifie les conditions statiques (1) et (2) 
(7) + J '" 0 ,
-{.. 

" 
A Partant d'un CSA <r...t ' on peut lui associer un "champ de déforma­
tions e, .. par la loi de comportement (4), mais, puisque crOt' ne doit pas
• A.t ,.. 
vérifier les équations de Beltrami, on ne pourra en général pas calculer un 
A 
déplacement..u.. par intégration de (5). A fortiori, les conditions (3) ne 
.... 
seront elles pas vérifiées. 
Avec cette terminologie, le problème d'élasticité (1) à (5) se 
ramène à la recherche d'un CCA.IA.. et d'un CSA (1" •• reliés par la loi de

'" .... " 4. 
comportement (4). Toute la suite de ce chapitre sera basée sur le lemme 
suivant -généralisation du théorème des travaux virtuels (111.46) 
... 
.Lemme fondamental. Soit ..u.. un champ de déplacements (virtuels) quelconque
.... 
A 
et ~.. un CSA, alors
"'a 

Dem.  La  démonstration  est directement calquée  sur  celle du  §  1.2.1  .  Nous  
partons  du  premier membre  et  utilisons  la  symétrie de  (J..  
....~  

~ 
" f. .. d.nr : ~ CfA L<-" .. ·...u*.. · . JN-EAcr.. ..u. . c\;Ir
"
-=
)~ ~"'À ....~ SI. i~ (..<-,a â , ....) 
.Q .... ~ :,-,~
.Q i % cl.'Ir -~ A~ . .u-." 0"'IT
A " 
= 
1.
~(J'~à.u."')'i SI. ;'i'~ 
Par-utilisation du théorème de la divergence, le premier terme donne l'inté­
grale de surface du second membre de (8), tandis que le second terme donne l'intégrale de volume par (7). On retrouve le théorème des travaux virtuels 
A 
en prenant comme CSA (J;"i le champ solution O"Â.a 
-133 ­
...
Théorème des travaux virtuels. Pour tout champ de déplacements virtuels AL. 
" 

n'un point de vue algébrique, et en revenant à la structure dé­
crite à la fin du § 111.2.3 , on peut généraliser (111.50) en 
... A 
(10) «é,(»> = 

... A 
valable pour tout champ de déplacements AL· et tout CSA 0"..• Plus précisé­
.. kA 
ment, on a la situation suivante 
~ <-duali té < J > --. 'l1 
E j 1> 
1 
J <-dual ité <oc ~ -.. $ 
L'opérateur D est l'opérateur 
.. .. 
( Il) A . é ..
.. 
...~ 

donnant les déformations en fonction des déplacements, et l'opérateur E est 
l'opérateur 
~ 

"­
(12) ().. 
, T 2 ().. flI.. )
...
..
~ 
"'a ls 
"...
associant au champ de contraintes (j".'-â les forces volumiques t et les efforts 
A 
de surface T qUI lui correspondent. La relation (10) montre que les opéra­
... 
teurs 1) et E sont adjoints l'un de l'autre. C'est une structure que l'on retrouvera dans toutes les théories de Mécanique des Solides en petite pertur­bations. 
Il faut remarquer que, bien que nous les ayons présentés dans un contexte d'élasticité, toutes les définitions et tous les résultats de ce paragraphe sont indépendants de la loi de comportement; en particulier, on les retrouvera en plasticité. La loi de comportement se présente comme une relation entre les déformations et les contraintes ( voir § IV.I.3.). En élasticité, cette relation est une application linéaire reliant les valeurs instantanées des déformations et des contraintes, cette application étant de plus supposée symétrique (auto-adjointe) et définie positive (§ V. 1.1). 
-134 ­
1.2 LE THEOREME DE L'ENERGIE POTENTIELLE 
~ 	~ 
Soit..u,. un CCA, on calcule e·· par (5), et on peut donc définir 
4 	4~ 
l'énergie de déformation du CCA;i.. par'
4 

On introduit également le travail des efforts (volumiques et surfaciques) 
, ~ 
donnés dans le deplacement M". 
4 

Pour les CL mixtes (VI.7) choisiAe, le travail des efforts surfaciques don­nés, Tt .s'exprime simplement. Pour un problème régulier quelco~que, l'ex­pression peut être plus compliquée, mais, comme on l'a vu au § VI. 1. 1,' le travail des efforts de surface se décompose sans ambiguïté en T! et Tt (voir par exemple {VI. 10) et le § 1.4). 

...
L'énergie potentielle du CCA .u.. est 
4 
N 
1< (..u..J ­

On 	démontre alors 
Théorème de l'énergie potentielle. Parmi tous les CCA, la (ou les) solution ~. minimise l'énergie potentielle
4 

'" 
( 16) 
V..u.. CCA 
N 
...
Dem. Soit ..... une solution du problème (I) à (5), et .u.. un CCA. Nous définis­
~ 	4 
sons 
( 17) 

, 

= 0
= ...u...... 
.... 
et..u.. est un CCA pour le problème homogène associé 
K(;';:;) = ~ %.Q. A...!~J" (€.i.~ + è:1i)(Eftl+ ËivJ-d.nr 
-~	n {,., (.u.... + ~:) cW--§ T/ (..u..... + ,;:;,.:) clS ~ St 

+i m.Q. A"i~1" 

+ ~ A..~lv,., f:':à ~f..v... cm­
,0 

Hais ..1.(.. est solution, et le théorème des travaux virtuels (9) donne, C~
.. 
..u... _ .,u....
prenant * "'.
.. ... 
"'0 ... 
é .. cW
~.n.A...iU E·"'a. E~v.. .k = ~n. Cf"'â ...~ D $ i-.û:: cl,,-+ .u."'. J,:,
s:!. .. las:!. G".i.t l'fI,à <­
.,-­
'" $ t~: cW + § T..d. ;{o. tl.S .... ~ (J.. IYV ..u... tl.5
:a/ t 
n. s( 5~ .l­
puisque sur S, ona (2), et que, d'après 0) et (6), est nul sur S.u,.. Finalement, on obtient donc 
.. 
(18) 1< (.u:..) 

Or le second terme est positif, puisque la matr~ce d' élasticité est défini~ positive (§ V.I.I). Ceci démontre (16). cq f c 
On tire également de cette démonstration 
f
ThéOrème d'existence et d'unicité. Pour un problème dé' t::-;::>E:' l ou un problèfi'!t mixte, il existe une solution unique. ! Pour un problème de type Il. il existe une s()L~::.i(1n à~finit :, un !mouvement de solide rigide près SSi 
càd SSi les efforts appliqués forment un torseur nul. 
Dem. Unicité. Soit .u.:-et..ll.~ deux solutions, ils sont aussi ceA, et l'appli­
.. .. 
cation de (16) montre que 



soit, d'après (18) 

Il en résulte, puisque A;'i\t est défini positif 

et les deux solutions ne diffèrent que d'un mouvement de solide. Si 5 ~ existe 

~ l-
permettent de montrer que 
et donc ..lLA.-=.J.L,è,..' d'où l'unicité. 
Par contre, si S est vide (problème de type II), on ne peut plus éliminer 
.... 
ce mouvement de solide qui reste indéterminé. 
Existence. L'existence d'une solution peut par exemple se démontrer en cons­truisant, dans un espace fonctionnel approprié, une suite minimisante pour 
la fonctionnelle K -voir cours de Mathématiques -. Pour un problème de type II, la condition d'équilibre (19) apparaît naturellement, car sinon la 
fonctionnelle n'est pas minorée. Pour les autres prob~èmes, cette condition 
d'équilibre n'apparaît pas, car les efforts donnés sont équilibrés par les efforts de liaison (inconnus a priori) s'exerçant à travers S ...... 
1.3 LE THEOREME DE L'ENERGIE COMPLEMENTAIRE 
... 
Si nous partons cl' un CSA (j.. , nous définissons son énergie de 
""~ 
déformation par 

et son travail dans les déplacements donnés par 
ci A
(23) T (T".. ) = 
.u. ""â 

et nous définissons 
A 
Définition. L'énergie complémentaire du CSA 0" .. est 
... t 

cl. ~ 
[ (24) 
T (0".. ) 
.... ""â 
et on obtient le résultat suivant 

Théorème de l'énergie complémentaire. Parmi tous les CSA, la solution <r~~ 
maximise l'énergie complémentaire 

Vrr.· C5A 
~a 
A 
Dem'. 50 i t (J.. la solution, <J .. un CSA, et posons 
~a ~~ 
~ 
"'.
(26) (J CT•• ... CT· • 
~i 
4a
= "'a 

est un CSA pour le problème hOlllOgène associé ( ~:= 0 

-137 ­
'" 0 ".
° 
(27) If... =-iL ",0 , ~ 0 
.(.i'~ ~~ f1\.à 1Sf ~ 
H(cr;) " -~ ])1). fI-'-i H (G"1 + ~.i.i) ((J~~ + ~i~î o..'lr 
+ §S (G-,-~ + éf;.·~ )lY1.~ -U-~ c\,S 
oU. 
= H(G) _-: crr !I.r,.1.. Œ.O. Œ-~. drtr
Nâ ~ JjJ n. ka ..~,...'" 

"0
En appliquant le lennne fondamental à (J.. , CSA pour le problème homogène
"'il associé, èt à ~. , déplacement solution, on obtient 
.... 

" ~~.Q6-~à é;'À d.v 

+ ~d.S 

Les deux premiers termes disparaissent d'après (27), tandis que sur S~ 
d.
-U-. ".u,. par (3). Il vient finalement 
~ N 

d'où la conclusion, pU1sque, comme A-'-â~t' la matrice ":"à\\.9,. est définie po­
sitive. 
Théorème de comparaison! Soit (..u.,., (J.. ) la solution d'un problème régulier, 

<'V ..A Jo, ....... ~ 

..(.t.. un CCA et (J".. un CSA 
"" "'"3 

Les deux théorèmes de l'énergie potentielle et de l'énergie com­plémentaire permettent d'écrire les deux inégalités. Il reste donc à montrer 
l'égalité 
(30) 

'" . Il. ( CS· . \ 
""Il J 
pour la solution. 
Dem. Pour la solution, on a 
"" 

E..
W(JJ..;.) = W(CS-'-"t) 
""3 
A partir de (15) et (24), il vient alors 
-138 ­


-r lJ.. 1Y1.. ..u.. cLS 
'" 0
~è.Q "'â il ... 
comme il résulte du théorème des travaux virtuels (9), en prenant comme dé­placement virtuel le déplacement sol'ution .(.l,.• cqfd
'" 
Au passage nous aVOns démontré 
Théorème du travail. Dans un problème élastostatique, l'énergie de déforma­tion est égale à la moitié du travail des efforts extérieurs dans le dépla­cement solution. 

On peut d'ailleurs obtenir directement ce résultat par une appro­che de type énergétique. Partons en effet du bilan énergétique en élasticité du § VI. 1.2 -équation (VI.14) ­
cL.u. cL
+ <1". • 1Y1.. --~ S
§
ôSl. "'â a o.t 
Pour un problème quasi-statique, on néglige les variations d'énergie cinéti­que, et on obtiendra l'énergie de déformation associée à (4., 6· . ) par in­
... "'II 
tégration de (32) par rapport au temps sur un processus quasi-statique fai­
sant passer de l'état de référence (..u-. = 0, <r.. =0, W= 0) à l'état final 
'" A.t 


Or, on peut obtenir un tel processus par un chargement proportionnel: d'après la linéarité, (),JA,., Â cs" ) est la solution quasi-statique ou statique asso­
....* 
c iée  aux  données  ... (').'f..  '  '). T...d.  , ') ..u.t ). L' é ta t  de  référence correspond  a lors  à  
'"  '" 0  et  l'état  final  à  ").  =1.  .  On  obtient alors  (d.u.. ~ ...  AL· N  cL))  
W  '"  ~ ~  l N.Q~ t .l.l. 0N­ + K Â (f"i lY1.à Al... ô.a  cLS l ~  J.)  


cÂAr -t (\ (J. . 1Y1.. AL. cLS 1 
t %n t AL" .l)ÔQ ~~ ~ '" ~ 
Le coefficient 1/2 dans (31) traduit donc physiquement la mlse en charge pro­~ressive du Qilieu. 
1.4 APPLICATION A LA TORSION 
Ainsi, le théorème de comparaison permet un encadremenE de la 50­lut~: -: par des solutions approchées. A titre d'application, nous allons rncn-' 
trer -:ormnent il permet d'encadrer le module de rigidité à la torsion d'un arbre cylindrique (§ VII.2). Nous avons formulé au § VII.2.3 le problème ré­gûlier le plus commode correspondant à cette sollicitation 
x, 

SI. ; [0,1) ;<. ~ 
X 1 
t 
=0 
Sf-': [o).t] x dt (J'. ' rn.. : 0 
"'li' à 

(33) L 0 -:c.. : 0 cr11 = 0 J.l,2. =-U.., = 0
j 
LIX, =t (J'--I--I = 0 .Ll.l,. = -O(.Q I,l:~ .M-~= Of 1 'x­
1 l, 
~ 
Un CSA (J.. doit vérifier les équations d'équilibre et les CL de 
.4~ 
type statique, à savoir 
~ 
[ 
<J, m.. = 0 sur [O,~J x "ôL
Li Ô' 
(34) A 0 en -X, =0 -X , =1
~1 " 
Nous inspirant de la solution du § VII.2.2, nous prenons un ch3mp de contraintes de la forme (VII.48), Les CL (34) sur les extrémités so~t alors automatiquement vérifiées. Conune au § VII.2.2, les équations d'équi­libre permettent d'introduire une fonction de contrainte ~ et les CL (:.i) sur la surface la téraI e exigent que p sa i t null e sur () L . Par con t :-0:', pour un CSA, la fonction ~ ne doit pas vérifier l'êquation (VII.3~) ~\Jl Lésul~ait des équations de Beltrami. Ainsi, un CSA est défini par une :C~L­
~ 
tion 1I\(t:J; ~) nulle sur 'OL avec
':! .L"~3 
A 
A 

~.. ~~Ol "-~
o~ è<D
(35 ) (if ~ 0 ! ~­
r~--I~ ~ 0 (j ---F. 
--1" '0 tr;" ,: 3.1:
-:c~
0 0 1 1
l~, 
0 
Pour ..:..:flculer l' énergie de déformation, en élasticité isotrope, on ...it':': ~::--::: les formules suivantes, qui s'obtiennent directement à partir des fc~~ du chapitre V 

(36) 
-140 ­
et à partir de (35) on obtient 

.!. cl.
Pour calculer T.... et Tf il faut expliciter la décomposi~ion (VI.8) pour le problème (33) 

r. 
'~---------------------/ 
T cl. 
(r;..)
.LI., ... il 
et compte-tenu de la valeur des données (33) 
cl. . 
(38) 
T{ (.u....) " 0 

(39) 
T: (cr-"'t) ,. cd, § ( ,x,t cr-~ .. 


1:

1 
"­
où ~1 est le moment de torsion résultant des efforts associés au CSA Compte-tenu de (35), il vient 
" 
A 
T6. (A ô~ 
.u. (}:..il) -cd, o<l! ) d.tc~ ck..
" ))1: (rL~ + l.C."
'OrL,t '(1):3 
9, 01.1 A 
= ip ck,tdt.cô
~r. 
en reprenant le calcul de (VII.63). Finalement 

où 'fA est une fonction de x2,xnulle sur 'dL
3 
N 
Un CCA .u.;.. doit unique~ent vérifier les CL de type cinématique 

" .v 
..u.~ ~ ..u. .. " 0 
(42) f 
;;.~ = -I)/..Q.~.. 

et, en nous inspirant de la structure (VII.76) de la solution, nous prenons pour CCA le champ 

qui vérifie automatiquement (42). Un CCA Sera donc défini par une fonction quelconque (effectivement, les restrictions (VII.78) imposé~s à lr pour la solution sont d'origine statique) . On a alors 
.. 
o'f>
0 -OC +
! 
'0 'l:.l, 

cl.
(44) ~ 
-I:C -t ~ 0 o
E = 
3J 
~ 3x~ 
'à~ 
-X.l,+~ 0 o 'à 'l:~ 
et on tire de (36) 

soit finalement, grâce à (38), 

Ainsi, compte tenu de (46) et (41), le théorème de comparaison permet d'encadrer 11(4...) =I«(Ï....~). Compte-tenu de (38) et (39), on a pour la sol ution 
(47) . li (..u.J = K«Ï...~) = W = 0(1, '1l\.~ -w 
compte-cenu de (VII.64). Le théorème de comparaison nous donne donc 


l 
(48) 


valable pour toute fonction 'fA nulle sur aE et toute fonction "i'.. . On voit 
donc que l'on peut encadrer le module de rigidité à la torsion et obtenir ain2i des valeurs approchées. 
On peut ainsi démontrer certains résultats généraux: par exemple,
.. 
en prenant y = 0 on obtient 

le moment polaire Io est un minorant du module de rigidité à la torsion (on a vu que c'était le module de rigidité à la torsion pour une section circulaire ou annulaire). 
Pour aller plus loin, considérons par exemple le cas d'une sec­tion rectangulaire. On a vu au § VII.2.4 que l'on pouvait obtenir une so­lution exacte par développement en série de Fourier. Les calculs précédents 
"­
vont nous fournir une valeur approchée. La fonction <f .doit être nulle Sur le bord, nous prenons 

On trouve alors par un calcul direct 

~ 
La fane tian "f es t quelconque par analogie avec la section elliptiqu~, nous prenons 
(52) 

et nous obtenons 

d'où l'encadrement (48) pour l . Pour obtenir l'encadrement optimal, nous 'choisissons la valeur .de 'Yn, qui maximise ~(~) et la-valeur de t qui mi­nimise ~~) . On trouve 
~ j,.~
:; (l. ­
=
1YY1.~ 
4(a,~t-%") tf = a-" t-Ji' 40 a-' ~3 4~ 0,; ~~ 
(54) ~ l .;
-
3 o."+-ir~ 9 é +-.,e,L 
En particulier pour la section carrée, on trouve 
(55 ) 0,139 

0, 167 
alors que la valeur exacte est de 0,141 . Bien entendu., on pourrait raffi­ner en prenant des fonctions A et l' ~ plus compliquées. Néanmoins, on
'f 
voit que notre CSA est déjà assez proche de la solution, et peut nous donner une approximation raisonnable du champ de contraintes réel. 
2. LES THEOREMES DE L'ENERGIE 
============================= 
2.1 LE THEOREME DE RECIPROCITE 
On considère un solide.. l.lastique pouvant être soumis à deux char­'f<f-S < <' 1) (t. .L) lIt'
gements dl erents. Olt lJJ-;.,<fct ......;.,(l'~~ es 50 u lOns correspon­
j
dantes. 
Théorème de réciprocité au de ~!ax-we.ll~Betti. Le travail des efforts exté­
Tleurs 2 dans le dênla2e~ent ~ es~ ê~al au travail des efforts extérieurs 

1 dans le déplacement 2. 
l(56) 


Dem. On utilise le théorème des h'a\'aux virtuels appliqué au problème 1 avec comme déplacement virtuel·}~ c2placement ~~ solution du problème 2. 
A. 
On obtient 

!arc 1 t,
+-< T<.!.l. c:l.S
\ ... .t-
On effectue la même opération en char.g~ant 1 et 2 

et on obtient (56) en remarquant que. ~'après la symétrie de la matrice cl' élasticité 

A titre d'exemple dlapp~i.::atl,_n, C'îDsidérons un problème du type !.!..? avec des données (i.i..' T...,d.). EY""_ .;;':néral, ~)!""! ne saura pas calculer la so­lution (~L' cr-:"â-). Par contre, c'2rtai:1s pr"'Jlè,,,es de type II peuvent être résolus pour le même domaine, par exemple teus ceux qUl admettent une solu­tion homogène: le problème caractérisé par les données 

où (j.o. est constant, admet en effet 14 solution 
A.a 

On applique le thécceO\e de Max""e\ l Betti en prenant connne problème 
-144 ­
1 le problème posé, et 	comme problème 2 le problème (57) avec sa solution 
(58) : 

Mais l'utilisation du théorème de la divergence donne 

et (58) donne des informations sur la valeur moyenne des déformations 

Par exemple, on obtiendra la valeur moyenne de E.(.( et €... ~ en prenant pour 
•
~Là un tenseur de traction simple et de cisaillement simple. En élasticité 
isotrope on obtient 
'" -1 ~f ~ [t~~ -vq~~.l.+t~~'!,)1 èw­
1/ E l ~ 
(60) 	
+ t~[T.cl.J.!., -V ( TJ,d./;!.~ -t T:/;!.3 î} d.,5 1 = ~ A+V r)j) (Î'...:: + t.2.:\t..) M t" (\ (T//;!.~ 1" \d.J.!.,) d.,s 1. 

V E SI. ~:o .ll".Q. 	J 
En particulier, la variation de volume est donnée par 

Plus généralement, le 	théorème de Maxwell Betti permet souvent 
d'obtenir sans calcul 	des résultats intéressants. 
2.2 LE THEOREt1E DE CASTIGLIANO 
On considère encore le même solide élastique pouvant être so,~is à deux systèmes de chargements 1 et 2. 
A ~ 
Théorème de Castigliano. Soit cr·· un CSA pour le problème 2. Le travail 
...~ 
des efforts extérieurs 2 dans le déplacement 1 est égal à la dérivée à l'origine de la fonction donnant l'énergie de déformation du champ de 
contraintes 

en fonction de ÎI 

Dem. On développe 

De sorte que 

A t­
(J.. , CSA pour le problème 2, et 
.(.~ 
cqfd 
d'après oU.--1 • 
.(. 
L'utilisation de ce théorème et du théorème de réciprocité est basée sur le fait que, en introduisant comme chargement 2 des chargements fictifs, ils permettent de calculer certains déplacements ou déformations. En effet, si on introduit par exemple comme chargement 2 une force concen­
->
trée  F  appliquée  a~  point M,  alors  le  travail  du  chargement  2 dans  le dé­ 
placement  1  se  réduit  à  
(63)  

d'où le calcul du 9éplacement du point M pour le problème 1. Ce type de mé­thode est peu ütilisé en MMe, pour deux raisons: 
1. 
L'introduction de forces concentrées en ~1C pose quelques problèmes liés à la singularité du chargement. On sait résoudre ces problèmes, mais ce n'est pas si simple. 

2. 
En MMC, il est en général très difficile de calculer le champ de contrain­tes solution ou de construire un CSA. 


Par contre, ces "théorèmes de l'énergie" -comme sont CQUram;;lcnt nommés le théorème de réciprocité et le théorème de Castigliano -seront utilisés de manière intensive en Résistance des Matériaux, où les deux dif­ficultés mentionnées ci-dessus disparaissent. De manière générale en effet, tous les théorèmes que nous avons dérnr rés depuis le début de ce chapitre 
sont valables pour toute théorie d. lieux continus élastiques. En fait, 
ils reposent sur la structure air _que décrite à la fin du § 1.1 
Ces théorèmes ne sont en prlDClpe valables que pour les problèmes réguliers. Pour un problème non régulier problème avec frottement ou avec contact unilatéral par exemple, voir § VI.l.1 -on peut avoir des résultats analogues, mais il convient de tout reprendre pour chaque cas particulier. C'est le champ d'étude des méthodes variationnelles ([24). 
3. LA ~ŒTHODE DES ELE~NTS FINIS 
======~========================= 
3.1 PRINCIPE DE LA METHODE 
Les théorèmes du § 1 énoncent des principes variationnels dont l'affirmation type est la suivante: "La solution minimise une certaine fonc­tionnelle dans un espace de fonctions admissibles"~ Nous avons présenté les deux principes variationnels traditionnels, mais il en existe bien d'autres, plus ou moins appropriés, suivant le type de problème que l'on envisage [21). L'intérêt de ces principes variationnels réside dans le fait qu'ils engen­drent à peu près automatiquement une méthode numérique pour calculer une so­lution approchée. Il suffit en effet de discrétiser l'espace des fonctions admissibles -càd de l'approcher par un espace de dimension finie -et de 
minimiser la fonctionnelle sur cet espace discrétisé. On obtient ainsi une 
solution approchée d'autant plus proche de la solution réelle que l'espace discrétisé approche mieux l'espace des fonctions admissibles. 
Pour discrétiser l'espace des fonctions admissibles, on peut par exemple introduire une base fonctionnelle de cet espace -base de fonctions sinusoïdales pour un domaine rectangulaire, par exemple -et approcher l'es­pace des fonctions admissibles par l'espace engendré par les TI premlers éléments de cette base. C'est la méthode de Galerkin, et on peut montrer que lorsque n-+M , la solution approchée ainsi calculée tend vers la solution réelle. 
Le tenn~ "méthode d'éléments finis" recouvre un ensemble de métho­des pour lesquelles l'espace discrétisé s'obtient 
1. 
en découpant le domaine .rL en un certain nombre de sous-domaines s lm­pIes (triangles ou rectangles): les éléments finis; 

2. 
en prenant sur chaque élément une forme analytique simple, de sorte que la valeur de la fonction en tout point est donnée par sa valeur en un ~()~hre 


limité de noeuds. A titre d'exemple, nous allons présenter les trois élé­
ments  les  plus simples  pour  un  espace  de  fonctions  à  valeur sCdlaire  (pour  
une  fonction  à  valeur vectorielle  ou  tensorielle,  il  suffit de  ~Jnsidérer  
séparément chaque composante)  dans  le plan.  

Exemple J. Elément triangulaire avec fonction linéaire sur chaque triangle 
X'1 ~,
i = a. + Jr/.l:1 .. .c.oc~ 
1____---->;.> 
La val eur de la fone tion t en un point du triangle ABC dépend de 3 paramètres, par exemple la va~eur de f X1 aux trois sorrnuets 


sont les "coordonnées barycentriques" du point H. 
=l: 
Exemple 2. Elément rectangulaire l 
C-
avec fonction bilinéaire 
C] b 
i" a..+l-1)o1" ~4:l-1-cll.c~4JJ 
A e 
La valeur de la fonction ~ en 
.1:, 
un point M du rectangle ABCD dépend de 4 paramètres: la valeur t aux
de 
noeuds A, B, C, D. 
(41 -.lC~)(o!C...-~) (J.C,-I.e:) (4.. -k~)
{ = tA + -tB
(Jee.f -).(,'0) (JJ:~ -~~) (~~~ _ ~,A) (IY-f -!:Cf) 
(65) 
(IY-, -J.C~) (11;.,-.I!lf) ().c, -.K,c) ( (:i;" -/:C~)
+ + tl)
ic (1:1.,' -,<Dl (4: -IY-t) (bOJ'-IY-n (JJ:{-J1-~) 
Exemple 3. Elérnent triangulaire avec fone tian quadra tique sur ..::naque trian­gle 

La valeur de la fone tian sur le triangle ABC
f 
·dépend de 6 paramètres: la val eur de { aux 6 
noeuds A, B, C, D, E, F, et ainsi de suite. On 
trouvera dans la littérature sur les éléments 
finis des "catalogues" d'éléments. 
) 
Après aVOlr découpé le modèle et choisi les éléments~ ,---ne fonction de l'espace discrétisé est définie par sa valeur en un certaiTl ~Jrnbre J~ noeuds, et on doit minimiser une fonction d'un nombre fini de '"'l~'iables 1 rro­blème qui se prête bien au calcul numérique. 

-l~g ­
PO'H lA torsion, par exemple, on pC<Jt à partir de (48) construire dc.:-'.x ::,':'::hodes d'§léments finis 1ère ~étbode. On discrétise 1. , et on rr.axir.lise la fcnctionn211e \\.(~) sur 
"­
l'espace des fonc:ticns <f nulles s'J.r le bord. 
2ème méthode. On discrétise L. , et on minimise la fonctionnelle ~(~) sur 
l'espace de toutes les f0uctions ~ 

3.2 APPLICATION A UN EXEMPLE 
Pour montrer la mise en 0euvre de la méthode, nous allons enVlsa­ger un exemple. Considérons en déIormations planes un barrage triaQgulaire 

Les forces de volume se réduisent à la pesanteur 
(67) 

p étant la masse volumique du béton. Quant aux conditions aux limites, elles traduisent ltabsence de contrainte sur OB, l'effet de la pression hydrostatique -(j)f'J:." (Ci') étant le poids spé­cifique de l'eau) sur OA, et la liaison supposée rigide avec le sol sur AB 
OB =' 0 ",,0
0:;1 ... <r-l~ "~,, ... cr~~ 
(68) OA cr "" (j) f'J:..ù 0:,,, "" 0
" 
AB Mo-0
, = .-U.,~ =
i 
Le théorème de l'énergie potentielle affirme que dans l'espace des CCA, 
( 69) 

~ 0 

la solution minimise la fonctionnelle 

Pour discrétiser ce problème, nous introduisons un découpage en éléments finis 

Par exemple, nous introduisons un maillage régulier et des éléments trian­
gulaires du type (64), et nous approchons 'U, par '/..6n = [ (ul,u), u1 et unuls sur AB et de la forme (64) sur
22 chaque triangle] 
Ainsi, une fonction de tG sera caractérisée par sa valeur aux 
"" 

n(n+J)/2 points du maillage non situés sur AB (si DA et AB sont découpés en n intervalles égaux). Ainsi, l'espace ~~ est un espace de dimension n(n+I), et une fonction de·~ sera caractérisée par un vecteur colonne 
U de n(n+l) éléments donnant "" les deux composantes de ..u.. en chaque point
<­
du maillage, la formule (64) donnant alors la valeur de ~. en tout point.
h 
Si l'on reporte cette fonction dans la définition (70) ou (15) de l'éner­gie potentielle, l'énergie de déformation \V(€~À) devient une forme qua­
• "'cl
dratique par rapport aux cCTIl?osantes de \! et le travail des efforts Ti 
devient une forme linéaire par rapport à ces mêmes composantes. Ainsi 
(7 J) 

= 

où la matrice carrée of-est la matrice de rigidité, symétrique et définie 
positive, et où le vecteur colonne ~ caractérise les efforts extérieurs. Pour minimiser l'énergie potentielle, il faut donc résoudre le système 
linéaire 
(72) JI: u F 
Nous utilisons leI le théorème de l'énergie potentielle, car en 
HHC il est en général très difficile d'engendrer des champs statiquement 
admissibles, et le théorème de l'énergie complémentaire est peu utilisé dans ce contexte. 
3.3 ETUDE D'UN ELEMŒNT 
Pour calculer les intégrales qui interviennent dans (70), il [dU­dra sonnner les contributions de chaque triangle pour les intégrales de sur face, et de chaque segment pour les intégrales de ligne. Nous allons donc 
-150 ­
dêjà examiner la contribution d'~n ~lê~ent triang~laire abc à l'€nergie de déforw.3tiony au travail des efforts de volume, et au travail des efforts de surface. 
Sur chaque élément, nous pouvons donc écrire le déplacement en fonction de la/valeur aux sommets d, b et c de cet élément 

"ù Âa.' 'À.e. et Â" sont les coordonnées barycentriques du point 
définies par 
ÂQ 
(74) 
?'j,. 
Â" 
et qui dépendent 
-1 
~ 
-1
-i -{ -i 
Ir
J.lCo. 
4, K'", 
J.lC
-
~ 
.. 
J.lC.0-J.J:.Ir ",,-'" 
4.,
J,
fi. ~ 
uniquement de la géométrie de l'élément . (73) 
(75) ,: 
.u;<L »..Jr JL'" 
. 2,
[~:1 
2, 
~ 
a) Discrétisation de l'énergie. 
)L, 4,
[.'L: .Ir ~ 1 
Le tenseur des 
A
[EH é"
(76) 
~
1 
[:~
E'l-EJ,~ 
AS 
désigne la matrice 
= 
déformations 
.Ir 
4, ..u.:] 
.u;.&­
..u.~'<' 
~ 
A symétrJ SéF' 
= 
..-\ -1 
r-{ [. -1 
'J.. 
."( /.<:, I.X;"', 
4, 
/.<:.Ir '. <­
L 1. "./, 
[' 
~l. 
C· . -:ilt, ,; constant et 
"i " 
-1 1 
0 0 
-I.er
/f." '" 
-1 0
, :<:, 4, 
,
,. i 
l.X;a 
0 -i
-X, 
-x:J
·t 
(77) 
= 
= LI intégratian de l' énerg ie de déforma tion r: ~e tl Langle abc 


(xI ,x2), 
donne donc 
donné par 
r 

1 
) 
donne 
(78) 

où  ~  désigne le vecteur  colonne  des  dé~  'lcements  aux  noeuds,  donc  
sous-vec teur  de  U  ,  et où  'lne  éitrice  symétrique  6x6  qui  
sul te  de  l' il ~·êgl-:11­ ·'il  sur  le  triang  et  li.  d2j?cnd  donc  unique.:1ent  
la  géométrie  ~ /  t­-i:,  gle.  
Si  l'e.  dE.  ~"c  la  forme  ql  itit  ~  W {]a.,,«­ par  rapport  aux  

un ré­de 
dé­
-151 ­
.­
placements des noeuds, on obtient un vecteur colonne r 
~b 
(79) a .u 
= a.&..c.. 
f~~c.'---~bc:-o....:...~ 
qUI peut s'interpréter énergétiquement comme donnant les "forces élastiques 
{a.., [.2,-, l,(, ", qui 
doivent être appliquées aux noeuds pour produire le dé­
placement ~ . Ceci résulte par exemple du bilan énergétique (32) qui montre 
que pour une variation dQ des déplacements aux noeuds, la variation de 
l'énergie de déformation 
(80) 
est égale au travail des efforts exercés sur l'élément. On peut donc inter­préter (80) comme donnant le travail des "forces élastiques" CsuPPo3ées 
... 
concentrées aux noeuds) ia., ~h et ~~. Il faut bien garder à l'esprit, 
cependant, qu'il ne s'agit là que d'une interprétation, et que ces "forces 
élastiques" sont fictives et sans aucune existence réelle; ce ne sont pas 
des forces, malS des dérivées de l'énergie de déformation. 
b) Discrétisation des forces de volume. Calculons la contribution du triangle abc au travail des forces de volume, càd dans (70) des forces de pesanteur. En reportant (73) dans 
(70) et en remarquant que 
(81 ) 
on obtiendra 
(82) _ _ œ (.u: + 4;+"'< ) = 
) 
(83 ) 1 -œ 1 -P'à'S 1 3 ~ 
D'un point de vue énergétique, on peut interpréter (82) en disant que les 
---1'0.. -"'t--">
forces de volume sont équivalentes aux trois forces concentrées 'f ,tt' ,~,c.. appliquées aux noeuds. 
De manière générale, l'écriture du travail des efforts de volume conduit à décomposer ces efforts en trois forces concentrées appliquées aux noeuds. 
-152 ­c) Discrétisation des efforts surfaciques 
Soit ac un côté du triangle abc appartenant à 

les efforts surfaciques sont donnés 
X't -t----------', :r., 

b
b
" 
La restriction de (75) à ac donne, dans le cas particulier du barrage (ac vertical) 

De sorte que la contribution de ac au travail des efforts surfaciques est 

et on obtient finalement 

= "fT ..u. 
(86) 'fT = (-~ ll~ (-xt + '\~) 1 0 

D, 0)
, D,
1 ­
~ 
D'un point de  vue  énergétique,  les  forces  de  contact  exercées  
sur  ac  sont  équivalentes  à deux  forces  ~ concentrees  ---. Q."t'  et  -1> ,"1'  exercées  
aux  noeuds.  Par contre,  ~~ 4  n'intervient  pas  dans  (84),  et donc  -tr'-1'=0.  
3.4 ASSEMBLAGE  

Pour calculer Kc.iL.), càd pour expliciter (71), on doit soonner la contribution de tous les triangles pour l'énergie de déformation et le 
travail des forces de volume, ainsi que la contribution de tous les segments de S{ pour le travail des forces de surface. Pour minimiser la fonction K(!:!) résul tante, il faut annuller la dérivée de K par rapport à chaque déplacement de chaque noeud. 
a) Cas d'un noeud intérieur a. 

Chaque noeud intérieur appartient à n triangles T).T•... T (n=6 dans notre
2 n exemple). D'après ce qu'on a vu au paragraphe précédent, le déplacement ~~ 
N 
n'interviendra dans K que par la contribution de ces n triangles. On pour­ra donc écrire 

o Wr.

(87) 


--' + ... 
'(Lu;"< 

.p~.,T..
où les quantités fi... d sont les composantes des forces élastiques appli­quées en a sur chacun des éléments T1,T"" T entourant a .
Zn 
Q.., T" .0.., T......
0' § p~ ;;:" 0...:. cioc~ '" 0 = -( tf~ T ." + '1\ ) 
.Q
(88) 	'0-<" 3 E.î' "") ( ..,T, o., T"" ) 
fL..r 3-il-~ c.k. 04), = ~ (5 T ...... 5 '" -<f~ ... '-. ... Cf~
Ô..llQ. 
" 
0.. T· 
où les quantités ~~' â sont les forces concentrées en a équivalentes aux forces de volume exercées sur chacun des éléments TJ,T"" T " Enfin,
Zn ~~ n'interviendra pas dans le travail des efforts de contact, puisque ce­
... 
lui-ci ne 	fait apparaître que les déplacements des noeuds de Sf. Ainsi, la minimisation de K par rapport à .JJ..~ pourra s'écrire
• 
_r<L,T~
(89) 

= 0 
On peut interpréter cette équation comme exprimant l'équilibre au noeUG a sous l'action des forces qui lui sont appliquées -forces élastiques exercées par tous les élémetits entourant a, qUl sont égales à l'opposé des fprces élastiques rQ..,T~ exercées sur l'élément; -+ -'1: -?T 
-la force 'fa. = tfCl, "'1 ... _" + <fa.., """ qui est la force concentrée équivalente en a aux forces volumiques appliquées aux éléments entourant a. Dans notre exemple, cette force est égale au tiers (car les trois sommets de chaque triangle participent) du poids des éléments entourant a. 
b) Cas d'un noeud frontière. càdg'un noeud appartenant à (car sur SM.
• le déplacement est donné). L'analyse précédente reste valable. mais il 
faut rajouter la contribution des 2 segments de Sf lSSUS de a. <\. 
On obtient alors 

3 Q., T1 
.., \f.Q,T~
(90) l'JJ (C:l, 4, ch,), = ­
OA 
~ 
• ~
(lA'" JCf 
et finalement, la minimisation de K donne 
(91) 

o
+ 

= 
On peut encore interpréter (91) comme exprimant l'équilibre du noeud a, à appliquées

qui est la force concentrée équiva­
lente en a aux forces de cont~ct appliquées aux éléments entourant a. 
Ainsi, on peut interpréter la méthode des éléments finis comme exprimant l'éguilibre des tous les noeuds sous l'action des forces élas­tigues exercées par chaque élément, et des forces extérieures (volumiques ou de contact) appliquées, ces forces étant rapportées aux noeuds. 
-155 ­
Chapitre X 
PLASTICITE CLASSIQUE 
1. LOIS DE CO~1PORTEMENT 
======================= 
1.1 LE COMPORTENENT PLASTIQUE 
Pour les métaux, comme on l'a vu au § IV.2.1, le comportement est élastique jusqu'à un certain seuil. Au delà de ce seuil de limite d'élasti­cité ou de plasticité, le comportement devient plastique, ce qui se traduit en particulier par une non linéarité de la courbe de traction et par une irréversibilité 


Au départ, le matériau se comporte comme un matériau élastique, tant que l'on ne sort pas du domaine élastique initial 
( 1) 
0-;..' < CT < (J/\ 

avec en général (fA' = -<SA Si l'on charge au delà du seuil, .alors apparalS­sent les déformations plastiques. Si, arrivé au point B, on relâche la con­trainte, alors on redescend le long de la droiteBB', et le comportement re­devient élastique, avec une déformation résiduelle E.~ , et tant que l'on reste dans le nouveau domaine élastique, on a 

avec. en général Û"1O' t -<re, et même \<re,'\<\<rA'\ (effet Bausc.hinger). 
Ainsi, on peut à chaque instant décomposer la déformation f.. en Ee
une partie élastique et une partie plastique ou résiduelle Et-. Dans le domaine élastique, la déformation plastique reste constante 

tandis que si Iton est sur le seuil, rr:.cr~ par exemple, alors on peut avoir un processus de charge avec déformation plastique ou un processus de décharge (retour dans la zone élastique) sans déformation plastique 

la constante K étant un "module d'écrouissageft , 
On peut simplifier ce comportement élasto-plastique avec écrouis­
sage par un 	comportement idéalisé. Le modèle le plus simple est celui de 
la plasticité parfaite, càd sans écrouissage-. Le domaine élastique reste 

c'est le modèle rhéologique décrit au § IV.2.2 . D'après (IV.39) et (IV.40), la loi de comportement de ce modèle peut s'écrire sous la forme 
(5) 	(J" E E" lO"I ~a;. é:: e6 + g-\' et si 1<1\ <: <rD ou 1Cl" 1=<ro laf <: 0
~ 0 
(6) 	if-~ 0 si 0-=cr. Cf . :: 0 
charge
•
(i l' ~ 0 si 0-	(j 1
s: -~ " 0 
Par analogie 	avec ce que nous avons fait au § IV.I.2 pour les lois de frot­
tement (IV.14) ou pour les conditions de contact unilatéral (IV.23), nous pouvons réécrire (6) sous uneforme plus synthétique 
(7) 	ft ... ~"'I. (cr)
= " 
[ 
si 	'). ,,0
\crI <: <rD 
0 
(8) si 1\T\ = <1". , 'À~O 1er\' ~O ').<r 
'" 
On utilise aussi parfois, notamment pour des problèmes qUi suppo­sent de grandes déformations plastiques (mise en forme des métaux), l'appro­ximation rigide-plastique, qui revient à négliger les déformations élasti­
()
ques. 
1--/-------';,t 


avec écrouissage 	parfaitement plastique 
1.2 PLASTICITE PARFAITE 
Nous nous limiterons désormais à la plasticité parfaite, qui con­
duit à la théorie classique de la plasticité. Cette théorie peut en effet 
être poussée assez loin, et permet d'obtenir de nombreux résultats. 
La définition du seuil de plasticité dans une théorie tridimension­nelle a fait l'objet du § V.3 . De manière générale, le domaine élastique est défini par 
(9) 

où i est la fonction seuil ou fonction de fluage. En plasticité parfaite, le domaine élastique ne change pas, et cette fonction est définie une fois pour toutes. Dans le cas isotrope, on a discuté au § V.3.1 la forme que pre­
nait ce critère pour les métaux 
(10) = 0
i(Â':~ ) i(J~, J 3 ) ~ 
et les deux cri tères les plus utilisés sont le critère de von Mises 
(j,t
Â
(1 1) .h . . A .. , e. 
'1 ...~
2 2> 
et le critère de Tresca (§ V. 3.2) 
( 12) (cr;, -<Tl ) $ (fe ( (f, ~ <rJ, ~ (J3 ) 
Pour les sols, la pression hydrostatique intervient essentielle­ment dans le critère, et on obtient en général de bons résultats en prenant le critère de Coulomb 
[C'i"" " COOIO-: 
(13) 1 T~ 1 < .c. 

où ..e. est "la cohésion" et if 111' angle de frottement interne". En particu­lier, pour un sol sans cohésion, ~= 0 (cas des matériaux granulaires com­
me le sable), le critère de Coulomb exprime simplement une loi de frotte­ment coulombien sur chaque facette. 

TI:: C'est La critère du type (V.62Y, 
courbE: 
la courbe intrinsèque étant une 
intrinsèque 
droite. 
T" 
Comme dans le cas unidimensionnel, nous décomposons la déformation 
en une partie élastique et une partie plastique 
-1 se ­

(14 ) 

et la contrainte est donnée par une loi élastique en fonction desdéforma­tions élastiques 
(15 ) 

et il reste à compléter la théorie par une loi d'écoulement plastique don­nant l'évolution de la déformation plastique au cours du temps. Il convient donc de g~néraliser la loi (7),(8) du cas unidimensionnel. Auparavant, re­prenons, dans le cas de l'élasta-plasticité, le bilan theITlodynamique expn­mé par (1.60). Compte-tenu de (14) et (15), nous pouvons écrire 
. e "'\'­
û' :D .. ~ G:. E ~ (J.. E. .. ... V-S· 
.c~ .c~ ...~ .v~ .<.~ .<.~ .<.~ J,.;j 
(16) 


o '-~ . e cl.:W 

~
(J. E. . .,{;,,\\. C· = 
~ô .ca A'<'~k~ A~ dt 
où 'Ur est l'énergie de déforrnat·ion élastique . (17) 
ce qUl permet d'identifier (16) il (l.59) avec 
(18) 

La puissance l'élastique" se transforme en énergie interne élastique, tandis que la puissance "plastique" est dissipée. Le second principe de la thermo­dynamique donne alors l'inégalité 
(18) 
restriction thermodynamique sur la loi d'écoulement plastique. 


1.3 POTENTIEL PLASTIQUE 
seuil 

charge
décharg Domaine 
élastique 
IJ 

'J 
Comme dans le cas unidimensionnel, la déformation plastique reste constante Sl l'on est en évolution élastique, càd si l'on est dans le domai­ne élastique ( '1;<0) ou sur le seuil ~=O si l'on a un processus de dé­charge ( {<O ) 
si i <0 domaine élastique
't
(19) 	E.. : 0 .(.~ t81 {= 0 {<a décharge 
La déformation plastique ne varIe donc que sur le seuil et dans un processus 
de charge (-t=0, t=0). La loi d'évolution est alors habituellement défi­
nie par 
Principe du travail maximal. Dans un état de contraintes ~l~ , le taux de 
déformation plastique vérifie l'inégalité 
.,. 
'"\'­
(20) 	( <r:Ci-Cf.. ) ê .. >--0 
...~
'"* 
.. 	.. 
pour tout (J"•• acceptable, ~ 0
<t 	i ( <r.i.~ ) 
En particulier, il s'ensuit que la restriction thermodynamique (18) 
.. 
est automatiquement vérifiée (il suffit de prendre (J"... ~ =' 0 dans (20», 
Géométriquernent,en se.plaçant dans l'espace vectoriel des contrain­
tes (de dimension 6), l'inégalité (20) se traduit par l'inégalité 
(21) 

~ 0 
où [ est le point représentatif de l'état de contraintes, z:~ un point quel­conque du domaine élastique, et ­
clEf-le vecteur représentatif de l' incré­ment de déformation plastique. 
Si l'on se place en un point du seuil où le plan tangent est conti­nu, alors, en faisant varier L~, on constate que l'inégalité (21) sera vé­rifiée pour tout r." SSi le vecteur incrément de déformation plastique est 

cr· .
'1 
dirigé selon la normale extérieure à la surface seuil en L 
'f­
(22) 
E.. Â 	~ ? ~ 0
'" 
""i d(f:..~ 
La fone tian seuil est un "potentiel plastique". On tire également de (20) la propriété de convexité du domaine élastique 
En effet, SI ce domaine n'est 
pas convexe, alors on constate aisément qu'il est impossible 
.,...... de trouver un vecteur clEt vé­
rifiant (21) pour tout L* 
Ainsi, on tire du principe du travail maximal les propriétés de convexité 
(du domaine élastique) et de normalité (de à la frontière seuil). 
Par analogie avec ce que nous avons fait au § IV.I.2 pour la con­
dition de frottement, nous pouvons réécrire la loi d'écoulement plastique 
sous la forme 
( 
't
1: .. "-a Sl i(<r:..~) < 0 
'i

(23) 
si

€.'1' .. = 'À of l (<f';'a) ~ 0 
"! 
ôa:
"t 
~ ~ 0 0 = 0

1~ Â t 
et la distinction entre processus de charge et de décharge s'effectue au­tomatiquement par le jeu des deux inégalités et de l'inégalité sur? et t Cela peut sembler bien compliqué, c'est néanmoins la "bonne formulation ma­thématique" qui permet de démontrer de nombreux résultats. 
En particulier, si le critère ne dépend pas de la pression hydro­statique forme (10) alors il résulte de (23) que 
(24) = o 
Les déformations plastiques se font à volume constant (incompressibilité plastique) . 
Si  la  surface seuil  ;,::-f.3-~=-:::::  '-"":1  point anguleux  
le cas  du critère de  Tresca  - alors  le  principe du  trava~l maximal  montre  
......  
que  le  vecteur  clEt  est  dans  le  cône  des normales  
Cône  des  normales  

_--Ltf=~ 

Domaine élastique 

Le principe du travail maximal est une hypothèse qUl sera ou non vérifiée selon les matériaux. Elle est vérifiée en première approximation pour les métaux; elle n'est pas vérifiée par contre pour les sols. Ce prin­clpe permet d'engendrer une classe de modèles! les matériaux standard, qui permettent de traiter de nombreux problèmes. Les conclusions obtenues à 
partir de ce type de modèle seront plus ou mo1ns valables selon les problè­
meSa 
Dans le cas du critère de von Mises (II), on obtient 
''\'
(25) 	é '. 
-'-t 


et, par combinaison avec (15), on a 
(26) 

loi de comportement incrémentale de la plasticité. 
2, EXEMPLES DE PROBLEHES 
======================== 
2 , 1 FLEXION D'UNE POUTRE 
Considérons un arbre êlastoplastique, et soumettons le à un moment 
de flexion 'TQ croissant (voir § VII. 1, l, problèmes 5 et 6). Au départ, la 
solution élastique du § VII.I.3 est valable, et elle le reste tant que 
(27) 

Au delà de "'l,e.' une zone plastique se développe à l'extérieur de la poutre, tandis qu'au milieu subsiste une âme élastique. En supposant que, en chaque 
point, l'état de contraintes reste un état de traction simple (VII.13), nous 
sommes amenés à prendre 
-(Je, 
• ôe (:t~
(28) 
~~ =. 
} 
... cre, 

en supposant la poutre symétrique par rapport à l'axe x
3 Ce èhamp de contrain­
tes vérifie les équations d'é­
quilibre et les CL 	sur la sur­
face latérale. Il vérifie aus-	.x. 
J 
si les CL sur les extrémités 
avec 


b(~) 

et S1 on introduit la fonction ~oc~) donnant la largeur de la poutre en 
-162 ­
fonc t ion de x
2 

où la fonction 'YYl(~) est une fonction qU1 croît de ""1.. donné par (27) à ~donné par 

lorsque '§. décroît de !V..,/J-à 0, càd lorsque la zone plastique s'étend JUS­qu'à occuper tout r 
Il reste à calculer les déplacements. Dans la zone élastique, le calcul du § VII. 1.3 reste valable en remplaçant crrt./-:r par (Je! ~ , et la re­lation (VII.34) devient 
(31 ) 
x. = 
qui, combiné avec (29) , donne la relation entre le moment et la courbure. 
'I1t --~~--,,--­
'fIl.t 


Il reste à étendre cette solution au'domaine plastique, par inté­gration de (23). Cela pose davantag~ de problèmes, mais il est possible de calculer un champ de déplacements répondant au problème. Ce champ n'est ce­pendant pas unique: en général, on n'a pas unicité du champ de déplacements en plasticité. 
Finalement,. le comportement é1astop1astique d'une poutre en flexion 
est le suivant 
-comportement élastique pour tri\. <: 'YI\.,e au delà de ~: apparition d'une zone plastique, mais les déformations plastiques restent limitées ou "contenues" par le noyau élastique. 
-pour '11\.-mL' le noyau élastique disparaît, et les déformations plas­tiques n'étant plus limitées, il y a ruine de la structure. 
2.2 RESERVOIR SPHERIQUE 
Reprenons en élastoplasticité le problème du réservoir sphérique que nous avons résolu, en élasticité, au § VI.2.2 Si l'on augmente pro­gressivement la pression intérieure 1( , alors la solution élastique reste valable jusqu'à la pression 
(32) 

Au delà, une zone plastique apparaît à l'inrérieur du réservoir. Par raison 
de symétrie, cette zone sera 
limitée par une sphère 1t. .. ~ 

Dans la zone élastique, ~~'I.,.e., l'analyse du § VI.2.2 subsiste, et le ten­seur des contraintes est donné par 


« et ~ étant deux constantes à déterminer. Dans la zone plastique, a.~'!. <~ , d',,?rès la symétrie sphérique du problème, le tenseur des contraintes aura la forme suivante 

où TIC'!.) représente la partie sphérique du tenseur des contraintes, et 't:('t,) son déviateur. Ce tenseur est de révolution autour de la direction radiale, et les contraintes principales sont 

Les équations d'équilibre appliquées à (34) donnent 
(36) 	.2, "t;'C'I.) 1" G 1:('t,) _ ït'(IL) = 0 It. 
équation différentielle reliant les deux fonctions n('!..) et 'CC'\,) . 
D'autre part, dans la zone plastique on doit aVOlr 
= 



en adoptant par exemple le critère de von Mises (comme on l'a vu en VI.2.2 le critère de Tresca donnerait-le même résultat). On en tire donc 
(37) 

(le Ch01X du signe se fait par continuité avec la solution élastique du § VI.2.2). En reportant dans (36) et en intégrant, on obtient 
(38) 

Ainsi, dans le domaine plastique 
/.lC.... ~â'
(39) 0'.. 
~ <Te (~'t, +x) S..~ 0'... Il}
A.t 
(40) 0': 1 ,,(l'J-= ~ (je (~~ +·t) (j'~ = ~ <T~ ( ~IL .. ~ -1! 
où t est une constante d'intégration. Ainsi, le champ de contraintes, défi­
ni par (33) pour ~~Jt.,.Rret par (39) poura..,~,'t., dépend de trois constantes d'intégration ~ ,~ , ~ Ces trois constantes d'intégration s'obtiennent en écrivant la CL en '1. = 1r et la condition de contin~ité de (J'A ' (J'~ et <l', au travers de la surface 't,= ~ . Remarquons ici que la condition de discon-' tinuité (1.22) impose seulement la continuité de <T~ cependant, en plasti­
cité, on doit écrire qu'à la frontière élasto-plastique, l'état élastique est un état limite, ce qui revient à écrire la càntinuité de 0-", et 0"1-. Les 
trois constantes d'intégration étant ainsi déterminées, la CL en '1.= a. donne la valeur de 1" qui correspond à ~ , d'où la fonction t(~) qui croît de f", à ft lorsque ~ croît de Il. jusqu'à ..a-. La valeur limite ft de f s'ob­tient lorsque !. = ir , càd lorsque la zone élastique disparaît. Le champ de contraintes est alors donné par (39), (40) pour tout 't. . La CL en. '1. = 1r donne alors y et il vient 
(41) 

et en faisant Jt. =Il. on trouve 

(42) 

Comme en 'fl~xion, le réservoir se comporte élastiquement jusqu 1 à 
te . Au delà de +e des déformations plastiques 'apparaissent, mais ces dé­
fonnations restent contenues par la zone élastique jusqu'à ce que t attei­
gne la valeur 1 imite' 1"1. ' qui correspond à la ruine du réservoir. 
3. METHODES VARIATIONNELLES 
=========================== 
3.1 LE PROBLEME EN VITESSES 
En plasticité, il n'y a pas rel'ation biunivoque entre les contrain­tes et les déformations; il est donc tout à fait clair qu'un problème stati­que régulier, tel que nous l'avons formulé au § VI.l.1, est automatiquement mal posé. En effet, l'état de contraintes et de déformations dans un maté­
riau élasto-plastique ne dépend pas seulement de la sollicitation appliquée 
à l'instanr considéré, mais aussi de tout ce qui s'est passé auparavant. Par contre, connaissant l'état actuel de contraintes et de déformations, et con­naissant la variation de sollicitation, on peut espérer trouver la solution. Dans le cadre de l'hypothèse quasi-statique, on est donc amené à se poser un problème incrémentaI (ou problème en vitesses). 

Problème incrémentaI. Connaissaht à l'instant t le champ des contraintes champ des déplacements .l.L... (f):, k) , trouver leurs varia-et .u.l. 0<,,.1;) vérifiant -les équation d'équilibre incrémentaI es 
(43) o
+ t = 
-les c9nditions ~ux limites 
• J. 
(44 ) : 
T·..
<f~~ m~ 1 Sf 
cl.
(45) ..u, . 
..u, ... 15.... = .. 
-la loi de comportement incrémentale 

'1'
avec ~.. donné par la loi d'écoulement plastique.
La 
Si nouS acceptons le principe du travail maximal, c'est encore 
un problème bien posé. 
ThéOrème cl 'unicité. En acceptant le principe du travail maximal, le problème [posé plus haut admet une solution unique en rr·· 
L~ 
Dem. Supposons en effet qu'il existe deux solutions (0-.•. ,i.J..~ ) et (ir.1, ,;"l,).
~à ~ ~~ ~ 
Leur différence 
• 0 • 0 • .t • "'\
." ••
(47) cr .. = 0".. -0".. .u. . .u.. -M.. 
... ... ...
"! "i "'i 
vérifie les équations 
(48 ) , 
(Par contre, elle ne vérifie pas nécessairement la loi de comportement (46) 

• 0 • 0 
qui est non linéaire). En appliq'twnt à <r.. et.u.. le lemme fondamental du "t '" 
§ IX.I. 1 , on obtient alors 
(49) ~ 0 Mais 
. t . tl,· .,
+ <r.. é.. .. -(L . 
.L~ "'i "'~ 
En tout point x, on connait le tenseur des contraintes, on connait donc la 
zone élas tique CL... où f((fL!!) <0 et la zone plas tique n. t où f(<f.;,~) = 0 . 
Dans la zone élastique, les taux de déformations plastiques sont identique­
ment nuls. Dans la zone plastique, on vérifie directement que, d'après le 
principe du travail maximal: 
i\
t-.; • < E. =(). 
(J 
charge décharge 
(50) = 0 ~ 0 
Ainsi, on peut écrire à partir de (49),(5U) 
(51 ) 
~
( 
'.~ .t, . 
= (f é .. .. CT., o
~~ Ni ... ~ Ai
.Qt 
or, est défini positif. On obtient donc 
(52) 

(53) 
cqfd 


Par contre, on ne sait pas démontrer l'unicité des déformations. 
Ces problèmes sont des problèmes mathématiques difficiles et encore mal con­
nus. 
On sait également démontrer des théorèmes variationnels analogues à ceux des § IX J.2 et IX.J.3, qui donnent naissance à des méthodes numéri­ques de solution du problème incrémentai. La résolution d'un problème élas­to-plastique quasi-statique se fait donc pas à pas, par résolution d'une suite de problèmes en vitesses. Mais il est important de remarquer que pour toutes ces questions, la formulation de la loi d'écoulement plastique par le principe du travail maximal joue un rôle essentiel. 
3.2 INTRODUCTION A L'ANALYSE LIMITE 
Nous considérons un problème où le chargement dépend d'un seul paramètre 
S.... = rf ou 

(54)  T.el. ...  =  ? T/o  
Pour les problèmes envisagés  au  §  2,  on  a  '.l.:'hl  pour  la  poutre  en  flexion  
et ').: t  pour le réservoir sphérique.  
Si  l'on fait croître le chargement,  on  obtient  toujours  le même  
' 

comportement. Jusqu'à ~ = ~e la solution élastique convient. Au delà 
apparaît une zone plastique qui progresse au fur et à mesure que ~ aug­mente, et jusqu'à ce que pour'). = ?,t on ait ruine de la structure par dé­
formations plastiques illimitées. 
La charge correspondant à 'il =?,t est appelée "charge limite". C'est la charge maximale supportable et, d'un point de vue pratique, c'est le résultat le plus intéressant et le plus significatif d'un calcul en plas­ticité. On a donc cherché à développer des méthodes permettant de calculer 
directement cette charge limite, sans réso~urion cc~~lète du problème élas­
to-plastique: c'est le domaine de l'analyse limite. 
La théorie s'appuie sur deux théorèmes fondamentaux. 
Défini tion 1. Un champ de contraintes â-.. sera un champ licite pour un 
~~~~~~. ~~ 
chargement (t,T;.,tL) s'il est statiquement admissible et si en tout point 
[ 
il vérifie le critère de plasticité. 
~ 
Définition 2. Un champ de vitesses ~. sera un champ cinématiquement et 
~ 
plastiquement admissible (CCPA) s'il est cinématiquement admissible et si 
0' 
en  chaque point  il existe  un  tenseur  de  contraintes  tel que f... puis­ 
.~  
se  être la déformation plastique associée.  

-168 ­
Dans le cas des métaux par exemple, cette dernière condition re­vient simplement à imposer la condition 
':' 
(55) e .. • o 
..... 
On d~finit alors la fonction de dilsipation 
(56) 

-
N
fonction définie sans ambiguïté, bien qu'il puisse exister plusieurs a:. 
"'a
compatibles avec Pour le critère de von Mises par exemple 

• -i' '-i'
(57) 
E. .' c. .. 
..t ...t 

On peut alors démontrer les deux thêorèmes suivants 
Théorème statique. S'il existe un champ de contraintes licite pour le char­[ gement (J. ,TeL), alors la structure peut supporter ce chargement.
fA. .. 
Théorème cinématique. S'il existe un CCPA tel que 
(58) 

.. (l' TeL 1. cLS )
~s k At 
f 

alors la structure ne peut pas supporter le chargement (p.. ,reL). 
~.. "" 
Le premier théor~me permet de construire des bornes inférieures de la charge limite. En effet, soit â-.~ un CSA pour le chargement (1~ ,T.d.o).
""'6 6"..... .... 
Alors, le champ (').~; ) sera CSA pour le chargement (') ( ,?T"-d.·). En choisissant pour ~ la plus grande valeur conduisant à un champ licite 

ft. 
alors, le thêorème statique permet d'affirmer que ~ est une borne inférieu­re de la charge limite. 
De la même manière, le second théorème permet d'obtenir des bornes supêrieures de la charge limite. Soit en effet 1t~ un CCPA, alors la struc­ture ne pourra pas supporter les chargements vérifiant' (58). La quantité 

est donc une borne supérieure de ~i . On en déduit un encadrement 
(61) 1). , '>-! ~ '). 
qui permet d'approcher relativement simplement la charge limite. On notera 
-169 ­
l'analogie de cette méthode avec celle présentée en élasticit~. 
L'analyse limite est utilis'e 
-en mfcanique des structures, car la charge limite caractérise bien la capacitf de résistance d'une structure -bien mieux en tout cas que la charge élastique-~e -. La tendance actuelle de la règlementation en matiè­re de calcul d'ouvrage consiste à substituer un calcul plastique au calcul élastique traditionnel. 
-pour les problèmes ~mise en forme des métaux (laminage, fiiage, etc.), car elle permet d'évaluer les efforts nécessaires. 
en mécanique des sols, pour calculer la capacité de résistance d'un ouvrage. Il convient alors d'agir avec précaution, car les th'orèmes que nous avons énonc's supposent le principe du travail maximal, principe non vérifié pour les 801s. 
-171 -
Chapi tre XI 
THERMOELASTICITE LINEAIRE 
Jusqu'à présent, nous n'avons pas tenu compte de la variable température. Dans de nombreux problèmes, cependant, il est nécessaire de la prendre en compte (problèmes de contraintes thermiques, par exemple). 
1. LOIS DE COMPORTEMENT 
======================= 
1.1 LA THEORIE THERMOELASTIQUE 
Un matériau thermoélastique est un matériau dans lequel la seule source de dissipa tion es t la conduction thermique. Nous considéroD", une théorie de petites perturbations autour d'un état de référence à contrain­tes et déformations nulles et à la température de référence 9• Nous
0 
poserons 
(1) 

avec petit. D'autre part, l'entropie n'étant définie qu'à une constante près, nous la prendrons nulle dans cet état de référence. Finalement, f .. ~, <t.ë.t ' li et ~ seront des variables de perturbation, et nous pourrons né­gliger les termes d'ordre supérieur par rapport à ces variables. 
Dans ces conditions, l'équation de conservation de l'énergie (1.47) et l'inégalité de Clausias-Duhem (1.57) s'écrivent 
(2)  Po  cie cU  - <LE ..Ir.. _ ...~ Lt d,I;  + It.  -9"t,t  
(3)  -p.(;  -(1  ~)  +  0:. "'1  ci€,i.a tU  ..f e.  ~.,  e.,-"  >.,  0  
En élasticité,  e.  dépendait de  f."â  . En thermoélasticité,  .(..  dé­ 
pendra de  f.....~  et de  ,."'\  

(4) 


L'inégalité de Clausius-Duhem (3) donne alors 

.• 172 ­
et, pour que la seule source de dissipation soit la conduction.thermique, on doit avoir 

(J,. =
(6) 
9 = 
"'t 

(7) 


Comme dans le cas élastique, nous pouvons faire un développement en série de .e. 
P. e (E~,"f}) = Po.e. ... a...~ ê..i-~ 1-Po a.. "l
(8) 
f 	... i A.i.~ Vtt E.i.à ê~R. i-!-'1 &.i.â "1 ... 


~.i.â t.i.â-+ NT\, "l. A...àU: €it + Â.i.t "1 
Or, dans la configuration de référence, on a E,· =1'11= 0, (\': ·.0
"i l -4* 
On doit donc avoir a.:: e <1;•• =0 et on peut aussi 'prendre
• 0 
"a

obtient alors 
p'.e .. Po~"l + Po .ë(E.i.i ,"l)
( 10) 
= ... -R,.., e: III i-: l'II\. trI."

P.'I" -1 	.9, A.i.âu' E..i.â EU "'t "a J, a--~ë 
(J 1) Ci:. = e. ~=A··u EU 1-Â"1~ , p'9 .. p"_= l.:e.. i-1'tII."l 
...~ 	o· • 'd"l ""t A.t
• ae'à "'a 
Compte-tenu de (6), l'équation (2) devient, après linéarisation 
(12) 	
o El cl] 

r. 
• clt 



Pour compléter la théorie, il faut écrire une loi de conductio~ thermique, donnant le flux de chaleur en fonction du gradient de température. En.théorie linéaire, on prend la loi de Fourier 
(13) 


avec un tenseür de conduction K~t symétrique cette symétrie est expéri­mentalement bien vérifiée, et théoriquement bien fondée par les relations d'Onssger; elle n'est toutefois pas indispensable, on peut imaginer un ten­
seur de conduction non symétrique -et défini positif -d'après le second principe (7) ­
Le formalisme (10),(11) définit les contraintes et la température en fonction des déformations et de l'entropie, ceci étant lié au choix de l'énergie interne comme potentiel thermodynamique. Une transformation de Le­gendre sur ~(t...tl"l) permettra d'écrire d'autres relations. Par exemple, 
-173 ­
on peut introduire une "cnthal pie 1 ibre" tr·· t: ..
%(cr .. , 9) 	+ p. S'Y) -P. e.
{ 
P. 	" 
"'II .~ "'a 
(14) 
E.. ~
'7 " ~ 	" P.
...~
~9 	ë1 cr· . 
.~ 
A 	..(
(15) Po '3' " -r:! \f.. Û.i.i (lu" + 1. .i,.~ û. ë + rn, e-
l. "'â $ E.. " r.. ~H, ûftt + 1.. 9 
.~ 	..,~
(16) 
{ 1.. 	û·. + rn, ~
F. '1. " .~ "'a­
où les coefficients de la forme quadratique Po '3' peuvent s'exprimer à partir de ceux de la forme p. e. par les relations 
$ (cM. ! ..
"" -	""â A "aU 

D·
(17) 
.,L 
~! -"" A ""tU J...f.& 
Il + !YI. Il Il t-ft.
r.i.~ ~L = .;.if. t 	-4t ""',., f~ -­
"if' 
De même, on pourra exprimer E~t et 9 en fonction de l'enthalpie t(û;.~, "l) , e t (f.i.~ et 1. en fonc tion de l'énergie libre 't' (t:.i,.~, ë). 
1.2 THERMOELASTICITE CLASSIQUE 
Dans le cas isotrope, les différents tenseurs prennent des formes simples. Un tenseur du 4ème ordre prend la forme (V.22) et dépend donc de 2 coefficients. Un tenseur du second ordre est sphérique et fait donc interve­nir un coefficient. La théorie fera donc intervenir 5 coefficients scalaires: 4 coefficients pour le potentiel thermodynamique (10) ou (15), et 1 pour le tenseur de conclue tion l{L~' En effet, (13) devient 
(18) 	q. _ -k '" 

où t est la conductivité thermique. 
Nous convenons désormais d'omettre les barres sur 12-et e . En par­ticulier, e désignera désormais non pas la température absolue, malS sa va­riation par rapport à la température de référence. 
Si QI est la coefficient de dilatation linéaire du matériau, alors en l'absence de contraintes, une variation de température 9 entraîne une di­latation thermique 
(19) 

D'autre part, à la température de référence 9. ' le matériau se comporte com­
me un matériau élastique isotrope, aVeC un module d'Young E et un coeffi­cient de Poisson Y, mesurés en conditions isothermes •.Nous pouvons donc 
écrire 
(20) E .. v 

+ 0/. El b.. 
"'t 
"'~
E E 
relation que l'on peut inverser en 

-~ 1< 0/. e $..
(21 ) 
"'t 
où ~I'(,,?,~ +~~ est le module de rigidité à la compression isotherme, et 
~ et ~ sont les coefficients de Lamé isothermes donnés par (V.33) à par­
tir de E et Y 
Quant au coefficient de 9" dans ·le potentiel thermodynamique, on peut le relier à lachale'ur spécifique à contraintes ou déformations cons­tantes 
(22) .G = oC = e (~:J,f2,) , "'.. 
f • ae!l. . €"',t = Cl1 

ce qui nous permet d'écrire, à partir de (16), 
(23) 0( cr.. + Po"'.. e
Po "1 = 
... '" 
9. ou bien, en fonction de déformations, 
(24) f.") 

et l'on obtient la relation 
(25) 

relation entre les chaleurs spécifiques à déformations constantes et con­traintes constantes. 
Ainsi, la théorie classique de la thermoélasticité fait intervenir 5 coefficients: deux coefficients élastiques isothermes, le coefficient de dilatation ot. , la chal(>IH" i)f!l' i fiqU!·~ :iéformations constantes .(, et la conductibilité thermique ~ 
-175 ­
2. PROBLEMES DE THERMOELASTICITE 
================================ 
2.1 PROBLEMES AUX LIMITES 
Les équations de la thermoélasticité sont les équations du mouve­ment et l'équation de l'énergie (12) 
Ô~ .u.. 4
(26) 	(j...
P. = + -t 
ô.t~ "'a'3 
(27) 	'0 'Yi = IL
Po e. 
ôk -'1i,t 
complétées par les lois de comportement, (18),(21) et (24) par exemple, 
des CI 

v' (f.(.)
[ 
, 


.... 
(28) 
e (t:c,o) = e'(-x,) 
et des CL, qui sont les mêmes qu'en élasticité pour les variables mécaniques 
et qui, pour les variables thermiques, donnent, soit la température, soit le flux de chaleur. Nous prendrons par exemple des CL mixtes 
= 4.cl. 	Ta 
....
(29) 	4.;./S.... (j...~ 1'I1.i-15, = .. cL =
e /S8 eJ. 	91.N1... /s = 9 
~ 
avec S =s.... + s, = 59" 5'1 problème est bien posé, et on peut démontrer un théorème d'unici~é analogu, à celui du § VI.I.2 . 
Compte-tenu des lois de comportement (21),(24) et (18), les équa­tions (26) et (27) donnent 
i'.A.(..
(30) p. --< = (Â .. e-) ..Qi,l.i1l. .. e-.Q.À.,U -~\(o( e, ... + -t
'O.t~ 
J.L.
(31) .c Oe = Iv .. -k e.. -3\(0/ e. t ...
P-	,..<.J"
~,t 'il.\(.... 'Ok Dans de nombreux cas, on peut raisonnablement négliger le dernier terme de (31). On arrive ainsi à la "thermoélasticité découplée". En effet, le déplacement disparaît de l'équation (31), on peut donc déjà résoudre le problème thermique 
oe 
'l. .. t t,e
Po'" = 
ôk 
(32) a
e	= 'Ici.
915a = 9;, 1'11." 1~ (j(~, 0) '" eO (le) 
problème classique pour l'équation de la chaleur et qUl permet de calculer la répartition de température indépendamment des déformations. Une fois con­nu le champ de température, on peut revenir au problème mécanique 

+ { -31<0( ~e 
, <-:0 ..u.cl = rd.
(33) 
cr
-u-I • A-â N\.a I<­
S.... Sf 
0
.Q~ (~, 0) = -U-(.OC.) o.u.~ («,0) = V~ 0 (.le)
'" ok 
qui est un problème d'élasticité classique, le couplage thermoélastique se 
manifestant seulement par une modification des données { .. et T4,A. 
Nous avons formulé ici le problème dynamique. mais l'on peut aussi 
envisager le problème statique: les dérivées par rapport au temps et les con­ditions initiales disparaissent. On peut alors démontrer des principes varia­
tionnels analogues à ceux du cas élastique. On peut également envisager un 
problème mécanique quasi-statique avec un problème thermique dynamique. 2 • 2 UN EXEHPLE 
A titre d'exemple, nous allons calculer les contraintes thermiques engendrées par l'échauffement d'une cavité sphérique de rayon ~ dans un mas­
sif infini. Nous considérons donc le problème défini par 
(34) IL
t,= 0 = 0 
(35) '1. ..... CD Il -. 0 , cr.. ~ 0 
"'~ 
(36) 't,=CL Il = cr.. N\,. 0
e. 
~3 ~ 
Il s'agit d'un problème statique et l'équation thermique (31) donne 
(37) 
= o 

(38) 


D'après la symétrie du problème, e dépend uniquement de 'L e = e(IL). On a 
= 

et l'équation (37) s'intègre en 
A
(39) e= + B 
où A et ~ sont deux constantes d'intégration que nous déterminons à partir des CL pour IL= a. et IL_ CIO • On obtient 
(40) 
e = 
et nous avons résolu le problème thermique qui, en statique, est toujours dé­
couplé du problème mécanique. 
Pour résoudre le problème mécanique, nous partons de l'équation du 
... 
mouvement (30) qui, en statique, donne, en supposant ~= 0, 
= 0 
ou bien, voir (V.2S), 
D'après la symétrie du problème, le déplacement est radial 
(42) 
et, comme on l'a vu au § V.2.2, Ji.el;ii: :0. L'équation s'intègre alors pour 
donner 
(43) e '!l A 
où A est une constante d'intégration. Compte-tenu de~(V.47) et ce (40), on obtient alors pour I}(I!,) l'équation différentielle suivante 
(44) 3A 
qui s'intègre pour donner 
(45) A 
Il reste à déterminer les deux .:onsta.:ltes :::t:':-_:f 6:";?:ion A et 5 en écrivant les conditions aux limites sur les contraintes. Pour cela, nouS po­sons 
(46) 
C. = et nous calculons 

(47) 
f. .. = (A + ~ + Co) S.. 


"'3 IL~ 't. ~J 
et les déformations principales sont 
(48) 
€~ = ê2, = la valeur propre E étant associée à la direction radiale. On ~eut (lflsuite 
~ 
calculer les contraintes par (21), et la condition à l'infini donne directe­ment A= 0 . On calcule ensuite cr?, 

et  en  ê<:rivant  que  0'3  est  nul  pour  '1. = (t  ,  on  obtient la valeur de  B  
(50)  Pl,.  -C (t~o  
Soit  finalement  
(51 )  ~("') cr, c:. =Cf~  ~KCl! 9. 0.• 1 ('h.tt,L) '" '~l<o(9.-('h~t'-) • -~!:,-kd ~ O ... l,t,L)  0. sr. Cl. li.  ",J. _ o.J. "," /1," -0.1, "," Jt-'-o.~ ","  

-179 .. Annexe A NOTATIONS TENSORIELLES 
L'objectif de cette annexe est de familiariser le lecteur avec les notations tensorielles. Les résultats que nous démontrerons sont tout à fait classiques, et il convient de les considérer comme des exercices pour l'ap­prentissage des manipulations indicielles. 

1.1 NOTATIONS INDICIELLES 
Nous nous plaçons ~n. l'espace l euclidien i trois dimensions. -. .. .. -:
50lt• ~4' ~&'~' une base orthonoraée. Un vecteur , est alors représenté 
par se. composantes "4' Va ,V. 

en utilisant la convention de sommation: chaque fois que dans une expression un indice est répété, il convient de faire varier cet indice de J à 3 et de faire la SOUDe. Dans l'expression (J), l'indice i e st un "indice muet": on 
V .. V. ..
aurait aussi bien pu écrire a.e... ou le,. 
Soit A une application linéaire, alors dans la base ê, ' l", ' ê3 ' cecee application est représentée par une ~~~Ti~ê :43 
(2) A 
-

et,  si  ..W=AV  ,  les composantes de  W  sont données  par  
W1 = W~'" W3 =  A.... Vi Ali Vi ~V,  + A-t~ V" .. Au Va, + Au V~  .. T ..  A~~ V3 ~~ V3 A33 V3  
que  nous  pouvons condenser  en  

(3) 


F indice j est un indice muet: on aurai t aussi bien pu écrire A.4~ Vi . L' in­
-1HO ­
dice i est un indice libre. Dans une égalité, on doit avoir pour chaque terme les mêmes indices libres. 
Nous 	introduisons les symboles de Kronecker 
(4 ) 	S si "'''1 
~~ ={ : ~ti
si 
En particulier, l'application identité 11 est représentée par la matrice de composantes b. 
~a 
0 0 
-11 	: Â 0
S.t~ 	s.1,2. S~:.
4"'­
0 ..j
dM O~t S,~. 	1
r" '" '.. l [: 

... 	.... ... 
Si la base 	est orthonormée, on a
~1 , 	e~ , R,3 
... -+

(5) 	S .. 
R,J.' .e,i = 
""il' 
et le produit scalaire de deux vecteurs est 
... -+ 
... ..
V. W V.1,. . w. ê. V. W. e ..~.
'" 	= ... ...
(6) 	"" "" ~ t a ~{ : V. 'vJ. S.. = 
'IL vJ:...
J..a
"" i 
De même, la composition de deux applications linéaires se traduît par le pro­
duit 	de leurs matrices représentatives, càd en notations indicielles 

1 • 2 CHANGEHENT DE REPERE 
Soit "l. une base orthonormée et ~~ une autre base orthonbrrnée . 
.... 
Soit Q.. la matrice de passage 
~
... ..., 	.... 
+
.e.~ = Q~~ ê. -t Q~t e.l, Q~, ê.! 
-", 
: 	1­
e", Q:H t, + Q2..t ê,t Q.t~ê3 
.... , .... 
.e~ 	= Q~~ .e.< + Qôl, ê~ .. Q~"l. ê~ 
ou en notations indicielles 
(8) 

Les deux bases étant orthonormées, on doit aVOlr 

= Q.<.~ Qie t~. li 

qui montre que la matrice inverse de Q.. est la matrice transDosée 
... ~ 
(9) 

-181 ­En particulier, on tire de (8) la'relation inverse 
( 10) 

qui s'obtient par le calcul suivant: on multiplie (8) par Q.L~ et 0'. utilise 
(9) 



càd  (10)  à  un  changement  d'indices près.  
1.3 VECTEURS  
Soit V un vecteur, V. ses composantes dans la base ... celles dans la base .... (II ) V " VI ...1 . .e. ... .. Pour obtenir 1es lois de transformation permettant de passer de nous utilisons (10)  -:J.v ... V. à..  et V~ ...  V.I ...  

v . v. "i 

... N 

et par identification avec (II), il vient 
( 12) 

, V· " Q .. V:
a .va'" 
formule de transformation des composantes d'un vecteur. 
On appelle "invariant" une fonction des composantes d'un ou plu­sieurs vecteurs indépendante du repère choisi. Par exemple, l'invariant pro­duit scalaire est défini par 
......... 

( 13) V. W = 

C'est un invariant car, d'après (12) et (9), 

1.4 APPLICATIONS LINEAIRES 
.... 
Soit lA une application linéaire de E dans E . Dans la base R. ... 
-1 
elle est représentée par une matrice A .. et dans la base e ~ par une autre 
"'"3 
matrice AI.. • Pour obtenir les lois de transformation, nous partons de 0) 
...~ 
et (12) 
-182 ­
'vJ. " A.. Vi..
.. "~. 
, 
W:,. : Q.i.f. W,,-:; 

et par identification il vient 
(14 ) c 

, A.. 
"'"i En particulier, on a vu que les symboles de Kronecker b .. étaient

A.t les composantes de la matrice associée à l'application identité. Par applica­
tion de (14). 

et l'application identité est représentée dans toute base par la même matrice. 
1.5 FORMES BILINEAlRES 
Soit A une forme bilinéaire sur E , càd une application bilinéai­re Ex E -1< . Dans une base :ê., , elle est représentée par une matrice À.. 
-A.a 
telle que 
(16) A( --V, W) = A.. V. W·
"'$ J. i 
Pour obtenir la loi de transformation de A.. nous partons de (16) et (12)
A.t 
(17) A(V,W) = A.1/. vJ· = A'.. V~ W:
"'a J. t ..a.... ~ 
A" 
= ...~ QVt.. v~ Q!~ \lit 
d'où par identification 

càd la même loi de transformation que pour une application liné~ire. Ceci est évidemment dû au fait que nous n'envisageons que des repères orthonormés. En particulier, la forme bilinéaire représentée dans toute base par les symboles de Kronecker est le produit scalaire. 
1.6 TENSEURS DU SECOND ORDRE 
Il résulte de ce qui précède que l'on peut identifier application linéaire et forme bilinéaire sur E • Nous appellerons "tenseur du 2nd ordre" cette entité mathématique, généralisation de la notion de vecteur. Algébrique­ment, on peut la définir en introduisant une opération bilinéaire "produit 
-183 ­
tensoriel", notée @ , et le tenseur lA sera défini à partir de ses compo­
santes A.. par 
.~ 
(I9) 
lA 

formule qU1 généralise (1). On obtient alors directement la loi de transfor­mation (14) ou (18) à partir de (8) ou (10) 
Il = A..
.. 
= ~ = 
... ~ 
lA A.. e. ® e,. 
~â "" a 
= 

d'où par identification 
(20) A'.. 

A.. = 
.~
'<'a 

et un tenseur du second ordre pourra représenter, suivant les circonstances, 
une application linéaire (exemp'le: le tenseur des contraintes) ou une forme 
bilinéaire (exemple: le tenseur des déformations). 
Un tenseur sera dit symétrique si 
(21) 
antisymétrique si 
(22) 
et isotrope SI 
(23) 
= 

Un tenseur quelconque peut toujours être décomposé en une partie symétrique 
AS et une partie antisymétrique IAA 
IAA AS
t4 = + fA'" A $ AA + 
.c~ .c~ 
"'~ 
A s 
.j 
(24) A. = 1(AA.~ -Ai~) A. = -( A. 1" A· )
-cJ ...~ i.e
$, ""â 
1.7 TENSEURS n'ORDRE SUPERIEUR 
Considérons, par exemple, une application linéaire de l'espace des 
tenseurs d'ordre deux dans lui-même (exemple: le tenseur d'élasticité). Dans une base , il est représenté par une quantité Â'<'il\. t à 4 indices 
(25) lA = A[Œ3) A.. = 
'"â 

La loi ri. transformation des fI li s'obtient directement à partir df' (18) 
-'-1 "'" 

~
A ; 1\ A t\~~H I:,'H
-'-i ... ~U BH .~ 
A Q A Q Q 

'j ..,..-. Q~", ,,"" ."'" 1\ """"f<t e,H
~'" 
Q Q Q Q 1\ ~ 

.~ 
i'" 'l''' ~ L "",,,,,H 1"9 d'où la Icd dE-transformation 
Q
Il;.~~t Q ~'" Q~f Q!9
(26) ~ " ""
{ , 
~
!\ Q 
Qi"'-Q~t Qt~
Iv,...
"""""f~ 

forme analogue à (20) ou (1). Nous introduisons donc le "tenseur du 4èrne or­
(27) = 

qui, suivant les circonstances~ sera une application linéaire" de l'espace des 
tenseurs du second ordre dans lui-même, une forme bilinéaire sur CE. rneme €f:­pace~ une forme quadrilinéaire sur l'espace des vecteurs, etc ... 
1 . 8 INVARIANTS 
On appelle "invariant" du tenseur du second ordre fA l'ne fonction des A·· indépendante du repère choisi. Par exemple" 1es fonctions suivantes
Li 
Ah,J'.\ = A
L. 
1
,.lA. A= A. A fiï, 1A'n. = ""i i''''
(28) 
Il A~.\. = kA 
....~ .<oj 

~-effet. on a par exemple 
A'. 
.<-.<-Q <.~ Q"R, AU = 8~e AU 
= Q.~ Q., Ao, Q. Q . 


.<oR ~~ .." .... "" â-"" 
= Sf\.crn, 3i m. Au A""""" = 

On définit de la même manière les invariants de plusieurs tenseurs, ou d'un tenseur et de plusieurs vecteurs. Par exemple, les quantités suivantes 
A(Ae4:) = A.. 
e,i~ c~...
""a 
(29) Il:e :: A. e,. 
.. ~ kt 
....
\ V. Il W -= v. A. IN
• 
.... i * 
sont des invariants. 

En particulier, on rt:rnarque qut-A: B ul:fini l un pruduit 5c.11atrt; 
------"-"-­sur l'espace des tenseurs du second ordre. Pour ce pruduit scalaire, l'espa­ce dt.'s tL~nseurs symétriques est orthogonal à l'espacé déS tt-'nsclIrs <'"!ntisym~­triques. En effet, si /Ii est un tenseur symétrique et!2.. un tt:"ns€ur antisy­métrique, 
(30) 
On a alors 


(31 ) /A:Q = 
la première transformation résultant d'un échange des indice!> wuetfo i el J, la seconde de (30). De même~ on peut décomposer un tenst:!ur symétriq112 en cJar­tie sphérique et déviateur 
(32) 
A

1)" = 0 
~~ 
et, à nouveau, c'est une décomposition en sous-espaces orthogonAuJo... 
2. PER}illTATIONS ET DETERMINANTS 
=============================== 
2.1 LES SYMBOLES DE PERHUTATIOK 
Nous introduisons les symboles de permutation S1 .t.. J 1-,fit permutation paire de 1,2,3 
(33) 

si .c, i,~ permutation impaire d~ 1,2,3 
51 deux indices sont répétés On peut relier ces symboles aux produits mixtes des vrcteurs d" has, 
(34) 
On démontre alors sans difficulté (c'est une simple question d~ patience) les 



relations suivantes 
s. 
Jo""­
"'''''
['t S. 
E. 'i ( ,e"",m, ~ cltt 8:~ S. S·1....
"'1 !""­
Su 0\"", Sft"" 
(35) 
E. .~ E. '" 8. Sil, -s· ô'­
;~ "''''''' fT\. 1""" "" ~"" '"'­
E" " = ~ S~
E. "v" 
... t l'n\.
"'~ 
é.;..~~ €;. i ~ = " 
-186 ­
2.2 DETER}1lNANT D'UNE MATRICE On peut démontrer que dans un changement de repère 
(36) 
= 

suivant que le changement de base est direct ou non. Si nous nous limitons aux repères orthonormés directs, alors 
(37) 

= 

et les E. .. 1l. sont les composantes d'un tenseur du 3ème ordre, qui représente,
""!" 
par exemple, la forme trilinéaire produit mixte 
-. ........ )

(38) ( U, 1/, W "" € .• o. U. V. \ND
""â'" ... ~ '" Les symboles de permutation permettent le calcul du déterminant 
d'une matrice par 
(39) 

= 

A."""", 

ou par (35) 

En particulier, d'après (36), dtb A est un invariant du tenseur du second ordre lA. • On peut également donner pour l'inverse d'une matrice (ou d'un 
tenseur) l1expression suivante 
(4 J) 

, 'O .. = 
L 
-a-

En effet, on a, par (35) et (39), 


càd 

2.3 POLYNOME CARACTERISTIQUE 
Les valeurs propres d'un tenseur du second ordre sont obtenues par résolution de l'équation caractéristique 
(42) -Pp.) ., 

soit, par développement de (40), 


(43) 'PO.) " 
avec 

sont appelés invariants fondamentaux du tenseur lA . On peut 
montrer 
que tout invariant de A peut s'exprimer à partir de ceS trois inva­
riants fondamentaux. Cela résulte, en particulier, de l'équation de Cayley -Hamil ton 
(45) P(A) 

qui permet d'exprimer 14~ , et par récurrence IN/, 1A5 , etc ... en fonction de lA et IA~ . 
2.4 	ADJOINT D'UN TENSEUR ANTISY}ŒTRIQUE Soit il un tenseur antisymétrique. Sa matrice représentative est 
(46) 


Nous pouvons lui associer le vecteur 
(47) 

• 

Le vecteur ~ est le vecteur adjoint du tenseur antisymétrique !L . Cette relation est exprimée par la relation 
{ 
Q. .. 
= E"'à~ 	"'ft
(48) 	""'a 
../
w.
... " 	e..~~ Q.i~
2: 
... 
~ 
En particulier, le vecteur 1I;}:o n IX. est donné par 

3. CALCUL VECTORIEL ET ANALYSl VECTOIlIELLE 
========================================== 
3.1 CALCUL VECTORIEL 
On a vu que les notations indicielles permettaient d'exprimer sim­
plement le produit scalaire par (13) et le produit mixte par (jS). Le produit 
vectoriel s'exprime aussi simplement par 
..... 
( 50) ,c = 

, 

comme on peut s'en convaIncre par exemple en écrivant 

ou bien par un calcul direct. 
En particulier, si sn. est un tenseur antisymétrique et son vectéur adjoint, alors, Sl. g= ~?t, on a par (48) 
'1h = Q -'-3 1.(,i 
(51) 
fii, = 

On peut également démontrer facilement les identités du calcul v~~tori(~~ par exemple, 
(52) ( ~ -) ­
Q, "J,. ",c. 


~ .Jr -.-) -+
En effet, si ~= (Q,,,J,. 1\ /:, , on a d'après (50) et (35) 
Il., ~ E . Il, t J:
IX.. <-= € ..",e· J.."" ,c~ Eir.. .... ""-.,
!TV
"'~ i rm."" "" ~ "'" "" 
Cl.
= 
( 6fl, "'" &,;."", Sl"" S'-"'" ) ""'-t-"" /:,~ 
~ 0...
Q.&.c~ J,.,;. .&,., .ck 
"­
En guise d'exercice, on pourra démon t rer de manière analogue (53). 
3.2 ANALYSE VECTORIELLE 
Nous considérons maintenant des fonctions à valeurs scalaires, vec­torielles ou tensorielles, définies sur un ouvert (l . Nous noterons d'une virgule la dérivée partielle par rapport à x. 
1 
'à
(54) ,k 
= 
Ô I.C.. 
N 
Par exemple, S1 i est une fonction scalaire, nous définissons son 
-1 K9 
gradient 
(55) vecteur 
et son laplacien 
(56) scalaire 
.... 
Si V est une fonction vectorielle, nous dêfinissons sa divergence 
(57) scalaire
= 
son rotationnel 
(58) 
son gré1dient 
(59) tenseur son laplacien 
(60) vecteur 
et l'on a, en Darticulier 
(61 ) 
= 
-> ....\-.~­
En e f f et, si ,(A, =Jt,O'V llO'V '\T , on a, par (58) et (35), 
)..l. = ( I\T ).
'-' €.'<'à'tv t ft 'Th, "" 
"",""-'i 
= I\T nT.. 11T.EL~ t. ~tm."" "'"l ""'i '" Iv J .A,. 3 "'j~ 
càd le second membre de (61), par (55),(57) et (60). 
Enfin, si lA est un tenseur, on définit sa djverg~nre 
vec teuT 
On démontre alors facilement un certain nombre de relations utiles, par exemple, 
(63) 
d-W({ii:) = 

(64) 
cI.W-(â" t) = 


et ainsi de suite. 
-190 ­
3.3 TRANSFORMATIONS D'INTEGRALES 

Soit  .n..  un  domaine de l'espace,  'dQ.  sa  fron­ 
tière,  et  .... ~ sa  normale  extérieure.  
'Ost.  On démontre  en  mathématiques  le théorème  sui­ 
vant  

Théorème. Si <f est une fonction continue e·t à dérivée continue dans rL , 
et si (jSl. admet un plan tangent continu par morceaux, alors on a 
(65) 

= 

et toutes les formules de transformation d'intégrales utilisées dans ce cours peuvent se déduire de ce théor,ème. 
4. COORDONNEES CURVILIGNES 
========================== 
Les formules du § 3.2 et les formules que nous avons écrites dans 
ce cours sont valables en coordonnées cartésiennes. En coordonnées curvili­
~, il faut prendre quelques précautions, car les vecteurs de base chan­gent. Cela ne modifie en rien les relations entre tenseurs (par exemple, la 
loi de comportement), mais cela intervient chaque fois que l'on a des déri­vations (par exemple, dans les équations d'équilibre ou dans la définition des déformations à partir des déplacements). Nous allons donner un formulaire pour les coordonnées sphériques et cylindriques 
-191 ­
4.1 COORDONNEES CYLINDRIQUES 

e 
Gradient et Laplacien d'une fonction scalaire 
...
~.t + -.4 ~ ~6 + ~-.e.­
~{ '" ah, 't. IL àe 9~ r 
..r -4
d{ = ~(~ ~) ... ~ ... ~ 
"-'h 'd 't, 1f.,J, (le" ô~" 
Définition des déformations 

Equations d'équilibre 
..j
Ô <T",,,, (l (1""'9 Cl <r~'t (fIlA. -<il//) 0
+ -+ + + 
'" 
'ô 'v 't. oS 'L i"
°îr 
'q <iJl.9 ..r ô<Jee Ô <r9~ ~<ï"'9 ,.. 0
+ + + + 1B
ô1\, 'L as Ô'lf IL 
o<r",'t ..r a(Jill( d <r..,lr <f",y 0
+ + + + = 
IL '(}e 't. l'b'
Ô'" '()~ 
-192 ­
4.2 COORDONNES SPHERIQUES 
... 
~
Repère local: -
CIt. , et:; , 
'"'f 
• 
(l"'t9 cr = (f'te (fee (Je,\,
~ -1:; J ["" ." J 
(f . (J'f'l'
It.'f Gè<f 
Gradient et Laplacien d'une fonction scalaire 


Définition des déformations 

E. 	= ~(_-\_ 'GA... ;-'0 ..u.~ _ ..t.l'f) 
"''1' ~ Il..bi.t.te "If f} '" 't, 

~ 'OJ.t'l. \ 
t'LEI:; ~('OAe ..u" 
$, ra 't, 11-+ '1, '06 ) Equations d'équilibre 

-1 Ô<i'iL9 ..\ 
= 0 't, ra e '" hun.6 
dcrl\.e ..l 009& ..\ 
--+ 
.. 0
+ --+ --'00"9'1' + i [(<tElS-O"'f'fJ ~e + ~ <J",e l tOi = 
'ô", 't. ô 6 't,ivn,e 'Of iL O<ïiL'f + ~ 'O<t~ .. À 'O(f'f'f' + :!. (:'<rlt.'f + ~ tJ: ~e) =0 
8'V ft, ()e 1\, bv..-6 'O'f ft, e'l' 	~'f 



